
\section{正合范畴与三角范畴拾遗}

\subsection{正合范畴}

短正合列 (下简称 ses) 是同调代数的基本研究对象, 正合范畴研究了一般加法范畴上的短正合列理论.

\begin{definition}\label{def:exact-category}
    (Quillen 的正合范畴). \textbf{正合范畴} $(\mathcal{C}, \mathcal{E})$ 的基本资料是加法范畴 $\mathcal{C}$ 与一族图 $\mathcal{E}$. $\mathcal{E}$ 中对象称为 \textbf{$\mathcal{C}$ 中的短正合列 (ses)}, 形如
    \begin{equation}
    0 \to X \xrightarrow i Y \xrightarrow p Z \to 0,
    \end{equation}
    其中 $i = \ker p$, 称 $i$ 为\textbf{容许单态射}; $p = \operatorname{cok} i$, 称 $p$ 为\textbf{容许满态射}. 正合范畴 $(\mathcal{C}, \mathcal{E})$ 需满足以下公理:
    \begin{enumerate}
        \item[EX0] 对任意对象 $X$, 恒等映射 $\mathrm{id}_X$ 是容许单态射;
        \item[EX0'] 对任意对象 $X$, 恒等映射 $\mathrm{id}_X$ 是容许满态射;
        \item[EX1] 容许单态射在复合下封闭;
        \item[EX1'] 容许满态射在复合下封闭;
        \item[EX2] 容许单态射 $X \to Y$ 对任意态射 $X \to Z$ 有推出, 且 $Z \to Y \sqcup_X Z$ 仍为容许单态射;
        \item[EX2'] 容许满态射 $B \to C$ 对任意态射 $A \to C$ 有拉回, 且 $B \times^ CA \to A$ 仍为容许满态射.
    \end{enumerate}
\end{definition}

\begin{remark}
	推出 (拉回) 所得的态射在相差个一个同构的意义下是唯一的. 因此, EX2 与 EX2' 表明所有同构都是容许单态射与容许满态射. 我们通常直接约定 $\mathcal{E}$ 对同构封闭.
\end{remark}

以下说明所有可裂 ses 都是正合范畴的 ses.

\begin{remark}
	实际上, EX2 与 EX2' 构造的交换方块是推出拉回方块. 更精细的表述见\Cref{thm:homotopy-pullback}.
\end{remark}

\begin{lemma}\label{lem:split-in-exact}
	$\binom {1}{0} : X \to Y$ 与 $(1 \ \ 0): X \oplus Y \to X$ 分别是容许单态射与容许满态射.
	\begin{proof}
		注意到以下两个方块均是推出拉回方块:
		\begin{equation}
			% https://q.uiver.app/#q=WzAsOCxbMCwwLCIwIl0sWzEsMCwiWCJdLFswLDEsIlkiXSxbMSwxLCJYIFxcb3BsdXMgWSJdLFszLDAsIlggXFxvcGx1cyBZIl0sWzQsMCwiWCJdLFszLDEsIlkiXSxbNCwxLCIwIl0sWzAsMV0sWzEsMywiXFxiaW5vbSAxMCJdLFsyLDMsIlxcYmlub20gMDEiXSxbMCwyXSxbNCw1LCIoMSBcXCBcXCAwKSJdLFs1LDddLFs0LDYsIigwIFxcIFxcIDEpIl0sWzYsN11d
\begin{tikzcd}[ampersand replacement=\&]
	0 \& X \&\& {X \oplus Y} \& X \\
	Y \& {X \oplus Y} \&\& Y \& 0
	\arrow[from=1-1, to=1-2]
	\arrow[from=1-1, to=2-1]
	\arrow["{\binom 10}", from=1-2, to=2-2]
	\arrow["{(1 \ \ 0)}", from=1-4, to=1-5]
	\arrow["{(0 \ \ 1)}", from=1-4, to=2-4]
	\arrow[from=1-5, to=2-5]
	\arrow["{\binom 01}", from=2-1, to=2-2]
	\arrow[from=2-4, to=2-5]
\end{tikzcd}.
		\end{equation}
	\end{proof}
\end{lemma}

\begin{lemma}
	$\mathcal{E}$ 对直和封闭. 此处的直和指二元双积 $\oplus$, 后不重复解释.
	\begin{proof}
		给定正合范畴的 ses $0 \to A \xrightarrow i B \xrightarrow p C \to 0$ 与 $0 \to D \xrightarrow{j} E \xrightarrow{q} F \to 0$, 下证明 $0 \to A \oplus D \xrightarrow {\binom{i \ \ 0}{0 \ \ j}} B \oplus E \xrightarrow {\binom{p \ \ 0}{0 \ \ 1}} C \oplus F \to 0$ 是正合范畴的 ses. 首先, 这一直和是范畴的正合列, 故只需证明其中的满态射是容许满态射. \Cref{lem:split-in-exact} 说明 $\binom 1 0$ 是容许单态射, EX1 说明 $\binom i 0 : A \to B \oplus E$ 与 $\binom 0 j : D \to C \oplus E$ 也是容许单态射. 相应地, $\binom {1 \ \ 0}{0\  \ q}$ 与 $\binom {p \ \ 0}{0\  \ 1}$ 是容许满态射. 由 EX1', $\binom {1 \ \ 0}{0\  \ q} \circ \binom {p \ \ 0}{0\  \ 1}$ 是容许满态射.
	\end{proof}
\end{lemma}

\begin{corollary}
	在承认剩余公理的情况下, 可以将 EX0 与 EX0' 等价替换为:
	\begin{itemize}
		\item[$\overline{\text{EX0}}$] 任意可裂 ses 都属于 $\mathcal{E}$.
	\end{itemize}
\end{corollary}

\begin{example}\label{ex:ext-small}
	有时假定正合范畴的扩张类是局部本质小的. 换言之, 对任意对象 $X,Z$, 形如 $0 \to X \to ? \to Z \to 0$ 的 ses 的同构类是一个集合. 正合范畴满足这一假定, 若以下任意一则条件成立,
	\begin{enumerate}
		\item $\mathcal{C}$ 有足够投射对象, 或足够内射对象. (由维数移动.)
		\item $\mathcal{C}$ 是本质小的, 即所有态射构成集合. (显然.)
		\item $\mathcal{C}$ 中对象可遗忘作集合, 态射可遗忘作映射, 且遗忘函子保持核 (例如, 遗忘是右伴随). 对任意基数 $\kappa$, 基数为 $\kappa$ 的对象构成本质小范畴. 模范畴复形范畴的常见正合结构满足这一假定.
		\begin{proof}
			给定 $X$ 与 $Z$, 则扩张项 $E$ 满足 $|E/X| \simeq |Z|$. 由于所有基数不超过 $|X| \cdot |Z|$ 的对象构成本质小范畴, 故所有形如 $0 \to X \to ? \to Z \to 0$ 的 ses 构成本质小范畴.
		\end{proof}
	\end{enumerate}
\end{example}

\begin{proposition}
    正合范畴的反范畴是正合范畴.
    \begin{proof}
        公理 EX$n$ (EX$n$') 的在反范畴中的表述是 EX$n$' (EX$n$), 故成立.
    \end{proof}
\end{proposition}

后文提及的``对偶可证''是基于反范畴的考量.

\begin{remark}
    正合范畴的一般理论见\cite{buhlerExactCategories2010a}与\cite{kellerDerivedCategoriesTheir1996a}, 一些较有意义计算与反例参阅\cite{buhlerElementaryConsiderationsExact}.
\end{remark}

正合范畴是特殊的外三角范畴. 由于正合范畴的部分主要结论 (如``六项长正合列'') 可从外三角范畴者退化得到, 此节无需提及这类结论. 以下几则正合范畴的重要结论被推广作外三角范畴定义的一部分, 应当给出证明.

第一则结论是, EX2 (EX2') 中的推出 (拉回) 方块给定了正合列的推出 (拉回).

\begin{theorem}\label{thm:homotopy-pullback-ex}
	EX2 中的推出方块定义了正合范畴中正合列的推出; 特别地, 这也是推出拉回方块.
	\begin{proof}
		取定正合范畴的 ses $0 \to A \xrightarrow i B \xrightarrow p C \to 0$, 作 $i$ 关于任意给定态射 $f : A \to A'$ 的推出. 依照泛性质取 $p'$ 使得下图交换:
		\begin{equation}
			% https://q.uiver.app/#q=WzAsOCxbMSwwLCJBIl0sWzIsMCwiQiJdLFswLDAsIjAiXSxbMSwxLCJBJyJdLFsyLDEsIkUiXSxbMywwLCJDIl0sWzQsMCwiMCJdLFszLDEsIkMiXSxbMiwwXSxbMCwxLCJpIl0sWzMsNCwiaSciXSxbMSw1LCJwIl0sWzUsNl0sWzAsMywiZiJdLFsxLDQsImYnIl0sWzUsNywiIiwwLHsibGV2ZWwiOjIsInN0eWxlIjp7ImhlYWQiOnsibmFtZSI6Im5vbmUifX19XSxbMyw3LCIwIiwyLHsib2Zmc2V0IjoyLCJjdXJ2ZSI6MX1dLFs0LDcsInAnIiwwLHsic3R5bGUiOnsiYm9keSI6eyJuYW1lIjoiZGFzaGVkIn19fV1d
\begin{tikzcd}[ampersand replacement=\&]
	0 \& A \& B \& C \& 0 \\
	\& {A'} \& E \& C
	\arrow[from=1-1, to=1-2]
	\arrow["i", from=1-2, to=1-3]
	\arrow["f", from=1-2, to=2-2]
	\arrow["p", from=1-3, to=1-4]
	\arrow["{f'}", from=1-3, to=2-3]
	\arrow[from=1-4, to=1-5]
	\arrow[equals, from=1-4, to=2-4]
	\arrow["{i'}", from=2-2, to=2-3]
	\arrow["0"', shift right=2, curve={height=6pt}, from=2-2, to=2-4]
	\arrow["{p'}", dashed, from=2-3, to=2-4]
\end{tikzcd}.
		\end{equation}
		下证明 $p' = \operatorname{cok} i'$. 依照泛性质定义, 任取定 $\varphi : A' \to E$ 使得 $\varphi \circ i' = 0$. 由 $p = \operatorname{cok} i$, 存在 $\psi$ 使得 $\psi \circ p = \varphi \circ f'$:
		\begin{equation}
			% https://q.uiver.app/#q=WzAsOCxbMSwwLCJBIl0sWzIsMCwiQiJdLFswLDAsIjAiXSxbMSwxLCJBJyJdLFsyLDEsIkUiXSxbMywwLCJDIl0sWzQsMCwiMCJdLFsyLDIsIkYiXSxbMiwwXSxbMCwxLCJpIl0sWzMsNCwiaSciLDAseyJjb2xvdXIiOlszNTksMTAwLDYwXX0sWzM1OSwxMDAsNjAsMV1dLFsxLDUsInAiXSxbNSw2XSxbMCwzLCJmIl0sWzEsNCwiZiciXSxbNCw3LCJcXHZhcnBoaSAiLDAseyJjb2xvdXIiOlszNTksMTAwLDYwXX0sWzM1OSwxMDAsNjAsMV1dLFs0LDUsInAnIiwyLHsic3R5bGUiOnsiYm9keSI6eyJuYW1lIjoiZGFzaGVkIn19fV0sWzMsNywiMCIsMix7ImNvbG91ciI6WzM1OSwxMDAsNjBdfSxbMzU5LDEwMCw2MCwxXV0sWzUsNywiXFxwc2kiLDAseyJjdXJ2ZSI6LTIsImNvbG91ciI6WzM1OSwxMDAsNjBdLCJzdHlsZSI6eyJib2R5Ijp7Im5hbWUiOiJkYXNoZWQifX19LFszNTksMTAwLDYwLDFdXV0=
\begin{tikzcd}[ampersand replacement=\&]
	0 \& A \& B \& C \& 0 \\
	\& {A'} \& E \\
	\&\& F
	\arrow[from=1-1, to=1-2]
	\arrow["i", from=1-2, to=1-3]
	\arrow["f", from=1-2, to=2-2]
	\arrow["p", from=1-3, to=1-4]
	\arrow["{f'}", from=1-3, to=2-3]
	\arrow[from=1-4, to=1-5]
	\arrow["\psi", color={rgb,255:red,255;green,51;blue,54}, curve={height=-12pt}, dashed, from=1-4, to=3-3]
	\arrow["{i'}", color={rgb,255:red,255;green,51;blue,54}, from=2-2, to=2-3]
	\arrow["0"', color={rgb,255:red,255;green,51;blue,54}, from=2-2, to=3-3]
	\arrow["{p'}"', dashed, from=2-3, to=1-4]
	\arrow["{\varphi }", color={rgb,255:red,255;green,51;blue,54}, from=2-3, to=3-3]
\end{tikzcd}.
		\end{equation}
		依照推出的泛性质, $(i' \ \ f'): A' \oplus B \to E$ 是满态射. 由 $(\varphi) \circ (i' \ \ f') = (\psi \circ p') \circ (i' \ \ f')$, 得 $\varphi = \psi \circ p'$. 这说明``被 $i'$ 零化的 $\varphi$ 唯一决定了 $\psi$'', 换言之, $p'$ 是泛性质决定的 $i'$ 的余核.
		\\
		最后说明以上方块是拉回. 给定 $x$ 与 $y$ 使得 $f' \circ y = i' \circ x$, 如下如图所示:
		\begin{equation}
			% https://q.uiver.app/#q=WzAsMTEsWzEsMSwiQSJdLFsyLDEsIkIiXSxbMCwxLCIwIl0sWzEsMiwiQSciXSxbMiwyLCJFIl0sWzMsMSwiQyJdLFs0LDEsIjAiXSxbMywyLCJDIl0sWzAsMiwiMCJdLFs0LDIsIjAiXSxbMCwwLCJNIl0sWzIsMF0sWzAsMSwiaSJdLFszLDQsImknIiwwLHsiY29sb3VyIjpbMzU5LDEwMCw2MF19LFszNTksMTAwLDYwLDFdXSxbMSw1LCJwIl0sWzUsNl0sWzAsMywiZiJdLFsxLDQsImYnIiwwLHsiY29sb3VyIjpbMzU5LDEwMCw2MF19LFszNTksMTAwLDYwLDFdXSxbNSw3LCIiLDAseyJsZXZlbCI6Miwic3R5bGUiOnsiaGVhZCI6eyJuYW1lIjoibm9uZSJ9fX1dLFs0LDcsInAnIl0sWzgsM10sWzcsOV0sWzEwLDMsIngiLDIseyJjdXJ2ZSI6MiwiY29sb3VyIjpbMzU5LDEwMCw2MF19LFszNTksMTAwLDYwLDFdXSxbMTAsMSwieSIsMCx7ImN1cnZlIjotMiwiY29sb3VyIjpbMzU5LDEwMCw2MF19LFszNTksMTAwLDYwLDFdXSxbMTAsMCwicyIsMix7ImNvbG91ciI6WzM1OSwxMDAsNjBdLCJzdHlsZSI6eyJib2R5Ijp7Im5hbWUiOiJkYXNoZWQifX19LFszNTksMTAwLDYwLDFdXV0=
\begin{tikzcd}[ampersand replacement=\&]
	M \\
	0 \& A \& B \& C \& 0 \\
	0 \& {A'} \& E \& C \& 0
	\arrow["s"', color={rgb,255:red,255;green,51;blue,54}, dashed, from=1-1, to=2-2]
	\arrow["y", color={rgb,255:red,255;green,51;blue,54}, curve={height=-12pt}, from=1-1, to=2-3]
	\arrow["x"', color={rgb,255:red,255;green,51;blue,54}, curve={height=12pt}, from=1-1, to=3-2]
	\arrow[from=2-1, to=2-2]
	\arrow["i", from=2-2, to=2-3]
	\arrow["f", from=2-2, to=3-2]
	\arrow["p", from=2-3, to=2-4]
	\arrow["{f'}", color={rgb,255:red,255;green,51;blue,54}, from=2-3, to=3-3]
	\arrow[from=2-4, to=2-5]
	\arrow[equals, from=2-4, to=3-4]
	\arrow[from=3-1, to=3-2]
	\arrow["{i'}", color={rgb,255:red,255;green,51;blue,54}, from=3-2, to=3-3]
	\arrow["{p'}", from=3-3, to=3-4]
	\arrow[from=3-4, to=3-5]
\end{tikzcd}.
		\end{equation}
		由 $p \circ y = 0$, 则存在唯一的 $s$ 使得 $i \circ s = y$. 同时有等式 $f \circ s = x$, 两侧复合单态射 $i'$ 即可. $s$ 的取法也是唯一的; 若不然, 则有 $i \circ s' = y = i \circ s$, 消去单态射 $i$ 得到 $s=s'$ (矛盾).
	\end{proof}
\end{theorem}

\begin{corollary}
	$\mathrm{Ext}^1(Z,-)$ 与 $\mathrm{Ext}^1(-, X)$ 都是加法函子.
\end{corollary}

\begin{corollary}
	EX2 给出给出了配对
	\begin{equation}
		\mathrm{Ext}^1(Z,X) \times (X,X') \to \mathrm{Ext}^1(Z,X'), \quad (\theta, f) \mapsto \theta^\sharp (f).
	\end{equation}
	具体地, 如下图所示:
	\begin{equation}\label{eq:ext1-pairing}
		% https://q.uiver.app/#q=WzAsMTIsWzEsMCwiMCJdLFsyLDAsIlgiXSxbMywwLCJFIl0sWzQsMCwiWiJdLFs1LDAsIjAiXSxbNCwxLCJaIl0sWzUsMSwiMCJdLFsxLDEsIjAiXSxbMiwxLCJYJyJdLFszLDEsIkYiXSxbMCwxLCJcXHRoZXRhXlxcc2hhcnAgKGYpIl0sWzAsMCwiXFx0aGV0YSJdLFswLDFdLFsxLDJdLFsyLDNdLFszLDRdLFszLDUsIiIsMCx7ImxldmVsIjoyLCJzdHlsZSI6eyJoZWFkIjp7Im5hbWUiOiJub25lIn19fV0sWzUsNl0sWzcsOF0sWzgsOSwiIiwwLHsic3R5bGUiOnsiYm9keSI6eyJuYW1lIjoiZGFzaGVkIn19fV0sWzksNSwiIiwwLHsic3R5bGUiOnsiYm9keSI6eyJuYW1lIjoiZGFzaGVkIn19fV0sWzIsOSwiIiwxLHsic3R5bGUiOnsiYm9keSI6eyJuYW1lIjoiZGFzaGVkIn19fV0sWzEsOCwiZiIsMl0sWzEsOSwiXFx0ZXh0e1BPfSIsMSx7InN0eWxlIjp7ImJvZHkiOnsibmFtZSI6Im5vbmUifSwiaGVhZCI6eyJuYW1lIjoibm9uZSJ9fX1dLFsxMSwxMCwiIiwyLHsic3R5bGUiOnsidGFpbCI6eyJuYW1lIjoibWFwcyB0byJ9LCJib2R5Ijp7Im5hbWUiOiJkYXNoZWQifX19XSxbMjIsMTAsIiIsMCx7InNob3J0ZW4iOnsic291cmNlIjoyMH0sImxldmVsIjoxLCJzdHlsZSI6eyJ0YWlsIjp7Im5hbWUiOiJtYXBzIHRvIn0sImJvZHkiOnsibmFtZSI6ImRhc2hlZCJ9fX1dXQ==
\begin{tikzcd}
	\theta & 0 & X & E & Z & 0 \\
	{\theta^\sharp (f)} & 0 & {X'} & F & Z & 0
	\arrow[dashed, maps to, from=1-1, to=2-1]
	\arrow[from=1-2, to=1-3]
	\arrow[from=1-3, to=1-4]
	\arrow[""{name=0, anchor=center, inner sep=0}, "f"', from=1-3, to=2-3]
	\arrow["{\text{PO}}"{description}, draw=none, from=1-3, to=2-4]
	\arrow[from=1-4, to=1-5]
	\arrow[dashed, from=1-4, to=2-4]
	\arrow[from=1-5, to=1-6]
	\arrow[equals, from=1-5, to=2-5]
	\arrow[from=2-2, to=2-3]
	\arrow[dashed, from=2-3, to=2-4]
	\arrow[dashed, from=2-4, to=2-5]
	\arrow[from=2-5, to=2-6]
	\arrow[between={0.2}{1}, dashed, maps to, from=0, to=2-1]
\end{tikzcd}.
	\end{equation}
	我们希望这一配对是自然的.
\end{corollary}

\begin{lemma}
	\Cref{eq:ext1-pairing} 定义的配对关于 $(X,X')$ 项是自然的. 换言之, $\theta^\sharp$ 是自然变换.
	\begin{proof}
		假定 $\varphi : X' \to X''$. 依照定义, $\theta^\sharp (\varphi \circ f)$ 是正合列 $\theta^\sharp (f)$ 关于 $\varphi$ 的拉回, 从而 $\mathrm{Ext}^1(Z, \varphi) (\theta^\sharp (f)) = \theta^\sharp (\varphi \circ f)$. 更清楚地说, 以下是交换图:
		\begin{equation}
			% https://q.uiver.app/#q=WzAsOCxbMSwwLCIoWCwgWCcpIl0sWzIsMCwiXFxtYXRocm17RXh0fV4xKFosIFgnKSJdLFsxLDEsIihYLFgnJykiXSxbMiwxLCJcXG1hdGhybXtFeHR9XjEoWiwgWCcnKSJdLFswLDAsImYiXSxbMCwxLCJcXHZhcnBoaSBcXGNpcmMgZiJdLFszLDAsIlxcdGhldGFeXFxzaGFycCAoZikiXSxbMywxLCJcXHRoZXRhXlxcc2hhcnAgKFxcdmFycGhpIFxcY2lyYyBmKSJdLFswLDEsIlxcdGhldGFeXFxzaGFycCJdLFswLDIsIihYLFxcdmFycGhpKSIsMl0sWzIsMywiXFx0aGV0YV5cXHNoYXJwIiwyXSxbMSwzLCJcXG1hdGhybXtFeHR9XjEoWiwgXFx2YXJwaGkpIl0sWzQsNSwiIiwwLHsic3R5bGUiOnsidGFpbCI6eyJuYW1lIjoibWFwcyB0byJ9fX1dLFs2LDcsIiIsMCx7InN0eWxlIjp7InRhaWwiOnsibmFtZSI6Im1hcHMgdG8ifSwiYm9keSI6eyJuYW1lIjoiZGFzaGVkIn19fV1d
\begin{tikzcd}[ampersand replacement=\&]
	f \& {(X, X')} \& {\mathrm{Ext}^1(Z, X')} \& {\theta^\sharp (f)} \\
	{\varphi \circ f} \& {(X,X'')} \& {\mathrm{Ext}^1(Z, X'')} \& {\theta^\sharp (\varphi \circ f)}
	\arrow[maps to, from=1-1, to=2-1]
	\arrow["{\theta^\sharp}", from=1-2, to=1-3]
	\arrow["{(X,\varphi)}"', from=1-2, to=2-2]
	\arrow["{\mathrm{Ext}^1(Z, \varphi)}", from=1-3, to=2-3]
	\arrow[dashed, maps to, from=1-4, to=2-4]
	\arrow["{\theta^\sharp}"', from=2-2, to=2-3]
\end{tikzcd}.
		\end{equation}
	\end{proof}
\end{lemma}

第二则结论是 $\mathrm{Ext}^1$ 具有双函子性.

\begin{theorem}\label{thm:ext1-bifunctor}
    假定\Cref{ex:ext-small}. $\mathrm{Ext}^1 : \mathcal{C}^{\mathrm{op}}\times \mathcal{C} \to \mathbf{Ab}$ 将一组对象 $(Z,X)$ 对应至``所有形如 $0 \to X \to \ ? \to Z \to 0$ 的 ses'' 的同构类. 对任意态射 $f : X \to X'$, $\mathrm{Ext}^1(Z,f)$ 将短正合列 $\theta$ 对应至 $\theta'$, 机理是
    \begin{equation}
        % https://q.uiver.app/#q=WzAsMTIsWzEsMCwiMCJdLFsyLDAsIlgiXSxbMywwLCJFIl0sWzQsMCwiWiJdLFs1LDAsIjAiXSxbNCwxLCJaIl0sWzUsMSwiMCJdLFsxLDEsIjAiXSxbMiwxLCJYJyJdLFszLDEsIkYiXSxbMCwwLCJcXHRoZXRhIl0sWzAsMSwiXFx0aGV0YSciXSxbMCwxXSxbMSwyXSxbMiwzXSxbMyw0XSxbMyw1LCIiLDAseyJsZXZlbCI6Miwic3R5bGUiOnsiaGVhZCI6eyJuYW1lIjoibm9uZSJ9fX1dLFs1LDZdLFs3LDhdLFs4LDksIiIsMCx7InN0eWxlIjp7ImJvZHkiOnsibmFtZSI6ImRhc2hlZCJ9fX1dLFs5LDUsIiIsMCx7InN0eWxlIjp7ImJvZHkiOnsibmFtZSI6ImRhc2hlZCJ9fX1dLFsyLDksIiIsMSx7InN0eWxlIjp7ImJvZHkiOnsibmFtZSI6ImRhc2hlZCJ9fX1dLFsxLDgsImYiLDJdLFsxLDksIlxcdGV4dHtQT30iLDEseyJzdHlsZSI6eyJib2R5Ijp7Im5hbWUiOiJub25lIn0sImhlYWQiOnsibmFtZSI6Im5vbmUifX19XSxbMTAsMTEsIlxcbWF0aHJte0V4dH1eMShmLFopIiwyLHsic3R5bGUiOnsidGFpbCI6eyJuYW1lIjoibWFwcyB0byJ9LCJib2R5Ijp7Im5hbWUiOiJkYXNoZWQifX19XV0=
\begin{tikzcd}[ampersand replacement=\&]
	\theta \& 0 \& X \& E \& Z \& 0 \\
	{\theta'} \& 0 \& {X'} \& F \& Z \& 0
	\arrow["{\mathrm{Ext}^1(f,Z)}"', dashed, maps to, from=1-1, to=2-1]
	\arrow[from=1-2, to=1-3]
	\arrow[from=1-3, to=1-4]
	\arrow["f"', from=1-3, to=2-3]
	\arrow["{\text{PO}}"{description}, draw=none, from=1-3, to=2-4]
	\arrow[from=1-4, to=1-5]
	\arrow[dashed, from=1-4, to=2-4]
	\arrow[from=1-5, to=1-6]
	\arrow[equals, from=1-5, to=2-5]
	\arrow[from=2-2, to=2-3]
	\arrow[dashed, from=2-3, to=2-4]
	\arrow[dashed, from=2-4, to=2-5]
	\arrow[from=2-5, to=2-6]
\end{tikzcd}.
    \end{equation}
    对 $g' : Z \to Z'$, 可以通过拉回对偶地定义 $\mathrm{Ext}^1(g', X) : \mathrm{Ext}^1(Z', X) \to \mathrm{Ext}^1(Z, X)$. 今断言, 这是加法双函子, 其加法结构与 Baer 和匹配.
    \begin{proof}
        扩张的一般理论参阅\cite{mitchellTheoryCategories1965}章节 VII. 注意: Mitchell 通过 Baer 和证明 $\mathrm{Ext}^1$ 的双函子性; 为消解循环论证的疑虑, 以下直截了当地说明双函子性, 即证明如下引理.
        \begin{quoting}
            \begin{lemma}
            对 ses $\tau : 0 \to X \to M \to Z' \to 0$, 态射 $f : X \to X'$ 与 $g : Z \to Z'$, 总有
            \begin{equation}
                \mathrm{Ext}^1(g, X')(\mathrm{Ext}^1(Z', f)(\tau)) = \mathrm{Ext}^1(Z, f)(\mathrm{Ext}^1(g, X)(\tau)).
            \end{equation}
            方便起见, 将上式记作 $g^\ast f_\ast \tau = f_\ast g^\ast \tau$.
        \end{lemma}
        \end{quoting}
        以下证明 $f_\ast : g^\ast \tau \mapsto g^\ast f_\ast \tau$, 即, 存在 $\varphi : G \to F$ 使得下图中 $b'' = x' \circ \varphi$, 且 $\star$ 是推出:
        \begin{equation}
% https://q.uiver.app/#q=WzAsMzAsWzAsMiwiXFx0YXUiXSxbMSwyLCIwIl0sWzIsMiwiWCJdLFszLDIsIk0iXSxbNCwyLCJaJyJdLFs1LDIsIjAiXSxbMiwzLCJYJyJdLFsxLDMsIjAiXSxbMCwzLCJmX1xcYXN0IFxcdGF1Il0sWzAsNCwiZ15cXGFzdCBmX1xcYXN0IFxcdGF1Il0sWzEsNCwiMCJdLFsyLDQsIlgnIl0sWzQsMywiWiciXSxbNCw0LCJaIl0sWzUsMywiMCJdLFs1LDQsIjAiXSxbMywzLCJFIl0sWzMsNCwiRiJdLFsyLDEsIlgiXSxbMCwxLCJnXlxcYXN0IFxcdGF1ICJdLFsxLDAsIjAiXSxbMSwxLCIwIl0sWzMsMSwiRyJdLFs0LDEsIloiXSxbMiwwLCJYJyJdLFs1LDEsIjAiXSxbNSwwLCIwIl0sWzQsMCwiWiJdLFszLDAsIkYiXSxbMCwwLCJnXlxcYXN0IGZfXFxhc3QgXFx0YXUiXSxbMSwyXSxbMiwzLCJhIl0sWzMsNCwiYiJdLFs0LDVdLFsyLDYsImYiXSxbNyw2XSxbNiwxNiwiYSciXSxbMTYsMTIsIngiLDAseyJjb2xvdXIiOlszNTcsMTAwLDYwXX0sWzM1NywxMDAsNjAsMV1dLFsxMiwxNF0sWzQsMTIsIiIsMCx7ImxldmVsIjoyLCJzdHlsZSI6eyJoZWFkIjp7Im5hbWUiOiJub25lIn19fV0sWzYsMTEsIiIsMCx7ImxldmVsIjoyLCJzdHlsZSI6eyJoZWFkIjp7Im5hbWUiOiJub25lIn19fV0sWzMsMTYsImYnIiwwLHsiY29sb3VyIjpbMzU3LDEwMCw2MF19LFszNTcsMTAwLDYwLDFdXSxbMTcsMTYsImcnIiwyXSxbMTAsMTFdLFsxMSwxNywieSJdLFsxNywxMywieCciXSxbMTMsMTVdLFsxMywxMiwiZyIsMl0sWzIsMTYsIlxcdGV4dHtQT30iLDEseyJzdHlsZSI6eyJib2R5Ijp7Im5hbWUiOiJub25lIn0sImhlYWQiOnsibmFtZSI6Im5vbmUifX19XSxbMTYsMTMsIlxcdGV4dHtQQn0iLDEseyJzdHlsZSI6eyJib2R5Ijp7Im5hbWUiOiJub25lIn0sImhlYWQiOnsibmFtZSI6Im5vbmUifX19XSxbMjEsMThdLFsxOCwyMiwieiJdLFsyMiwyMywiYicnIiwwLHsiY29sb3VyIjpbMjM0LDEwMCw2MF19LFsyMzQsMTAwLDYwLDFdXSxbMjMsMjVdLFsxOCwyLCIiLDAseyJsZXZlbCI6Miwic3R5bGUiOnsiaGVhZCI6eyJuYW1lIjoibm9uZSJ9fX1dLFsyMyw0LCJnIiwwLHsiY29sb3VyIjpbMjM0LDEwMCw2MF19LFsyMzQsMTAwLDYwLDFdXSxbMjIsMywiZycnIiwwLHsiY29sb3VyIjpbMzU3LDEwMCw2MF19LFszNTcsMTAwLDYwLDFdXSxbMjcsMjMsIiIsMCx7ImxldmVsIjoyLCJzdHlsZSI6eyJoZWFkIjp7Im5hbWUiOiJub25lIn19fV0sWzI0LDI4LCJ5Il0sWzIwLDI0XSxbMjgsMjcsIngnIl0sWzI3LDI2XSxbMTgsMjQsImYiLDJdLFsyMiwyOCwiXFx2YXJwaGkgIiwyLHsic3R5bGUiOnsiYm9keSI6eyJuYW1lIjoiZGFzaGVkIn19fV0sWzAsOCwiIiwyLHsic3R5bGUiOnsidGFpbCI6eyJuYW1lIjoibWFwcyB0byJ9fX1dLFs4LDksIiIsMix7InN0eWxlIjp7InRhaWwiOnsibmFtZSI6Im1hcHMgdG8ifX19XSxbMCwxOSwiIiwwLHsic3R5bGUiOnsidGFpbCI6eyJuYW1lIjoibWFwcyB0byJ9fX1dLFsxOSwyOSwiIiwwLHsic3R5bGUiOnsidGFpbCI6eyJuYW1lIjoibWFwcyB0byJ9fX1dLFsyMiw0LCJcXHRleHR7UEJ9IiwxLHsic3R5bGUiOnsiYm9keSI6eyJuYW1lIjoibm9uZSJ9LCJoZWFkIjp7Im5hbWUiOiJub25lIn19fV0sWzI0LDIyLCJcXHN0YXIiLDEseyJzdHlsZSI6eyJib2R5Ijp7Im5hbWUiOiJub25lIn0sImhlYWQiOnsibmFtZSI6Im5vbmUifX19XV0=
\begin{tikzcd}[ampersand replacement=\&]
	{g^\ast f_\ast \tau} \& 0 \& {X'} \& F \& Z \& 0 \\
	{g^\ast \tau } \& 0 \& X \& G \& Z \& 0 \\
	\tau \& 0 \& X \& M \& {Z'} \& 0 \\
	{f_\ast \tau} \& 0 \& {X'} \& E \& {Z'} \& 0 \\
	{g^\ast f_\ast \tau} \& 0 \& {X'} \& F \& Z \& 0
	\arrow[from=1-2, to=1-3]
	\arrow["y", from=1-3, to=1-4]
	\arrow["\star"{description}, draw=none, from=1-3, to=2-4]
	\arrow["{x'}", from=1-4, to=1-5]
	\arrow[from=1-5, to=1-6]
	\arrow[equals, from=1-5, to=2-5]
	\arrow[maps to, from=2-1, to=1-1]
	\arrow[from=2-2, to=2-3]
	\arrow["f"', from=2-3, to=1-3]
	\arrow["z", from=2-3, to=2-4]
	\arrow[equals, from=2-3, to=3-3]
	\arrow["{\varphi }"', dashed, from=2-4, to=1-4]
	\arrow["{b''}", color={rgb,255:red,51;green,71;blue,255}, from=2-4, to=2-5]
	\arrow["{g''}", color={rgb,255:red,255;green,51;blue,61}, from=2-4, to=3-4]
	\arrow["{\text{PB}}"{description}, draw=none, from=2-4, to=3-5]
	\arrow[from=2-5, to=2-6]
	\arrow["g", color={rgb,255:red,51;green,71;blue,255}, from=2-5, to=3-5]
	\arrow[maps to, from=3-1, to=2-1]
	\arrow[maps to, from=3-1, to=4-1]
	\arrow[from=3-2, to=3-3]
	\arrow["a", from=3-3, to=3-4]
	\arrow["f", from=3-3, to=4-3]
	\arrow["{\text{PO}}"{description}, draw=none, from=3-3, to=4-4]
	\arrow["b", from=3-4, to=3-5]
	\arrow["{f'}", color={rgb,255:red,255;green,51;blue,61}, from=3-4, to=4-4]
	\arrow[from=3-5, to=3-6]
	\arrow[equals, from=3-5, to=4-5]
	\arrow[maps to, from=4-1, to=5-1]
	\arrow[from=4-2, to=4-3]
	\arrow["{a'}", from=4-3, to=4-4]
	\arrow[equals, from=4-3, to=5-3]
	\arrow["x", color={rgb,255:red,255;green,51;blue,61}, from=4-4, to=4-5]
	\arrow["{\text{PB}}"{description}, draw=none, from=4-4, to=5-5]
	\arrow[from=4-5, to=4-6]
	\arrow[from=5-2, to=5-3]
	\arrow["y", from=5-3, to=5-4]
	\arrow["{g'}"', from=5-4, to=4-4]
	\arrow["{x'}", from=5-4, to=5-5]
	\arrow["g"', from=5-5, to=4-5]
	\arrow[from=5-5, to=5-6]
\end{tikzcd}.
        \end{equation}
        $\varphi$ 由红, 蓝两组复合态射及右下方的拉回方块决定. 两组复合态射即
        \begin{equation}
            G \xrightarrow{f' \circ g''} E \xrightarrow{x} Z',\quad G \xrightarrow{b''} Z \xrightarrow{g} Z'.
        \end{equation}
        由拉回的泛性质, 存在唯一的 $\varphi$ 使得 $x' \circ \varphi = b''$ 且 $f' \circ g'' = g' \circ \varphi$. 因此右上方块交换. 为说明 $\star$ 的交换性, 只需验证 $y \circ f$ 与 $\varphi \circ z$ 均是如下拉回问题的解 $\mathfrak X$:
        \begin{equation}
            x \circ g' \circ \mathfrak X = g \circ x' \circ \mathfrak X.
        \end{equation}
        自行追图较阅读连等式更为快捷, 遂略去检验过程. 由 $\operatorname{cok} y \simeq \operatorname{cok} z$, 故 $\star$ 是推出.
    \end{proof}
\end{theorem}

\begin{remark}
	实际上, \Cref{thm:ext1-bifunctor} 的证明并不依赖于\Cref{ex:ext-small}. 如 \cite{mitchellTheoryCategories1965} 的章节 VII 所示, $\mathrm{Ext}^1(Z,X)$ 可以取作``Abel 类''. 若读者熟悉公理集合论, 大多数结论可推广至 ``Abel 类''.
	\\
	为规避过多技术性的细节, 本文一向承认\Cref{ex:ext-small}, 正如承认范畴的 $\mathrm{Hom}$-类是集合.
\end{remark}

第三则结论是 Noether 同构. 为精简记号, 通常作如下约定.

\begin{notation}
    使用 $\rightarrowtail$ ($\twoheadrightarrow$) 表示容许单态射 (容许满态射). 例如, 正合范畴的 ses 形如 $X \rightarrowtail Y \twoheadrightarrow Z$.
\end{notation}

\begin{theorem}\label{thm:noether-iso}
    假定容许单态射 $i$ 与 $j$ 可复合作 $j \circ i$, 则有三条 ses 作成的交换图:
    \begin{equation}
% https://q.uiver.app/#q=WzAsOCxbMCwwLCJYIl0sWzEsMCwiWSJdLFsyLDAsIloiXSxbMCwxLCJYIl0sWzEsMSwiQSJdLFsxLDIsIkIiXSxbMiwxLCJXIl0sWzIsMiwiQiJdLFswLDEsImkiLDAseyJzdHlsZSI6eyJ0YWlsIjp7Im5hbWUiOiJtb25vIn19fV0sWzEsMiwicCIsMCx7InN0eWxlIjp7ImhlYWQiOnsibmFtZSI6ImVwaSJ9fX1dLFsxLDQsImoiLDAseyJzdHlsZSI6eyJ0YWlsIjp7Im5hbWUiOiJtb25vIn19fV0sWzQsNSwicSIsMCx7InN0eWxlIjp7ImhlYWQiOnsibmFtZSI6ImVwaSJ9fX1dLFszLDQsImpcXGNpcmMgaSIsMCx7InN0eWxlIjp7InRhaWwiOnsibmFtZSI6Im1vbm8ifX19XSxbMCwzLCIiLDEseyJsZXZlbCI6Miwic3R5bGUiOnsiaGVhZCI6eyJuYW1lIjoibm9uZSJ9fX1dLFs1LDcsIiIsMCx7ImxldmVsIjoyLCJzdHlsZSI6eyJoZWFkIjp7Im5hbWUiOiJub25lIn19fV0sWzQsNiwiciIsMCx7InN0eWxlIjp7ImhlYWQiOnsibmFtZSI6ImVwaSJ9fX1dLFs2LDcsInEnIiwwLHsic3R5bGUiOnsiYm9keSI6eyJuYW1lIjoiZGFzaGVkIn0sImhlYWQiOnsibmFtZSI6ImVwaSJ9fX1dLFsyLDYsImonIiwwLHsic3R5bGUiOnsidGFpbCI6eyJuYW1lIjoibW9ubyJ9LCJib2R5Ijp7Im5hbWUiOiJkYXNoZWQifX19XV0=
\begin{tikzcd}
	X & Y & Z \\
	X & A & W \\
	& B & B
	\arrow["i", tail, from=1-1, to=1-2]
	\arrow[equals, from=1-1, to=2-1]
	\arrow["p", two heads, from=1-2, to=1-3]
	\arrow["j", tail, from=1-2, to=2-2]
	\arrow["{j'}", dashed, tail, from=1-3, to=2-3]
	\arrow["{j\circ i}", tail, from=2-1, to=2-2]
	\arrow["r", two heads, from=2-2, to=2-3]
	\arrow["q", two heads, from=2-2, to=3-2]
	\arrow["{q'}", dashed, two heads, from=2-3, to=3-3]
	\arrow[equals, from=3-2, to=3-3]
\end{tikzcd}.
    \end{equation}
    余核的反性质诱导了 $j'$ 与 $q'$. 今断言 $Z \overset{j'}\rightarrowtail W \overset{q'}\twoheadrightarrow B$ 也是 ses.
	\begin{proof}
		先作 $Y \rightarrowtail A \twoheadrightarrow B$ 关于 $p$ 的推出. 依照\Cref{thm:homotopy-pullback-ex} 的对偶表述, $\star$ 是推出拉回. 此时, 两个推出方块依照同构 $\varphi$ 分解:
		\begin{equation}
			% https://q.uiver.app/#q=WzAsOSxbMCwwLCJYIl0sWzEsMCwiWSJdLFsyLDAsIloiXSxbMCwxLCJYIl0sWzEsMSwiQSJdLFsxLDIsIkIiXSxbNCwyLCJXIl0sWzIsMiwiQiJdLFsyLDEsIlxcb3ZlcmxpbmUgVyIsWzM1OSwxMDAsNjAsMV1dLFswLDEsImkiLDAseyJzdHlsZSI6eyJ0YWlsIjp7Im5hbWUiOiJtb25vIn19fV0sWzEsMiwicCIsMCx7InN0eWxlIjp7ImhlYWQiOnsibmFtZSI6ImVwaSJ9fX1dLFsxLDQsImoiLDAseyJzdHlsZSI6eyJ0YWlsIjp7Im5hbWUiOiJtb25vIn19fV0sWzQsNSwicSIsMCx7InN0eWxlIjp7ImhlYWQiOnsibmFtZSI6ImVwaSJ9fX1dLFszLDQsImpcXGNpcmMgaSIsMCx7InN0eWxlIjp7InRhaWwiOnsibmFtZSI6Im1vbm8ifX19XSxbMCwzLCIiLDEseyJsZXZlbCI6Miwic3R5bGUiOnsiaGVhZCI6eyJuYW1lIjoibm9uZSJ9fX1dLFs1LDcsIiIsMCx7ImxldmVsIjoyLCJzdHlsZSI6eyJoZWFkIjp7Im5hbWUiOiJub25lIn19fV0sWzQsNiwiciIsMix7InN0eWxlIjp7ImhlYWQiOnsibmFtZSI6ImVwaSJ9fX1dLFs2LDcsInEnIiwwLHsic3R5bGUiOnsiYm9keSI6eyJuYW1lIjoiZGFzaGVkIn19fV0sWzIsNiwiaiciLDAseyJzdHlsZSI6eyJib2R5Ijp7Im5hbWUiOiJkYXNoZWQifX19XSxbNCw4LCJcXG92ZXJsaW5lIHtyfSIsMCx7ImNvbG91ciI6WzM1OSwxMDAsNjBdfSxbMzU5LDEwMCw2MCwxXV0sWzIsOCwiXFxvdmVybGluZSB7aid9IiwyLHsiY29sb3VyIjpbMzU5LDEwMCw2MF0sInN0eWxlIjp7InRhaWwiOnsibmFtZSI6Im1vbm8ifX19LFszNTksMTAwLDYwLDFdXSxbOCw3LCJcXG92ZXJsaW5lIHtxJ30iLDIseyJjb2xvdXIiOlszNTksMTAwLDYwXSwic3R5bGUiOnsiaGVhZCI6eyJuYW1lIjoiZXBpIn19fSxbMzU5LDEwMCw2MCwxXV0sWzEsOCwiXFxzdGFyIiwxLHsic3R5bGUiOnsiYm9keSI6eyJuYW1lIjoibm9uZSJ9LCJoZWFkIjp7Im5hbWUiOiJub25lIn19fV0sWzYsOCwiXFx2YXJwaGkgIiwxLHsiY29sb3VyIjpbMjI5LDEwMCw2MF0sInN0eWxlIjp7ImJvZHkiOnsibmFtZSI6ImRhc2hlZCJ9fX0sWzIyOSwxMDAsNjAsMV1dXQ==
\begin{tikzcd}
	X & Y & Z \\
	X & A & \textcolor{rgb,255:red,255;green,51;blue,54}{{\overline W}} \\
	& B & B && W
	\arrow["i", tail, from=1-1, to=1-2]
	\arrow[equals, from=1-1, to=2-1]
	\arrow["p", two heads, from=1-2, to=1-3]
	\arrow["j", tail, from=1-2, to=2-2]
	\arrow["\star"{description}, draw=none, from=1-2, to=2-3]
	\arrow["{\overline {j'}}"', color={rgb,255:red,255;green,51;blue,54}, tail, from=1-3, to=2-3]
	\arrow["{j'}", dashed, from=1-3, to=3-5]
	\arrow["{j\circ i}", tail, from=2-1, to=2-2]
	\arrow["{\overline {r}}", color={rgb,255:red,255;green,51;blue,54}, from=2-2, to=2-3]
	\arrow["q", two heads, from=2-2, to=3-2]
	\arrow["r"', two heads, from=2-2, to=3-5]
	\arrow["{\overline {q'}}"', color={rgb,255:red,255;green,51;blue,54}, two heads, from=2-3, to=3-3]
	\arrow[equals, from=3-2, to=3-3]
	\arrow["{\varphi }"{description}, color={rgb,255:red,51;green,88;blue,255}, dashed, from=3-5, to=2-3]
	\arrow["{q'}", dashed, from=3-5, to=3-3]
\end{tikzcd}.
		\end{equation}
		最后说明满态射 $r$ 诱导的 $q'$ 满足 $\overline {q'} \circ \varphi = q'$. 这一等式在复合满态射 $r$ 后显然相等.
	\end{proof}
\end{theorem}

\begin{remark}
    若记上述 $Z := Y/X$, $W := A/X$, 则 Noether 同构可表述为 $\frac{A/X}{Y/X} \simeq \frac{A}{Y}$. 这是 Noether 同构的初始形态.
\end{remark}

\subsection{三角范畴}

外三角是正合范畴与三角范畴的共同推广. 试回顾三角范畴的定义.

\begin{definition}\label{def:triangulated-category}
    三角范畴由资料 $(\mathcal{C}, \Sigma, \mathcal{E})$ 描述. 其中,
    \begin{enumerate}
        \item $\mathcal{C}$ 是加法范畴,
        \item $\Sigma : \mathcal{C} \to \mathcal{C}$ 是范畴的自等价, 称作平移函子,
        \item $\mathcal{E} = \{X \xrightarrow u Y\xrightarrow vZ\xrightarrow w \Sigma X\}$ 是 $\mathcal{C}$ 中好三角组成的类.
    \end{enumerate}
    约定两则术语:
    \begin{enumerate}
        \item (三角射). 两个好三角间的态射描述作三元组 $(\alpha , \beta ,\gamma)$, 使得下图交换 (横行是好三角):
        \begin{equation}\label{eq:tri-morphism}
            % https://q.uiver.app/#q=WzAsOCxbMCwwLCJYIl0sWzEsMCwiWSJdLFsyLDAsIloiXSxbMywwLCJcXFNpZ21hIFgiXSxbMCwxLCJBIl0sWzEsMSwiQiJdLFsyLDEsIkMiXSxbMywxLCJcXFNpZ21hIEEiXSxbMCwxLCJ1Il0sWzEsMiwidiJdLFsyLDMsInciXSxbNCw1LCJpIl0sWzUsNiwiaiJdLFs2LDcsImsiXSxbMCw0LCJcXGFscGhhICJdLFsxLDUsIlxcYmV0YSJdLFsyLDYsIlxcZ2FtbWEiXSxbMyw3LCJcXFNpZ21hIFxcYWxwaGEgIl1d
\begin{tikzcd}[ampersand replacement=\&]
	X \& Y \& Z \& {\Sigma X} \\
	A \& B \& C \& {\Sigma A}
	\arrow["u", from=1-1, to=1-2]
	\arrow["{\alpha }", from=1-1, to=2-1]
	\arrow["v", from=1-2, to=1-3]
	\arrow["\beta", from=1-2, to=2-2]
	\arrow["w", from=1-3, to=1-4]
	\arrow["\gamma", from=1-3, to=2-3]
	\arrow["{\Sigma \alpha }", from=1-4, to=2-4]
	\arrow["i", from=2-1, to=2-2]
	\arrow["j", from=2-2, to=2-3]
	\arrow["k", from=2-3, to=2-4]
\end{tikzcd},
        \end{equation}
        \item (旋转). 下图中, 中行是好三角, 首(尾)行是其逆(顺)时针旋转:
        \begin{equation}\label{eq:tri-rotation}
% https://q.uiver.app/#q=WzAsMTQsWzAsMSwiWCJdLFsxLDEsIlkiXSxbMiwxLCJaIl0sWzMsMSwiXFxTaWdtYSBYIl0sWzAsMCwiXFxTaWdtYV57LTF9WiJdLFsxLDAsIlgiXSxbMiwwLCJZIl0sWzMsMCwiWiJdLFswLDIsIlkiXSxbMSwyLCJaIl0sWzIsMiwiXFxTaWdtYSBYIl0sWzMsMiwiXFxTaWdtYSBZIl0sWzQsMCwiXFx0ZXh0eyjpgIYpfSJdLFs0LDIsIlxcdGV4dHso6aG6KX0iXSxbMCwxLCJ1Il0sWzEsMiwidiJdLFsyLDMsInciXSxbNCw1LCItXFxTaWdtYV57LTF9IHciXSxbNSw2LCJ1Il0sWzYsNywidiJdLFs4LDksInYiXSxbOSwxMCwidyJdLFsxMCwxMSwiLVxcU2lnbWEgdSJdXQ==
\begin{tikzcd}
	{\Sigma^{-1}Z} & X & Y & Z & {\text{(逆)}} \\
	X & Y & Z & {\Sigma X} \\
	Y & Z & {\Sigma X} & {\Sigma Y} & {\text{(顺)}}
	\arrow["{-\Sigma^{-1} w}", from=1-1, to=1-2]
	\arrow["u", from=1-2, to=1-3]
	\arrow["v", from=1-3, to=1-4]
	\arrow["u", from=2-1, to=2-2]
	\arrow["v", from=2-2, to=2-3]
	\arrow["w", from=2-3, to=2-4]
	\arrow["v", from=3-1, to=3-2]
	\arrow["w", from=3-2, to=3-3]
	\arrow["{-\Sigma u}", from=3-3, to=3-4]
\end{tikzcd}.
        \end{equation}
    \end{enumerate}
    特别地, 好三角类满足以下公理:
    \begin{enumerate}
        \item[TR1-1] $\mathcal{E}$ 关于同构封闭. 具体地, 若\Cref{eq:tri-morphism} 是 $\mathcal{C}$ 中通常的交换图, 且 $\alpha, \beta, \gamma$ 都是同构, 则上行是好三角当且仅当下行是好三角;
        \item[TR1-2] 任意态射 $u$ 可嵌入某一好三角 $X \xrightarrow u Y \xrightarrow v Z \xrightarrow w \Sigma X$;
        \item[TR1-3] 对任意对象 $X$, 平凡三角 $0 \to X \xrightarrow{\mathrm{id}_X} X \to 0$ 是好三角;
        \item[TR2] 好三角关于顺时针旋转封闭;
        \item[TR3] 假定\Cref{eq:tri-morphism} 的上行与下行是好三角, 若图中仅存在 $\alpha$ 与 $\beta$ 使得 $\beta \circ u = i \circ \alpha$, 则一定存在某一 $\gamma$ 使得上图交换;
        \item[TR4] 若下图中 $r_1$, $r_2$ 与 $c_2$ 均为好三角, 则存在虚线所示的好三角 $c_3$ 使得所有方块交换:
        \begin{equation}
% https://q.uiver.app/#q=WzAsMTcsWzEsMSwiWCJdLFsyLDEsIlkiXSxbMywxLCJXIl0sWzQsMSwiXFxTaWdtYSBYIl0sWzEsMiwiWCJdLFsyLDIsIloiXSxbMywyLCJSIl0sWzQsMiwiXFxTaWdtYSBYIl0sWzIsMywiUyJdLFszLDMsIlMiXSxbMiw0LCJcXFNpZ21hIFkiXSxbMyw0LCJcXFNpZ21hIFciXSxbMCwxLCJyXzEiXSxbMCwyLCJyXzIiXSxbMiwwLCJjXzIiXSxbMywwLCJjXzMiXSxbNCwzLCJcXFNpZ21hIFkiXSxbMCwxLCJ1Il0sWzEsMiwidiJdLFsyLDMsInciXSxbNCw1LCJ4Il0sWzUsNiwieSJdLFs2LDcsInoiXSxbMCw0LCIiLDAseyJsZXZlbCI6Miwic3R5bGUiOnsiaGVhZCI6eyJuYW1lIjoibm9uZSJ9fX1dLFszLDcsIiIsMCx7ImxldmVsIjoyLCJzdHlsZSI6eyJoZWFkIjp7Im5hbWUiOiJub25lIn19fV0sWzEsNSwiYSJdLFs1LDgsImIiXSxbMiw2LCIiLDAseyJzdHlsZSI6eyJib2R5Ijp7Im5hbWUiOiJkYXNoZWQifX19XSxbNiw5LCIiLDAseyJzdHlsZSI6eyJib2R5Ijp7Im5hbWUiOiJkYXNoZWQifX19XSxbOSwxMSwiIiwwLHsic3R5bGUiOnsiYm9keSI6eyJuYW1lIjoiZGFzaGVkIn19fV0sWzgsMTAsImMiXSxbMTAsMTEsIlxcU2lnbWEgdiIsMCx7ImNvbG91ciI6WzM1MiwxMDAsNjBdfSxbMzUyLDEwMCw2MCwxXV0sWzgsOSwiIiwwLHsibGV2ZWwiOjIsInN0eWxlIjp7ImhlYWQiOnsibmFtZSI6Im5vbmUifX19XSxbOSwxNiwiYyJdLFs3LDE2LCJcXFNpZ21hIHUiLDAseyJjb2xvdXIiOlszNTIsMTAwLDYwXX0sWzM1MiwxMDAsNjAsMV1dXQ==
\begin{tikzcd}
	&& {c_2} & {c_3} \\[-12pt]
	{r_1} & X & Y & W & {\Sigma X} \\
	{r_2} & X & Z & R & {\Sigma X} \\
	&& S & S & {\Sigma Y} \\
	&& {\Sigma Y} & {\Sigma W}
	\arrow["u", from=2-2, to=2-3]
	\arrow[equals, from=2-2, to=3-2]
	\arrow["v", from=2-3, to=2-4]
	\arrow["a", from=2-3, to=3-3]
	\arrow["w", from=2-4, to=2-5]
	\arrow[dashed, from=2-4, to=3-4]
	\arrow[equals, from=2-5, to=3-5]
	\arrow["x", from=3-2, to=3-3]
	\arrow["y", from=3-3, to=3-4]
	\arrow["b", from=3-3, to=4-3]
	\arrow["z", from=3-4, to=3-5]
	\arrow[dashed, from=3-4, to=4-4]
	\arrow["{\Sigma u}", color={rgb,255:red,255;green,51;blue,78}, from=3-5, to=4-5]
	\arrow[equals, from=4-3, to=4-4]
	\arrow["c", from=4-3, to=5-3]
	\arrow["c", from=4-4, to=4-5]
	\arrow[dashed, from=4-4, to=5-4]
	\arrow["{\Sigma v}", color={rgb,255:red,255;green,51;blue,78}, from=5-3, to=5-4]
\end{tikzcd}.
        \end{equation}
    \end{enumerate}
\end{definition}

\begin{example}
    以下是一些注意事项.
    \begin{enumerate}
        \item $\Sigma$ 是范畴自等价, 这并不意味着 $\Sigma X \simeq X$.
        \item TR3 中补全的态射的方式未必唯一.
    \end{enumerate}
\end{example}

\begin{remark}
    关于三角范畴的定义见\cite{AST_1996__239__R1_0}或\cite{neemanTriangulatedCategories2001}. 更通俗的读物是\cite{TriangulatedDMurfet}或\cite{TriangulatedJPMay}. 特别地, \cite{TriangulatedJPMay} 指出 TR3 可由其余公理推出.
\end{remark}

以下从正合范畴的视角``粗浅地''解释三角范畴; 诚然, 更好的工具是外三角范畴.

\begin{example}
    给定好三角 $X \xrightarrow u Y \xrightarrow v Z \xrightarrow w \Sigma X$. 若将 $X \xrightarrow u Y \xrightarrow v Z$ 类比作 ses, 则 $w \in \mathrm{Ext}^1(Z,X)$ 描述了扩张元. 特别地, $\mathrm{Ext}(Z,X) := (Z,\Sigma X)$. TR1-1, TR1-2, TR3 类似核与余核泛性质的推论. TR1-3 即 Ex0. TR4 类似 Noether 同构.
\end{example}

\begin{proposition}\label{prop:triangulated-facts}
    以下是预三角范畴的直接推论 (无需 TR4). 证明细节见通常的教材.
    \begin{enumerate}
		\item (TR3 的加强). TR3 即是说, 给定\Cref{eq:tri-morphism} 中的 $\alpha$ 与 $\beta$ 使得图交换, 则有 $\gamma$ 使得图交换. 实际上, 若给定 $\alpha$, $\beta$, $\gamma$ 中任意两者使得图交换, 则一定存在第三者使得图交换.
		\begin{proof}
			依照 TR2 与自然同构 $(\Sigma (?), \Sigma(-)) \simeq (?, -)$ 即得.
		\end{proof}
		\item (长正合列). 对任意好三角 $X \xrightarrow u Y \xrightarrow v Z \xrightarrow w \Sigma X$, 有上同调函子的长正合列
        \begin{equation}
            \cdots \to (-, \Sigma^{-1} Z) \to (-, X) \to (-, Y) \to (-, Z) \to (-, \Sigma X) \to \cdots,
        \end{equation}
        以及同调函子的长正合列
        \begin{equation}
            \cdots \to (\Sigma X, -) \to (Z, -) \to (Y, -) \to (X, -) \to (\Sigma^{-1} Z, -) \to \cdots.
        \end{equation}
        \item (长正合列的推论). 若\Cref{eq:tri-morphism} 中 $\alpha$, $\beta$ 与 $\gamma$ 中有两者为同构, 则第三者也是同构.
		\begin{proof}
			以上同调函子为例, 下证明对任意 $A$, 总有 $\ker (A,v) = \operatorname{im} (A,u)$. 给定 $\beta : A \to Y$,
			\begin{equation}
				% https://q.uiver.app/#q=WzAsOCxbMCwxLCJYIl0sWzEsMSwiWSJdLFsyLDEsIloiXSxbMywxLCJcXFNpZ21hIFgiXSxbMCwwLCJBIl0sWzEsMCwiQSJdLFsyLDAsIjAiXSxbMywwLCJcXFNpZ21hIEEiXSxbMCwxLCJ1IiwyXSxbMSwyLCJ2IiwyXSxbMiwzLCJ3IiwyXSxbNCwwLCJcXGFscGhhIiwwLHsic3R5bGUiOnsiYm9keSI6eyJuYW1lIjoiZGFzaGVkIn19fV0sWzQsNSwiIiwyLHsibGV2ZWwiOjIsInN0eWxlIjp7ImhlYWQiOnsibmFtZSI6Im5vbmUifX19XSxbNSw2XSxbNiw3XSxbNSwxLCJcXGJldGEiXSxbNiwyLCJcXGdhbW1hIiwwLHsic3R5bGUiOnsiYm9keSI6eyJuYW1lIjoiZGFzaGVkIn19fV0sWzcsMywiXFxTaWdtYSBcXGFscGhhIiwwLHsic3R5bGUiOnsiYm9keSI6eyJuYW1lIjoiZGFzaGVkIn19fV1d
\begin{tikzcd}[ampersand replacement=\&]
	A \& A \& 0 \& {\Sigma A} \\
	X \& Y \& Z \& {\Sigma X}
	\arrow[equals, from=1-1, to=1-2]
	\arrow["\alpha", dashed, from=1-1, to=2-1]
	\arrow[from=1-2, to=1-3]
	\arrow["\beta", from=1-2, to=2-2]
	\arrow[from=1-3, to=1-4]
	\arrow["\gamma", dashed, from=1-3, to=2-3]
	\arrow["{\Sigma \alpha}", dashed, from=1-4, to=2-4]
	\arrow["u"', from=2-1, to=2-2]
	\arrow["v"', from=2-2, to=2-3]
	\arrow["w"', from=2-3, to=2-4]
\end{tikzcd}.
			\end{equation}
			若存在 $\alpha$ 使得图交换, 则存在 $\gamma$ 使得图交换. 因此 $\ker (A,v) \supseteq \operatorname{im} (A,u)$. 倒置 $\gamma$ 与 $\alpha$ 的选取顺序, 得 $\ker (A,v) \subseteq \operatorname{im} (A,u)$, 从正合列在 $(-,Y)$ 处正合. 该正合列通过 TR2 右延, 通过 TR2 以及自然同构 $(\Sigma (?), \Sigma(-)) \simeq (?, -)$ 左延.
		\end{proof}
		\item 好三角的直和也是好三角;
		\begin{proof}
			对 $i  = 1,2$, 记好三角 $X_i \xrightarrow{u_i} Y_i \xrightarrow{v_i} Z_i \xrightarrow{w_i} \Sigma X_i$. 记直和嵌入的好三角 $X_1 \oplus X_2 \xrightarrow {u_1 \oplus u_2} Y_1 \oplus Y_2 \xrightarrow {(a \ b)} Z_0 \xrightarrow {\binom c d} \Sigma (X_1 \oplus X_2)$. 由直和的泛性质构造交换图
			\begin{equation}
				% https://q.uiver.app/#q=WzAsOCxbMCwwLCJYXzFcXG9wbHVzIFhfMiJdLFsxLDAsIllfMVxcb3BsdXMgWV8yIl0sWzIsMCwiWl8xIFxcb3BsdXMgWl8yIl0sWzMsMCwiXFxTaWdtYSBYXzFcXG9wbHVzXFxTaWdtYSBYXzIiXSxbMywxLCJcXFNpZ21hIFhfMVxcb3BsdXNcXFNpZ21hIFhfMiJdLFswLDEsIlhfMVxcb3BsdXMgWF8yIl0sWzEsMSwiWV8xXFxvcGx1cyBZXzIiXSxbMiwxLCJaXzAiXSxbMSwyLCJ2XzFcXG9wbHVzIHZfMiJdLFsyLDMsIndfMVxcb3BsdXMgd18yIl0sWzAsNSwiIiwwLHsibGV2ZWwiOjIsInN0eWxlIjp7ImhlYWQiOnsibmFtZSI6Im5vbmUifX19XSxbNSw2LCJ1XzFcXG9wbHVzIHVfMiJdLFs2LDcsIihhIFxcIGIpIl0sWzcsNCwiXFxiaW5vbSBjIGQiXSxbMyw0LCIiLDAseyJsZXZlbCI6Miwic3R5bGUiOnsiaGVhZCI6eyJuYW1lIjoibm9uZSJ9fX1dLFswLDEsInVfMVxcb3BsdXMgdV8yIl0sWzEsNiwiIiwwLHsibGV2ZWwiOjIsInN0eWxlIjp7ImhlYWQiOnsibmFtZSI6Im5vbmUifX19XSxbMiw3LCIoaSBcXCBqKSJdXQ==
\begin{tikzcd}[ampersand replacement=\&]
	{X_1\oplus X_2} \& {Y_1\oplus Y_2} \& {Z_1 \oplus Z_2} \& {\Sigma X_1\oplus\Sigma X_2} \\
	{X_1\oplus X_2} \& {Y_1\oplus Y_2} \& {Z_0} \& {\Sigma X_1\oplus\Sigma X_2}
	\arrow["{u_1\oplus u_2}", from=1-1, to=1-2]
	\arrow[equals, from=1-1, to=2-1]
	\arrow["{v_1\oplus v_2}", from=1-2, to=1-3]
	\arrow[equals, from=1-2, to=2-2]
	\arrow["{w_1\oplus w_2}", from=1-3, to=1-4]
	\arrow["{(i \ j)}", from=1-3, to=2-3]
	\arrow[equals, from=1-4, to=2-4]
	\arrow["{u_1\oplus u_2}", from=2-1, to=2-2]
	\arrow["{(a \ b)}", from=2-2, to=2-3]
	\arrow["{\binom c d}", from=2-3, to=2-4]
\end{tikzcd}.
			\end{equation}
			对所有对象与态射取米田嵌入 $E \mapsto (-, E)$, 上下两行均是长正合列. 由五引理, $(-, (i\ j))$ 是自然同构. 由米田引理, $(i\ j)$ 是同构.
		\end{proof}
		\item (TR3 的特殊情形). 沿用交换图\Cref{eq:tri-morphism}. 若 $\alpha$, $\beta$ 与 $\gamma$ 两者为同构, 则第三者必为同构.
		\begin{proof}
			对所有对象与态射取米田嵌入 $E \mapsto (-, E)$. 由长正合列与五引理, $(-,\alpha)$, $(-, \beta)$ 与 $(-,\gamma)$ 均是自然同构. 最后由依照米田引理完成证明.
		\end{proof}
		\item (TR1-2 的加强). 任意态射可以以任意位置嵌入某一好三角; 若选定该位置, 则其嵌入的好三角在同构意义下唯一.
		\begin{proof}
			依照 TR2 与自然同构 $(\Sigma(?), \Sigma(-)) \simeq (?, -)$, 态射的嵌入位置可以任意选定. 由上一条``TR3 的特殊情形'', 态射嵌入的好三角在同构意义下唯一.
		\end{proof}
		\item (TR2 的类似结论). 好三角关于逆时针旋转封闭. 顺(逆)时针旋转定义作\Cref{eq:tri-rotation}.
		\begin{proof}
			选定好三角 $X \xrightarrow u Y \xrightarrow v Z \xrightarrow w \Sigma X$. 下图上行是 $-\Sigma^{-1}w$ 嵌入的好三角, 下行是逆时针旋转:
			\begin{equation}
				% https://q.uiver.app/#q=WzAsOCxbMCwwLCJcXFNpZ21hXnstMX1aIl0sWzEsMCwiWCJdLFsyLDAsIlknIl0sWzMsMCwiWiJdLFszLDEsIloiXSxbMiwxLCJZIl0sWzEsMSwiWCJdLFswLDEsIlxcU2lnbWFeey0xfVoiXSxbMCwxLCItXFxTaWdtYV57LTF9IHciXSxbMSwyLCJ1JyJdLFsyLDMsInYnIl0sWzUsNCwidiJdLFs2LDUsInUiXSxbMSw2LCIiLDAseyJsZXZlbCI6Miwic3R5bGUiOnsiaGVhZCI6eyJuYW1lIjoibm9uZSJ9fX1dLFsyLDUsIlxcdmFycGhpIiwwLHsic3R5bGUiOnsiYm9keSI6eyJuYW1lIjoiZGFzaGVkIn19fV0sWzMsNCwiIiwwLHsibGV2ZWwiOjIsInN0eWxlIjp7ImhlYWQiOnsibmFtZSI6Im5vbmUifX19XSxbMCw3LCIiLDAseyJsZXZlbCI6Miwic3R5bGUiOnsiaGVhZCI6eyJuYW1lIjoibm9uZSJ9fX1dLFs3LDYsIi1cXFNpZ21hXnstMX0gdyJdXQ==
\begin{tikzcd}
	{\Sigma^{-1}Z} & X & {Y'} & Z \\
	{\Sigma^{-1}Z} & X & Y & Z
	\arrow["{-\Sigma^{-1} w}", from=1-1, to=1-2]
	\arrow[equals, from=1-1, to=2-1]
	\arrow["{u'}", from=1-2, to=1-3]
	\arrow[equals, from=1-2, to=2-2]
	\arrow["{v'}", from=1-3, to=1-4]
	\arrow["\varphi", dashed, from=1-3, to=2-3]
	\arrow[equals, from=1-4, to=2-4]
	\arrow["{-\Sigma^{-1} w}", from=2-1, to=2-2]
	\arrow["u", from=2-2, to=2-3]
	\arrow["v", from=2-3, to=2-4]
\end{tikzcd}.
			\end{equation}
			下解释 $\varphi$ 的选取方式. 对上下两行作顺时针旋转, 得好三角的交换图,  由``TR3 的加强''可得 $\varphi$ 使得图交换. 由``TR3 的特殊情形'', $\varphi$ 是同构.
		\end{proof}
    \end{enumerate}
\end{proposition}

\begin{proposition}\label{prop:triangulated-kernel-cokernel}
	对预三角范畴 (无需 TR4) 中的态射 $f$, 以下命题等价:
	\begin{enumerate}
		\item $f$ 存在核;
		\item $f$ 存在余核;
		\item $f$ 在前后复合同构的意义下形如 $\binom{1 \ \ 0}{0 \ \ 0} : X \oplus A \to Y \oplus A$.
	\end{enumerate}
	仅需证明 ($1 \to 3$). 反方向是显然的. 对 ($2\leftrightarrow 3$) 的证明是对偶的.
	\begin{proof}
		记 $i : K \to U$ 是 $f : U \to V$ 的核. 记好三角 $R \xrightarrow r K \xrightarrow i U \xrightarrow{s} \Sigma R$, 由长正合列知 $\operatorname{im}(-, r) = \ker(-, i) = 0$. 从而 $r$ 是零态射. 下图说明 $i$ 是可裂单态射:
		\begin{equation}
			% https://q.uiver.app/#q=WzAsOCxbMSwwLCJLIl0sWzIsMCwiVSJdLFsyLDEsIksiXSxbMSwxLCJLIl0sWzMsMCwiUiJdLFswLDAsIlIiXSxbMCwxLCIwIl0sWzMsMSwiMCJdLFswLDEsImkiLDAseyJzdHlsZSI6eyJ0YWlsIjp7Im5hbWUiOiJtb25vIn19fV0sWzAsMywiIiwwLHsibGV2ZWwiOjIsInN0eWxlIjp7ImhlYWQiOnsibmFtZSI6Im5vbmUifX19XSxbMywyLCIiLDAseyJsZXZlbCI6Miwic3R5bGUiOnsiaGVhZCI6eyJuYW1lIjoibm9uZSJ9fX1dLFsxLDQsInMiXSxbNSwwLCJyIl0sWzUsNl0sWzYsM10sWzIsN10sWzQsN10sWzEsMiwicCIsMCx7InN0eWxlIjp7ImJvZHkiOnsibmFtZSI6ImRhc2hlZCJ9fX1dXQ==
\begin{tikzcd}[ampersand replacement=\&]
	R \& K \& U \& R \\
	0 \& K \& K \& 0
	\arrow["r", from=1-1, to=1-2]
	\arrow[from=1-1, to=2-1]
	\arrow["i", tail, from=1-2, to=1-3]
	\arrow[equals, from=1-2, to=2-2]
	\arrow["s", from=1-3, to=1-4]
	\arrow["p", dashed, from=1-3, to=2-3]
	\arrow[from=1-4, to=2-4]
	\arrow[from=2-1, to=2-2]
	\arrow[equals, from=2-2, to=2-3]
	\arrow[from=2-3, to=2-4]
\end{tikzcd}.
		\end{equation}
		将 $i$ 嵌入好三角 $K \xrightarrow i U \xrightarrow q C \xrightarrow z \Sigma K$. 由``TR3 的特殊情形'', 得同构的好三角
		\begin{equation}
			% https://q.uiver.app/#q=WzAsOCxbMCwwLCJLIl0sWzEsMCwiVSJdLFsxLDEsIktcXG9wbHVzIEMiXSxbMiwwLCJDIl0sWzIsMSwiQyJdLFswLDEsIksiXSxbMywwLCJcXFNpZ21hIEsiXSxbMywxLCJcXFNpZ21hIEsiXSxbMCwxLCJpIiwwLHsic3R5bGUiOnsidGFpbCI6eyJuYW1lIjoibW9ubyJ9fX1dLFsxLDMsInMiXSxbMiw0LCIoMCBcXCAxKSJdLFszLDZdLFs2LDcsIiIsMCx7ImxldmVsIjoyLCJzdHlsZSI6eyJoZWFkIjp7Im5hbWUiOiJub25lIn19fV0sWzQsN10sWzUsMiwiXFxiaW5vbSAxMCJdLFswLDUsIiIsMCx7ImxldmVsIjoyLCJzdHlsZSI6eyJoZWFkIjp7Im5hbWUiOiJub25lIn19fV0sWzEsMiwiXFxiaW5vbSBwcyJdLFszLDQsIiIsMSx7ImxldmVsIjoyLCJzdHlsZSI6eyJoZWFkIjp7Im5hbWUiOiJub25lIn19fV1d
\begin{tikzcd}[ampersand replacement=\&]
	K \& U \& C \& {\Sigma K} \\
	K \& {K\oplus C} \& C \& {\Sigma K}
	\arrow["i", tail, from=1-1, to=1-2]
	\arrow[equals, from=1-1, to=2-1]
	\arrow["s", from=1-2, to=1-3]
	\arrow["{\binom ps}", from=1-2, to=2-2]
	\arrow[from=1-3, to=1-4]
	\arrow[equals, from=1-3, to=2-3]
	\arrow[equals, from=1-4, to=2-4]
	\arrow["{\binom 10}", from=2-1, to=2-2]
	\arrow["{(0 \ 1)}", from=2-2, to=2-3]
	\arrow[from=2-3, to=2-4]
\end{tikzcd}.
		\end{equation}
		这说明 $f$ 形如 $K \oplus C \xrightarrow {(0 \  \varphi)} V$, 其中 $\varphi$ 是单态射. 同样的论证说明 $\varphi$ 是可裂单, 且由直和关系 $V \simeq L \oplus D$. 此时, $f$ 形如 $\binom{1 \ \ 0}{0 \ \ 0} : C \oplus K \to D\oplus L$.
	\end{proof}
\end{proposition}

\begin{corollary}\label{cor:split-mono-epi}
	给定好三角 $X \xrightarrow u Y \xrightarrow v Z \xrightarrow w \Sigma X$, 则以下命题等价: (1) $u$ 是单态射; (2) $u$ 是可裂单态射; (3) $v$ 是满态射; (4) $v$ 是可裂满态射; $w = 0$.
\end{corollary}

\begin{lemma}
	特别地, 预三角范畴的所有可裂单(满)都有相应的直和项, 即\Cref{def:weakly-idempotent-complete} 定义的弱幂等完备性.
	\begin{proof}
		给定好三角 $X \xrightarrow u Y \xrightarrow v Z \xrightarrow w \Sigma X$, 其中 $v$ 是可裂单. \Cref{cor:split-mono-epi} 说明 $u = 0$. 可以补全以下三角射:
		\begin{equation}
			% https://q.uiver.app/#q=WzAsOCxbMCwxLCJYIl0sWzEsMSwiWSJdLFsyLDEsIloiXSxbMywxLCJcXFNpZ21hIFgiXSxbMCwwLCJYIl0sWzEsMCwiWSJdLFsyLDAsIlkgXFxvcGx1cyBcXFNpZ21hIFgiXSxbMywwLCJcXFNpZ21hIFgiXSxbMCwxLCIwIl0sWzEsMiwidiJdLFsyLDMsInciXSxbNCwwLCIiLDAseyJsZXZlbCI6Miwic3R5bGUiOnsiaGVhZCI6eyJuYW1lIjoibm9uZSJ9fX1dLFs1LDEsIiIsMCx7ImxldmVsIjoyLCJzdHlsZSI6eyJoZWFkIjp7Im5hbWUiOiJub25lIn19fV0sWzcsMywiIiwwLHsibGV2ZWwiOjIsInN0eWxlIjp7ImhlYWQiOnsibmFtZSI6Im5vbmUifX19XSxbNCw1LCIwIl0sWzUsNiwiXFxiaW5vbSAxIDAgIl0sWzYsNywiKDAgXFwgMSkiXSxbNiwyLCJcXHZhcnBoaSAiLDAseyJzdHlsZSI6eyJib2R5Ijp7Im5hbWUiOiJkYXNoZWQifX19XV0=
\begin{tikzcd}
	X & Y & {Y \oplus \Sigma X} & {\Sigma X} \\
	X & Y & Z & {\Sigma X}
	\arrow["0", from=1-1, to=1-2]
	\arrow[equals, from=1-1, to=2-1]
	\arrow["{\binom 1 0 }", from=1-2, to=1-3]
	\arrow[equals, from=1-2, to=2-2]
	\arrow["{(0 \ 1)}", from=1-3, to=1-4]
	\arrow["{\varphi }", dashed, from=1-3, to=2-3]
	\arrow[equals, from=1-4, to=2-4]
	\arrow["0", from=2-1, to=2-2]
	\arrow["v", from=2-2, to=2-3]
	\arrow["w", from=2-3, to=2-4]
\end{tikzcd}.
		\end{equation}
		由``TR3 的特殊情形'', $\varphi$ 是同构. 这说明 $\operatorname{cok} v = W$.
	\end{proof}
\end{lemma}

\subsection{同伦的推出拉回}

本节所述的``同伦的推出拉回''在外三角范畴的图表定理中大有作用. 以下给出一侧预加范畴中的观察.

\begin{example}
    给定预加范畴 $\mathcal{A}$ 中的交换方块(左), 以及其诱导的复形(右):
    \begin{equation}
        % https://q.uiver.app/#q=WzAsOSxbMCwwLCJYIl0sWzEsMCwiWSJdLFswLDEsIloiXSxbMSwxLCJXIl0sWzMsMSwiWCJdLFsyLDEsIjAiXSxbNCwxLCJZIFxcb3BsdXMgWiJdLFs1LDEsIlciXSxbNiwxLCIwIl0sWzAsMSwiZiJdLFsxLDMsImciXSxbMCwyLCJoIiwyXSxbMiwzLCJsIiwyXSxbNSw0XSxbNCw2LCJcXGJpbm9tIHtmfXtofSJdLFs2LDcsIihnIFxcIC1sKSJdLFs3LDhdXQ==
\begin{tikzcd}[ampersand replacement=\&]
	X \& Y \\
	Z \& W \& 0 \& X \& {Y \oplus Z} \& W \& 0
	\arrow["f", from=1-1, to=1-2]
	\arrow["h"', from=1-1, to=2-1]
	\arrow["g", from=1-2, to=2-2]
	\arrow["l"', from=2-1, to=2-2]
	\arrow[from=2-3, to=2-4]
	\arrow["{\binom {f}{h}}", from=2-4, to=2-5]
	\arrow["{(g \ -l)}", from=2-5, to=2-6]
	\arrow[from=2-6, to=2-7]
\end{tikzcd}.
    \end{equation}
    以下是一则转述定义即得的引理.
    \begin{quoting}
        \begin{lemma}
            左图是拉回(推出), 当且仅当右图左正合(右正合).
        \end{lemma}
    \end{quoting}
    称某交换方块是拉回(推出), 若相应的态射方程存在唯一的解. 倘若去除唯一性, 则称该交换方块是弱拉回(弱推出). 以下是一则转述定义即得的引理.
    \begin{quoting}
        \begin{lemma}
            左图是弱拉回, 当且仅当以下是函子的正合列:
            \begin{equation}
                (-, X) \xrightarrow{\binom{f}{h}_\ast } (-, Y) \oplus (-, Z) \xrightarrow{(g \ -l)_\ast } (-, W).
            \end{equation}
            左图是弱推出, 当且仅当以下是函子的正合列:
            \begin{equation}
                (W, -) \xrightarrow{(g \ -l)^\ast } (Y, -) \oplus (Z, -) \xrightarrow{\binom{f}{h}^\ast } (X, -).
            \end{equation}
        \end{lemma}
    \end{quoting}
\end{example}

\begin{proposition}
    由正合范畴的 Noether 同构, 容许单态射 $X \to Y$ 与容许满态射 $X \to Z$ 的推出是推出拉回方块. 特别地, 这一推出拉回方块诱导的 ses 是正合范畴的 ses.
    \begin{proof}
        只需证明, 若 $i: A \to B$ 是容许单态射, 则对任意 $f: A \to C$, $\binom{i}{f}$ 也是容许单态射. 以下证明一则更强的引理.
        \begin{quoting}
        \begin{lemma}\label{lem:inf-def}
            若 $q \circ p$ 是容许单态射, $q$ 是容许满态射, 则 $p$ 是容许单态射.
            \begin{proof}
                对态射 $q$ 与 $q \circ p$ 使用 Noether 同构, 得以下交换图:
                \begin{equation}
                    % https://q.uiver.app/#q=WzAsMTAsWzEsMiwiWCIsWzIzOCwxMDAsNjAsMV1dLFsyLDIsIlkiLFsyMzgsMTAwLDYwLDFdXSxbMywyLCJaIl0sWzIsMSwiRSIsWzIzOCwxMDAsNjAsMV1dLFsyLDAsIkEiXSxbNSwyLCJcXCwiXSxbMSwwLCJBIl0sWzEsMSwiRiJdLFswLDAsIlgiLFsyMzgsMTAwLDYwLDFdXSxbMywxLCJaIl0sWzAsMSwicSBcXGNpcmMgcCIsMCx7ImNvbG91ciI6WzIzOCwxMDAsNjBdLCJzdHlsZSI6eyJ0YWlsIjp7Im5hbWUiOiJtb25vIn19fSxbMjM4LDEwMCw2MCwxXV0sWzEsMiwiXFxwaSAiLDAseyJzdHlsZSI6eyJoZWFkIjp7Im5hbWUiOiJlcGkifX19XSxbNCwzLCJpIiwwLHsic3R5bGUiOnsidGFpbCI6eyJuYW1lIjoibW9ubyJ9fX1dLFszLDEsInEiLDAseyJjb2xvdXIiOlsyMzgsMTAwLDYwXSwic3R5bGUiOnsiaGVhZCI6eyJuYW1lIjoiZXBpIn19fSxbMjM4LDEwMCw2MCwxXV0sWzYsNCwiIiwwLHsibGV2ZWwiOjIsInN0eWxlIjp7ImhlYWQiOnsibmFtZSI6Im5vbmUifX19XSxbNiw3LCIiLDAseyJzdHlsZSI6eyJ0YWlsIjp7Im5hbWUiOiJtb25vIn19fV0sWzcsMywicCciLDAseyJzdHlsZSI6eyJ0YWlsIjp7Im5hbWUiOiJtb25vIn19fV0sWzcsMCwiXFxsYW1iZGEiLDAseyJzdHlsZSI6eyJoZWFkIjp7Im5hbWUiOiJlcGkifX19XSxbNywxLCJcXHNxdWFyZSIsMSx7InN0eWxlIjp7ImJvZHkiOnsibmFtZSI6Im5vbmUifSwiaGVhZCI6eyJuYW1lIjoibm9uZSJ9fX1dLFs4LDAsIjFfWCIsMCx7ImN1cnZlIjoyLCJjb2xvdXIiOlsyMzgsMTAwLDYwXX0sWzIzOCwxMDAsNjAsMV1dLFs4LDMsInAiLDAseyJsYWJlbF9wb3NpdGlvbiI6NzAsImN1cnZlIjotMSwiY29sb3VyIjpbMjM4LDEwMCw2MF19LFsyMzgsMTAwLDYwLDFdXSxbOCw3LCJzIiwwLHsic3R5bGUiOnsiYm9keSI6eyJuYW1lIjoiZGFzaGVkIn19fV0sWzMsOSwiIiwwLHsic3R5bGUiOnsiaGVhZCI6eyJuYW1lIjoiZXBpIn19fV0sWzksMiwiIiwwLHsibGV2ZWwiOjIsInN0eWxlIjp7ImhlYWQiOnsibmFtZSI6Im5vbmUifX19XV0=
\begin{tikzcd}
	\textcolor{rgb,255:red,51;green,58;blue,255}{X} & A & A \\
	& F & \textcolor{rgb,255:red,51;green,58;blue,255}{E} & Z \\
	& \textcolor{rgb,255:red,51;green,58;blue,255}{X} & \textcolor{rgb,255:red,51;green,58;blue,255}{Y} & Z && {\,}
	\arrow["s", dashed, from=1-1, to=2-2]
	\arrow["p"{pos=0.7}, color={rgb,255:red,51;green,58;blue,255}, curve={height=-6pt}, from=1-1, to=2-3]
	\arrow["{1_X}", color={rgb,255:red,51;green,58;blue,255}, curve={height=12pt}, from=1-1, to=3-2]
	\arrow[equals, from=1-2, to=1-3]
	\arrow[tail, from=1-2, to=2-2]
	\arrow["i", tail, from=1-3, to=2-3]
	\arrow["{p'}", tail, from=2-2, to=2-3]
	\arrow["\lambda", two heads, from=2-2, to=3-2]
	\arrow["\square"{description}, draw=none, from=2-2, to=3-3]
	\arrow[two heads, from=2-3, to=2-4]
	\arrow["q", color={rgb,255:red,51;green,58;blue,255}, two heads, from=2-3, to=3-3]
	\arrow[equals, from=2-4, to=3-4]
	\arrow["{q \circ p}", color={rgb,255:red,51;green,58;blue,255}, tail, from=3-2, to=3-3]
	\arrow["{\pi }", two heads, from=3-3, to=3-4]
\end{tikzcd}.
                \end{equation}
                最左纵列的短正合列可裂. 考虑可裂 ses $X \overset s \rightarrowtail F \twoheadrightarrow A$ 知 $s$ 是容许单态射. 由蓝线所示的拉回问题, 得 $p = p' \circ s$ 是容许单态射的复合, 故 $p$ 也是容许单态射.
            \end{proof}
        \end{lemma}
        \end{quoting}
        注意, 以上引理的证明未使用弱幂等完备的假定.
    \end{proof}
\end{proposition}

\begin{definition}
    (正合范畴中同伦的推出拉回). 称交换图是正合范畴中同伦的推出拉回方块, 若其诱导的链复形是正合范畴的 ses.
\end{definition}

\begin{example}
    将 Abel 范畴与所有 ses 作成正合范畴, 则推出拉回方块必然是同伦的推出拉回方块.
\end{example}

\begin{theorem}\label{thm:homotopy-pullback}
    (正合范畴的同伦推出拉回方块). 以下五类是正合范畴的同伦推出拉回方块:
    \begin{equation}
        % https://q.uiver.app/#q=WzAsMjgsWzAsMCwiXFxidWxsZXQiXSxbMSwwLCJcXGJ1bGxldCJdLFswLDEsIlxcYnVsbGV0Il0sWzEsMSwiXFxidWxsZXQiXSxbMiwwLCJcXGJ1bGxldCJdLFszLDAsIlxcYnVsbGV0Il0sWzMsMSwiXFxidWxsZXQiXSxbMiwxLCJcXGJ1bGxldCJdLFs0LDAsIlxcYnVsbGV0Il0sWzUsMCwiXFxidWxsZXQiXSxbNSwxLCJcXGJ1bGxldCJdLFs0LDEsIlxcYnVsbGV0Il0sWzYsMCwiXFxidWxsZXQiXSxbNywwLCJcXGJ1bGxldCJdLFs3LDEsIlxcYnVsbGV0Il0sWzYsMSwiXFxidWxsZXQiXSxbMSwyLCJcXGJ1bGxldCJdLFsyLDIsIlxcYnVsbGV0Il0sWzEsMywiXFxidWxsZXQiXSxbMiwzLCJcXGJ1bGxldCJdLFszLDIsIlxcYnVsbGV0Il0sWzMsMywiXFxidWxsZXQiXSxbNCwyLCJcXGJ1bGxldCJdLFs0LDMsIlxcYnVsbGV0Il0sWzUsMywiXFxidWxsZXQiXSxbNiwzLCJcXGJ1bGxldCJdLFs1LDIsIlxcYnVsbGV0Il0sWzYsMiwiXFxidWxsZXQiXSxbMCwxLCIiLDEseyJzdHlsZSI6eyJ0YWlsIjp7Im5hbWUiOiJtb25vIn19fV0sWzAsMiwiIiwxLHsic3R5bGUiOnsidGFpbCI6eyJuYW1lIjoibW9ubyJ9fX1dLFsyLDMsIiIsMSx7InN0eWxlIjp7InRhaWwiOnsibmFtZSI6Im1vbm8ifSwiYm9keSI6eyJuYW1lIjoiZGFzaGVkIn19fV0sWzEsMywiIiwxLHsic3R5bGUiOnsidGFpbCI6eyJuYW1lIjoibW9ubyJ9LCJib2R5Ijp7Im5hbWUiOiJkYXNoZWQifX19XSxbNCw1LCIiLDEseyJzdHlsZSI6eyJ0YWlsIjp7Im5hbWUiOiJtb25vIn19fV0sWzUsNiwiIiwxLHsic3R5bGUiOnsiYm9keSI6eyJuYW1lIjoiZGFzaGVkIn0sImhlYWQiOnsibmFtZSI6ImVwaSJ9fX1dLFs0LDcsIiIsMSx7InN0eWxlIjp7ImhlYWQiOnsibmFtZSI6ImVwaSJ9fX1dLFs3LDYsIiIsMSx7InN0eWxlIjp7InRhaWwiOnsibmFtZSI6Im1vbm8ifSwiYm9keSI6eyJuYW1lIjoiZGFzaGVkIn19fV0sWzgsOSwiIiwxLHsic3R5bGUiOnsidGFpbCI6eyJuYW1lIjoibW9ubyJ9LCJib2R5Ijp7Im5hbWUiOiJkYXNoZWQifX19XSxbOSwxMCwiIiwxLHsic3R5bGUiOnsiaGVhZCI6eyJuYW1lIjoiZXBpIn19fV0sWzgsMTEsIiIsMSx7InN0eWxlIjp7ImJvZHkiOnsibmFtZSI6ImRhc2hlZCJ9LCJoZWFkIjp7Im5hbWUiOiJlcGkifX19XSxbMTEsMTAsIiIsMSx7InN0eWxlIjp7InRhaWwiOnsibmFtZSI6Im1vbm8ifX19XSxbMTIsMTMsIiIsMSx7InN0eWxlIjp7ImJvZHkiOnsibmFtZSI6ImRhc2hlZCJ9LCJoZWFkIjp7Im5hbWUiOiJlcGkifX19XSxbMTMsMTQsIiIsMSx7InN0eWxlIjp7ImhlYWQiOnsibmFtZSI6ImVwaSJ9fX1dLFsxMiwxNSwiIiwxLHsic3R5bGUiOnsiYm9keSI6eyJuYW1lIjoiZGFzaGVkIn0sImhlYWQiOnsibmFtZSI6ImVwaSJ9fX1dLFsxNSwxNCwiIiwxLHsic3R5bGUiOnsiaGVhZCI6eyJuYW1lIjoiZXBpIn19fV0sWzE4LDE5LCIiLDEseyJzdHlsZSI6eyJ0YWlsIjp7Im5hbWUiOiJtb25vIn19fV0sWzE2LDE3LCIiLDEseyJzdHlsZSI6eyJ0YWlsIjp7Im5hbWUiOiJtb25vIn19fV0sWzE2LDE4LCIiLDEseyJsZXZlbCI6Miwic3R5bGUiOnsiaGVhZCI6eyJuYW1lIjoibm9uZSJ9fX1dLFsxNywxOV0sWzE3LDIwLCIiLDEseyJzdHlsZSI6eyJoZWFkIjp7Im5hbWUiOiJlcGkifX19XSxbMjAsMjEsIiIsMSx7InN0eWxlIjp7ImJvZHkiOnsibmFtZSI6ImRhc2hlZCJ9fX1dLFsxOSwyMSwiIiwxLHsic3R5bGUiOnsiaGVhZCI6eyJuYW1lIjoiZXBpIn19fV0sWzIyLDIzLCIiLDEseyJzdHlsZSI6eyJib2R5Ijp7Im5hbWUiOiJkYXNoZWQifX19XSxbMjMsMjQsIiIsMSx7InN0eWxlIjp7InRhaWwiOnsibmFtZSI6Im1vbm8ifX19XSxbMjQsMjUsIiIsMSx7InN0eWxlIjp7ImhlYWQiOnsibmFtZSI6ImVwaSJ9fX1dLFsyMiwyNiwiIiwxLHsic3R5bGUiOnsidGFpbCI6eyJuYW1lIjoibW9ubyJ9fX1dLFsyNiwyNywiIiwxLHsic3R5bGUiOnsiaGVhZCI6eyJuYW1lIjoiZXBpIn19fV0sWzI3LDI1LCIiLDEseyJsZXZlbCI6Miwic3R5bGUiOnsiaGVhZCI6eyJuYW1lIjoibm9uZSJ9fX1dLFsyNiwyNF0sWzAsMywiXFx0ZXh0e1BPfSIsMSx7InN0eWxlIjp7ImJvZHkiOnsibmFtZSI6Im5vbmUifSwiaGVhZCI6eyJuYW1lIjoibm9uZSJ9fX1dLFs0LDYsIlxcdGV4dHtQT30iLDEseyJzdHlsZSI6eyJib2R5Ijp7Im5hbWUiOiJub25lIn0sImhlYWQiOnsibmFtZSI6Im5vbmUifX19XSxbOCwxMCwiXFx0ZXh0e1BCfSIsMSx7InN0eWxlIjp7ImJvZHkiOnsibmFtZSI6Im5vbmUifSwiaGVhZCI6eyJuYW1lIjoibm9uZSJ9fX1dLFsxMiwxNCwiXFx0ZXh0e1BCfSIsMSx7InN0eWxlIjp7ImJvZHkiOnsibmFtZSI6Im5vbmUifSwiaGVhZCI6eyJuYW1lIjoibm9uZSJ9fX1dXQ==
\begin{tikzcd}[ampersand replacement=\&]
	\cdot \& \cdot \& \cdot \& \cdot \& \cdot \& \cdot \& \cdot \& \cdot \\
	\cdot \& \cdot \& \cdot \& \cdot \& \cdot \& \cdot \& \cdot \& \cdot \\
	\& \cdot \& \cdot \& \cdot \& \cdot \& \cdot \& \cdot \\
	\& \cdot \& \cdot \& \cdot \& \cdot \& \cdot \& \cdot
	\arrow[tail, from=1-1, to=1-2]
	\arrow[tail, from=1-1, to=2-1]
	\arrow["{\text{PO}}"{description}, draw=none, from=1-1, to=2-2]
	\arrow[dashed, tail, from=1-2, to=2-2]
	\arrow[tail, from=1-3, to=1-4]
	\arrow[two heads, from=1-3, to=2-3]
	\arrow["{\text{PO}}"{description}, draw=none, from=1-3, to=2-4]
	\arrow[dashed, two heads, from=1-4, to=2-4]
	\arrow[dashed, tail, from=1-5, to=1-6]
	\arrow[dashed, two heads, from=1-5, to=2-5]
	\arrow["{\text{PB}}"{description}, draw=none, from=1-5, to=2-6]
	\arrow[two heads, from=1-6, to=2-6]
	\arrow[dashed, two heads, from=1-7, to=1-8]
	\arrow[dashed, two heads, from=1-7, to=2-7]
	\arrow["{\text{PB}}"{description}, draw=none, from=1-7, to=2-8]
	\arrow[two heads, from=1-8, to=2-8]
	\arrow[dashed, tail, from=2-1, to=2-2]
	\arrow[dashed, tail, from=2-3, to=2-4]
	\arrow[tail, from=2-5, to=2-6]
	\arrow[two heads, from=2-7, to=2-8]
	\arrow[tail, from=3-2, to=3-3]
	\arrow[equals, from=3-2, to=4-2]
	\arrow[two heads, from=3-3, to=3-4]
	\arrow[from=3-3, to=4-3]
	\arrow[dashed, from=3-4, to=4-4]
	\arrow[tail, from=3-5, to=3-6]
	\arrow[dashed, from=3-5, to=4-5]
	\arrow[two heads, from=3-6, to=3-7]
	\arrow[from=3-6, to=4-6]
	\arrow[equals, from=3-7, to=4-7]
	\arrow[tail, from=4-2, to=4-3]
	\arrow[two heads, from=4-3, to=4-4]
	\arrow[tail, from=4-5, to=4-6]
	\arrow[two heads, from=4-6, to=4-7]
\end{tikzcd}.
    \end{equation}
    \begin{proof}
        使用引理: 若 $i$ 是容许单态射 ($p$ 是容许满态射), 则 $\binom{i}{?}$ 是容许单态射 ($(p \ ?)$ 是容许满态射).
    \end{proof}
\end{theorem}

\begin{definition}
	(三角范畴中同伦的推出拉回). 称交换图是三角范畴中同伦的推出拉回方块, 若其诱导的三项链复形是三角范畴中好三角前三项.
\end{definition}

满足 TR1-TR3 的加法范畴称作预三角范畴. 对预三角范畴, 下给出一则八面体公理的等价公理, 更多等价公理(不包括以下)可参阅\cite{neemanNewAxiomsTriangulated1991}.

\begin{theorem}
	预三角范畴是三角范畴, 当且仅当其满足如下等价公理.
	\begin{enumerate}
		\item 八面体公理.
		\item 给定红色处态射 $a_1$ 与 $b_1$, 则可以补全下图中的三个好三角与一处三角射:
		\begin{equation}
			% https://q.uiver.app/#q=WzAsMTIsWzAsMSwiWCIsWzM1OSwxMDAsNjAsMV1dLFsxLDEsIllfMiIsWzM1OSwxMDAsNjAsMV1dLFswLDIsIllfMSIsWzM1OSwxMDAsNjAsMV1dLFsxLDIsIloiXSxbMiwxLCJXIl0sWzAsMCwiWCJdLFsxLDAsIllfMiBcXG9wbHVzIFlfMSJdLFsyLDAsIloiXSxbMywwLCJcXFNpZ21hIFgiXSxbMywxLCJcXFNpZ21hIFgiLFsyMzYsMTAwLDYwLDFdXSxbMiwyLCJXIixbMjM2LDEwMCw2MCwxXV0sWzMsMiwiXFxTaWdtYSBZIl0sWzAsMiwiYl8xIiwwLHsiY29sb3VyIjpbMzU5LDEwMCw2MF19LFszNTksMTAwLDYwLDFdXSxbMCwxLCJhXzEiLDAseyJjb2xvdXIiOlszNTksMTAwLDYwXX0sWzM1OSwxMDAsNjAsMV1dLFsxLDMsImJfMiJdLFsxLDRdLFs1LDYsIlxcYmlub217YV8xfXtiXzF9Il0sWzYsNywiKGJfMSwtYV8yKSJdLFs3LDgsIlxcbXVcXGxhbWJkYSIsMCx7ImNvbG91ciI6WzIzNiwxMDAsNjBdfSxbMjM2LDEwMCw2MCwxXV0sWzQsMTAsIiIsMCx7ImxldmVsIjoyLCJzdHlsZSI6eyJoZWFkIjp7Im5hbWUiOiJub25lIn19fV0sWzQsOSwiXFxtdSIsMCx7ImNvbG91ciI6WzIzNiwxMDAsNjBdfSxbMjM2LDEwMCw2MCwxXV0sWzksMTEsIlxcU2lnbWEgYl8xIl0sWzIsMywiYV8yIl0sWzMsMTAsIlxcbGFtYmRhIiwyLHsiY29sb3VyIjpbMjM2LDEwMCw2MF19LFsyMzYsMTAwLDYwLDFdXSxbMTAsMTFdXQ==
\begin{tikzcd}
	X & {Y_2 \oplus Y_1} & Z & {\Sigma X} \\
	\textcolor{rgb,255:red,255;green,51;blue,54}{X} & \textcolor{rgb,255:red,255;green,51;blue,54}{{Y_2}} & W & \textcolor{rgb,255:red,51;green,65;blue,255}{{\Sigma X}} \\
	\textcolor{rgb,255:red,255;green,51;blue,54}{{Y_1}} & Z & \textcolor{rgb,255:red,51;green,65;blue,255}{W} & {\Sigma Y_1}
	\arrow["{\binom{a_1}{b_1}}", from=1-1, to=1-2]
	\arrow["{(b_2 \ -a_2)}", from=1-2, to=1-3]
	\arrow["{\mu\lambda}", color={rgb,255:red,51;green,65;blue,255}, from=1-3, to=1-4]
	\arrow["{a_1}", color={rgb,255:red,255;green,51;blue,54}, from=2-1, to=2-2]
	\arrow["{b_1}", color={rgb,255:red,255;green,51;blue,54}, from=2-1, to=3-1]
	\arrow[from=2-2, to=2-3]
	\arrow["{b_2}", from=2-2, to=3-2]
	\arrow["\mu", color={rgb,255:red,51;green,65;blue,255}, from=2-3, to=2-4]
	\arrow[equals, from=2-3, to=3-3]
	\arrow["{\Sigma b_1}", from=2-4, to=3-4]
	\arrow["{a_2}", from=3-1, to=3-2]
	\arrow["\lambda"', color={rgb,255:red,51;green,65;blue,255}, from=3-2, to=3-3]
	\arrow[from=3-3, to=3-4]
\end{tikzcd}
		\end{equation}
		\item 给定红色个好三角, 同伦推出拉回方块, 以及第一行的好三角. 此时存在 $\mu \lambda = \delta$ 使得下图包含三个好三角与一处三角射:
		\begin{equation}
			% https://q.uiver.app/#q=WzAsMTIsWzAsMSwiWCIsWzM1OSwxMDAsNjAsMV1dLFsxLDEsIllfMiIsWzM1OSwxMDAsNjAsMV1dLFswLDIsIllfMSIsWzM1OSwxMDAsNjAsMV1dLFsxLDIsIloiLFszNTksMTAwLDYwLDFdXSxbMiwxLCJXIixbMzU5LDEwMCw2MCwxXV0sWzAsMCwiWCJdLFsxLDAsIllfMiBcXG9wbHVzIFlfMSJdLFsyLDAsIloiXSxbMywwLCJcXFNpZ21hIFgiXSxbMywxLCJcXFNpZ21hIFgiLFszNTksMTAwLDYwLDFdXSxbMiwyLCJXIixbMjM2LDEwMCw2MCwxXV0sWzMsMiwiXFxTaWdtYSBZIl0sWzAsMiwiYl8xIiwwLHsiY29sb3VyIjpbMzU5LDEwMCw2MF19LFszNTksMTAwLDYwLDFdXSxbMCwxLCJhXzEiLDAseyJjb2xvdXIiOlszNTksMTAwLDYwXX0sWzM1OSwxMDAsNjAsMV1dLFsxLDMsImJfMiIsMCx7ImNvbG91ciI6WzM1OSwxMDAsNjBdfSxbMzU5LDEwMCw2MCwxXV0sWzEsNCwiIiwwLHsiY29sb3VyIjpbMzU5LDEwMCw2MF19XSxbNSw2LCJcXGJpbm9te2FfMX17Yl8xfSJdLFs2LDcsIihiXzEsLWFfMikiXSxbNyw4LCJcXG11XFxsYW1iZGEiLDAseyJjb2xvdXIiOlsyMzYsMTAwLDYwXX0sWzIzNiwxMDAsNjAsMV1dLFs0LDEwLCIiLDAseyJsZXZlbCI6Miwic3R5bGUiOnsiaGVhZCI6eyJuYW1lIjoibm9uZSJ9fX1dLFs0LDksIlxcbXUiLDAseyJjb2xvdXIiOlszNTksMTAwLDYwXX0sWzM1OSwxMDAsNjAsMV1dLFs5LDExLCJcXFNpZ21hIGJfMSJdLFsyLDMsImFfMiIsMCx7ImNvbG91ciI6WzM1OSwxMDAsNjBdfSxbMzU5LDEwMCw2MCwxXV0sWzMsMTAsIlxcbGFtYmRhIiwyLHsiY29sb3VyIjpbMjM2LDEwMCw2MF19LFsyMzYsMTAwLDYwLDFdXSxbMTAsMTFdXQ==
\begin{tikzcd}
	X & {Y_2 \oplus Y_1} & Z & {\Sigma X} \\
	\textcolor{rgb,255:red,255;green,51;blue,54}{X} & \textcolor{rgb,255:red,255;green,51;blue,54}{{Y_2}} & \textcolor{rgb,255:red,255;green,51;blue,54}{W} & \textcolor{rgb,255:red,255;green,51;blue,54}{{\Sigma X}} \\
	\textcolor{rgb,255:red,255;green,51;blue,54}{{Y_1}} & \textcolor{rgb,255:red,255;green,51;blue,54}{Z} & \textcolor{rgb,255:red,51;green,65;blue,255}{W} & {\Sigma Y_1}
	\arrow["{\binom{a_1}{b_1}}", from=1-1, to=1-2]
	\arrow["{(b_2 \ -a_2)}", from=1-2, to=1-3]
	\arrow["{\delta}", color={rgb,255:red,51;green,65;blue,255}, from=1-3, to=1-4]
	\arrow["{a_1}", color={rgb,255:red,255;green,51;blue,54}, from=2-1, to=2-2]
	\arrow["{b_1}", color={rgb,255:red,255;green,51;blue,54}, from=2-1, to=3-1]
	\arrow[color={rgb,255:red,255;green,51;blue,54}, from=2-2, to=2-3]
	\arrow["{b_2}", color={rgb,255:red,255;green,51;blue,54}, from=2-2, to=3-2]
	\arrow["\mu", color={rgb,255:red,255;green,51;blue,54}, from=2-3, to=2-4]
	\arrow[equals, from=2-3, to=3-3]
	\arrow["{\Sigma b_1}", from=2-4, to=3-4]
	\arrow["{a_2}", color={rgb,255:red,255;green,51;blue,54}, from=3-1, to=3-2]
	\arrow["\lambda"', color={rgb,255:red,51;green,65;blue,255}, from=3-2, to=3-3]
	\arrow[from=3-3, to=3-4]
\end{tikzcd}
		\end{equation}
	\end{enumerate}
	\begin{proof}
		($1 \to 2$). 若给定 $a_1$ 与 $b_2$, 则八面体公理(下图上)诱导了三角射(下图下):
		\begin{equation}
			% https://q.uiver.app/#q=WzAsMjEsWzAsNCwiWCIsWzM1OSwxMDAsNjAsMV1dLFsxLDQsIllfMiIsWzM1OSwxMDAsNjAsMV1dLFszLDQsIlxcU2lnbWEgWCJdLFsyLDQsIlciLFsxMTIsMTAwLDMzLDFdXSxbMCw1LCJZXzEiLFszNTksMTAwLDYwLDFdXSxbMSw1LCJaIixbMTEyLDEwMCwzMywxXV0sWzIsNSwiVyIsWzExMiwxMDAsMzMsMV1dLFszLDUsIlxcU2lnbWEgWV8xIl0sWzAsMSwiWCJdLFsxLDEsIllfMiJdLFsxLDAsIllfMiBcXG9wbHVzIFlfMSIsWzM1OSwxMDAsNjAsMV1dLFswLDAsIlgiLFszNTksMTAwLDYwLDFdXSxbMiwwLCJaIixbMTEyLDEwMCwzMywxXV0sWzIsMSwiVyIsWzExMiwxMDAsMzMsMV1dLFszLDEsIlxcU2lnbWEgWCIsWzExMiwxMDAsMzMsMV1dLFszLDAsIlxcU2lnbWEgWCJdLFsxLDIsIlxcU2lnbWEgWV8xIl0sWzIsMywiXFxTaWdtYSBaIl0sWzIsMiwiXFxTaWdtYSBZXzEiXSxbMSwzLCJcXFNpZ21hIFlfMiBcXG9wbHVzIFxcU2lnbWEgWV8xIl0sWzMsMiwiXFxTaWdtYSBZXzIgXFxvcGx1cyBcXFNpZ21hIFlfMSJdLFswLDEsImFfMSIsMCx7ImNvbG91ciI6WzM1OSwxMDAsNjBdfSxbMzU5LDEwMCw2MCwxXV0sWzAsNCwiYl8xIiwwLHsiY29sb3VyIjpbMzU5LDEwMCw2MF19LFszNTksMTAwLDYwLDFdXSxbMSw1LCJiXzIiXSxbNCw1LCJhXzIiXSxbMSwzLCJcXGxhbWJkYSBiXzIiXSxbNSw2LCJcXGxhbWJkYSIsMCx7ImNvbG91ciI6WzExMiwxMDAsMzNdLCJzdHlsZSI6eyJib2R5Ijp7Im5hbWUiOiJkYXNoZWQifX19LFsxMTIsMTAwLDMzLDFdXSxbNiw3LCJ0Il0sWzMsNiwiIiwwLHsibGV2ZWwiOjIsImNvbG91ciI6WzExMiwxMDAsMzNdLCJzdHlsZSI6eyJoZWFkIjp7Im5hbWUiOiJub25lIn19fV0sWzIsNywiXFxTaWdtYSBiXzEiXSxbMTAsOSwiXFxiaW5vbSAxMCJdLFs4LDksImFfMSJdLFsxMSw4LCIiLDAseyJsZXZlbCI6Miwic3R5bGUiOnsiaGVhZCI6eyJuYW1lIjoibm9uZSJ9fX1dLFsxMSwxMCwiXFxiaW5vbSB7YV8xfXtiXzF9IiwwLHsiY29sb3VyIjpbMzU5LDEwMCw2MF19LFszNTksMTAwLDYwLDFdXSxbMTAsMTIsIihiXzIsLWFfMSkiXSxbOSwxMywiXFxsYW1iZGEgYl8yIl0sWzEzLDE0LCJcXG11IiwwLHsiY29sb3VyIjpbMTEyLDEwMCwzM119LFsxMTIsMTAwLDMzLDFdXSxbMTQsMTUsIiIsMCx7ImxldmVsIjoyLCJzdHlsZSI6eyJoZWFkIjp7Im5hbWUiOiJub25lIn19fV0sWzEyLDE1LCJcXG11IFxcbGFtYmRhIl0sWzEyLDEzLCJcXGxhbWJkYSIsMCx7ImNvbG91ciI6WzExMiwxMDAsMzNdLCJzdHlsZSI6eyJib2R5Ijp7Im5hbWUiOiJkYXNoZWQifX19LFsxMTIsMTAwLDMzLDFdXSxbOSwxNiwiMCJdLFsxNiwxOCwiIiwwLHsibGV2ZWwiOjIsInN0eWxlIjp7ImhlYWQiOnsibmFtZSI6Im5vbmUifX19XSxbMTMsMTgsInQiLDAseyJzdHlsZSI6eyJib2R5Ijp7Im5hbWUiOiJkYXNoZWQifX19XSxbMTYsMTldLFsxOSwxN10sWzE4LDE3LCIiLDAseyJzdHlsZSI6eyJib2R5Ijp7Im5hbWUiOiJkYXNoZWQifX19XSxbMTgsMjAsIlxcYmlub20gMDEiXSxbMTQsMjAsIlxcYmlub217XFxTaWdtYSBhXzF9e1xcU2lnbWEgYl8xfSJdLFszLDIsIlxcbXUiLDAseyJjb2xvdXIiOlsxMTIsMTAwLDMzXX0sWzExMiwxMDAsMzMsMV1dXQ==
\begin{tikzcd}
	\textcolor{rgb,255:red,255;green,51;blue,54}{X} & \textcolor{rgb,255:red,255;green,51;blue,54}{{Y_2 \oplus Y_1}} & \textcolor{rgb,255:red,22;green,168;blue,0}{Z} & {\Sigma X} \\
	X & {Y_2} & \textcolor{rgb,255:red,22;green,168;blue,0}{W} & \textcolor{rgb,255:red,22;green,168;blue,0}{{\Sigma X}} \\
	& {\Sigma Y_1} & {\Sigma Y_1} & {\Sigma Y_2 \oplus \Sigma Y_1} \\
	& {\Sigma Y_2 \oplus \Sigma Y_1} & {\Sigma Z} \\
	\textcolor{rgb,255:red,255;green,51;blue,54}{X} & \textcolor{rgb,255:red,255;green,51;blue,54}{{Y_2}} & \textcolor{rgb,255:red,22;green,168;blue,0}{W} & {\Sigma X} \\
	\textcolor{rgb,255:red,255;green,51;blue,54}{{Y_1}} & \textcolor{rgb,255:red,22;green,168;blue,0}{Z} & \textcolor{rgb,255:red,22;green,168;blue,0}{W} & {\Sigma Y_1}
	\arrow["{\binom {a_1}{b_1}}", color={rgb,255:red,255;green,51;blue,54}, from=1-1, to=1-2]
	\arrow[equals, from=1-1, to=2-1]
	\arrow["{(b_2,-a_2)}", from=1-2, to=1-3]
	\arrow["{\binom 10}", from=1-2, to=2-2]
	\arrow["{\mu \lambda}", from=1-3, to=1-4]
	\arrow["\lambda", color={rgb,255:red,22;green,168;blue,0}, dashed, from=1-3, to=2-3]
	\arrow["{a_1}", from=2-1, to=2-2]
	\arrow["{\lambda b_2}", from=2-2, to=2-3]
	\arrow["0", from=2-2, to=3-2]
	\arrow["\mu", color={rgb,255:red,22;green,168;blue,0}, from=2-3, to=2-4]
	\arrow["t", dashed, from=2-3, to=3-3]
	\arrow[equals, from=2-4, to=1-4]
	\arrow["{\binom{\Sigma a_1}{\Sigma b_1}}", from=2-4, to=3-4]
	\arrow[equals, from=3-2, to=3-3]
	\arrow[from=3-2, to=4-2]
	\arrow["{\binom 01}", from=3-3, to=3-4]
	\arrow[dashed, from=3-3, to=4-3]
	\arrow[from=4-2, to=4-3]
	\arrow["{a_1}", color={rgb,255:red,255;green,51;blue,54}, from=5-1, to=5-2]
	\arrow["{b_1}", color={rgb,255:red,255;green,51;blue,54}, from=5-1, to=6-1]
	\arrow["{\lambda b_2}", from=5-2, to=5-3]
	\arrow["{b_2}", from=5-2, to=6-2]
	\arrow["\mu", color={rgb,255:red,22;green,168;blue,0}, from=5-3, to=5-4]
	\arrow[draw={rgb,255:red,22;green,168;blue,0}, equals, from=5-3, to=6-3]
	\arrow["{\Sigma b_1}", from=5-4, to=6-4]
	\arrow["{a_2}", from=6-1, to=6-2]
	\arrow["\lambda", color={rgb,255:red,22;green,168;blue,0}, dashed, from=6-2, to=6-3]
	\arrow["t", from=6-3, to=6-4]
\end{tikzcd}.
		\end{equation}
		此处省略校验过程.
		\\
		($2 \to 1$). 沿用 $(1 \to 2)$ 中构造即可.
		\\
		($3 \to 1$). 依照题设补全四个好三角
		\begin{equation}
			% https://q.uiver.app/#q=WzAsMTMsWzIsMCwiWCIsWzIyOCwxMDAsNjAsMV1dLFszLDAsIllfMiIsWzM1OSwxMDAsNjAsMV1dLFsyLDEsIllfMSIsWzIyOCwxMDAsNjAsMV1dLFszLDEsIloiLFsxMTQsMTAwLDM2LDFdXSxbNCwwLCJXIl0sWzQsMSwiVyIsWzExNCwxMDAsMzYsMV1dLFswLDAsIlxcU2lnbWFeey0xfVciLFsyMjgsMTAwLDYwLDFdXSxbMCwxLCJcXFNpZ21hXnstMX1XIl0sWzIsMiwiTCJdLFsyLDMsIlxcU2lnbWEgWCJdLFszLDIsIkwiLFsxMTQsMTAwLDM2LDFdXSxbMywzLCJcXFNpZ21hIFlfMiJdLFs0LDIsIlxcU2lnbWEgWCIsWzExNCwxMDAsMzYsMV1dLFswLDIsImJfMSIsMCx7ImNvbG91ciI6WzIyOCwxMDAsNjBdfSxbMjI4LDEwMCw2MCwxXV0sWzAsMSwiYV8xIiwwLHsiY29sb3VyIjpbMzU5LDEwMCw2MF19LFszNTksMTAwLDYwLDFdXSxbMSwzLCJiXzIiLDAseyJjb2xvdXIiOlszNTksMTAwLDYwXX0sWzM1OSwxMDAsNjAsMV1dLFsxLDQsIlxcbGFtYmRhIGJfMiJdLFs0LDUsIiIsMCx7ImxldmVsIjoyLCJzdHlsZSI6eyJoZWFkIjp7Im5hbWUiOiJub25lIn19fV0sWzIsMywiYV8yIiwwLHsiY29sb3VyIjpbMzU5LDEwMCw2MF19LFszNTksMTAwLDYwLDFdXSxbMyw1LCJcXGxhbWJkYSIsMCx7ImNvbG91ciI6WzExNCwxMDAsMzZdfSxbMTE0LDEwMCwzNiwxXV0sWzYsMCwiLVxcU2lnbWEgXnstMX1cXG11ICIsMCx7ImNvbG91ciI6WzIyOCwxMDAsNjBdfSxbMjI4LDEwMCw2MCwxXV0sWzYsNywiIiwyLHsibGV2ZWwiOjIsInN0eWxlIjp7ImhlYWQiOnsibmFtZSI6Im5vbmUifX19XSxbNywyLCItYl8xKFxcU2lnbWEgXnstMX1cXG11KSJdLFsyLDgsImwgYV8yIl0sWzgsOSwiayJdLFs4LDEwLCIiLDAseyJsZXZlbCI6Miwic3R5bGUiOnsiaGVhZCI6eyJuYW1lIjoibm9uZSJ9fX1dLFszLDEwLCJsIiwwLHsiY29sb3VyIjpbMTE0LDEwMCwzNl19LFsxMTQsMTAwLDM2LDFdXSxbOSwxMSwiXFxTaWdtYSBhXzEiXSxbMTAsMTEsIihcXFNpZ21hIGFfMSkgayJdLFswLDMsIlxcYmxhY2tzcXVhcmUiLDEseyJjb2xvdXIiOlszNTksMTAwLDYwXSwic3R5bGUiOnsiYm9keSI6eyJuYW1lIjoibm9uZSJ9LCJoZWFkIjp7Im5hbWUiOiJub25lIn19fSxbMzU5LDEwMCw2MCwxXV0sWzEwLDEyLCJrIiwyLHsiY29sb3VyIjpbMTE0LDEwMCwzNl19LFsxMTQsMTAwLDM2LDFdXSxbNSwxMiwiLVxcbXUiLDAseyJjb2xvdXIiOlsxMTQsMTAwLDM2XX0sWzExNCwxMDAsMzYsMV1dLFszLDEyLCI/IiwxLHsiY29sb3VyIjpbMTE0LDEwMCwzNl0sInN0eWxlIjp7ImJvZHkiOnsibmFtZSI6Im5vbmUifSwiaGVhZCI6eyJuYW1lIjoibm9uZSJ9fX0sWzExNCwxMDAsMzYsMV1dXQ==
\begin{tikzcd}
	\textcolor{rgb,255:red,51;green,92;blue,255}{{\Sigma^{-1}W}} && \textcolor{rgb,255:red,51;green,92;blue,255}{X} & \textcolor{rgb,255:red,255;green,51;blue,54}{{Y_2}} & W \\
	{\Sigma^{-1}W} && \textcolor{rgb,255:red,51;green,92;blue,255}{{Y_1}} & \textcolor{rgb,255:red,18;green,184;blue,0}{Z} & \textcolor{rgb,255:red,18;green,184;blue,0}{W} \\
	&& L & \textcolor{rgb,255:red,18;green,184;blue,0}{L} & \textcolor{rgb,255:red,18;green,184;blue,0}{{\Sigma X}} \\
	&& {\Sigma X} & {\Sigma Y_2}
	\arrow["{-\Sigma ^{-1}\mu }", color={rgb,255:red,51;green,92;blue,255}, from=1-1, to=1-3]
	\arrow[equals, from=1-1, to=2-1]
	\arrow["{a_1}", color={rgb,255:red,255;green,51;blue,54}, from=1-3, to=1-4]
	\arrow["{b_1}", color={rgb,255:red,51;green,92;blue,255}, from=1-3, to=2-3]
	\arrow["\blacksquare"{description}, color={rgb,255:red,255;green,51;blue,54}, draw=none, from=1-3, to=2-4]
	\arrow["{\lambda b_2}", from=1-4, to=1-5]
	\arrow["{b_2}", color={rgb,255:red,255;green,51;blue,54}, from=1-4, to=2-4]
	\arrow[equals, from=1-5, to=2-5]
	\arrow["{-b_1(\Sigma ^{-1}\mu)}", from=2-1, to=2-3]
	\arrow["{a_2}", color={rgb,255:red,255;green,51;blue,54}, from=2-3, to=2-4]
	\arrow["{l a_2}", from=2-3, to=3-3]
	\arrow["\lambda", color={rgb,255:red,18;green,184;blue,0}, from=2-4, to=2-5]
	\arrow["l", color={rgb,255:red,18;green,184;blue,0}, from=2-4, to=3-4]
	\arrow["{?}"{description}, color={rgb,255:red,18;green,184;blue,0}, draw=none, from=2-4, to=3-5]
	\arrow["{-\mu}", color={rgb,255:red,18;green,184;blue,0}, from=2-5, to=3-5]
	\arrow[equals, from=3-3, to=3-4]
	\arrow["k", from=3-3, to=4-3]
	\arrow["k"', color={rgb,255:red,18;green,184;blue,0}, from=3-4, to=3-5]
	\arrow["{(\Sigma a_1) k}", from=3-4, to=4-4]
	\arrow["{\Sigma a_1}", from=4-3, to=4-4]
\end{tikzcd}.
		\end{equation}
		由同伦推出拉回方块 $\color{rgb,255:red,255;green,51;blue,54}\blacksquare$, 上图除绿色方块 $\color{rgb,255:red,18;green,184;blue,0}?$ 处均交换. 依照 $3$ 的题设,
\begin{equation}
X\xrightarrow{\binom{a_1}{b_1}} Y_2\oplus Y_1\xrightarrow{(- b_2 \ a_2)} Z\overset {kl} \to \Sigma X
\end{equation}
与
\begin{equation}
X\xrightarrow{\binom{a_1}{b_1}} Y_2\oplus Y_1\xrightarrow{(b_2 \ - a_2)} Z\overset {\mu \lambda} \to \Sigma X
\end{equation}
均是好三角. 同时 $\mu \lambda = \delta = - kl$.
\\
($1 \to 3$). 沿用 $(1 \to 2)$ 中构造即可.
	\end{proof}
\end{theorem}

\begin{remark}
	特别地, 八面体公理中的方块 $\star$ 是同伦推出拉回.
	\begin{equation}
		% https://q.uiver.app/#q=WzAsMTcsWzEsMSwiWCJdLFsyLDEsIlkiXSxbMywxLCJXIl0sWzQsMSwiXFxTaWdtYSBYIl0sWzEsMiwiWCJdLFsyLDIsIloiXSxbMywyLCJSIl0sWzQsMiwiXFxTaWdtYSBYIl0sWzIsMywiUyJdLFszLDMsIlMiXSxbMiw0LCJcXFNpZ21hIFkiXSxbMyw0LCJcXFNpZ21hIFciXSxbMCwxLCJyXzEiXSxbMCwyLCJyXzIiXSxbMiwwLCJjXzIiXSxbMywwLCJjXzMiXSxbNCwzLCJcXFNpZ21hIFkiXSxbMCwxLCJ1Il0sWzEsMiwidiJdLFsyLDMsInciXSxbNCw1LCJ4Il0sWzUsNiwieSJdLFs2LDcsInoiXSxbMCw0LCIiLDAseyJsZXZlbCI6Miwic3R5bGUiOnsiaGVhZCI6eyJuYW1lIjoibm9uZSJ9fX1dLFszLDcsIiIsMCx7ImxldmVsIjoyLCJzdHlsZSI6eyJoZWFkIjp7Im5hbWUiOiJub25lIn19fV0sWzEsNSwiYSJdLFs1LDgsImIiXSxbMiw2LCIiLDAseyJzdHlsZSI6eyJib2R5Ijp7Im5hbWUiOiJkYXNoZWQifX19XSxbNiw5LCIiLDAseyJzdHlsZSI6eyJib2R5Ijp7Im5hbWUiOiJkYXNoZWQifX19XSxbOSwxMSwiIiwwLHsic3R5bGUiOnsiYm9keSI6eyJuYW1lIjoiZGFzaGVkIn19fV0sWzgsMTAsImMiXSxbMTAsMTEsIlxcU2lnbWEgdiIsMCx7ImNvbG91ciI6WzM1MiwxMDAsNjBdfSxbMzUyLDEwMCw2MCwxXV0sWzgsOSwiIiwwLHsibGV2ZWwiOjIsInN0eWxlIjp7ImhlYWQiOnsibmFtZSI6Im5vbmUifX19XSxbOSwxNiwiYyJdLFs3LDE2LCJcXFNpZ21hIHUiLDAseyJjb2xvdXIiOlszNTIsMTAwLDYwXX0sWzM1MiwxMDAsNjAsMV1dLFsxLDYsIlxcc3RhciIsMSx7InN0eWxlIjp7ImJvZHkiOnsibmFtZSI6Im5vbmUifSwiaGVhZCI6eyJuYW1lIjoibm9uZSJ9fX1dXQ==
\begin{tikzcd}
	&& {c_2} & {c_3} \\[-12pt]
	{r_1} & X & Y & W & {\Sigma X} \\
	{r_2} & X & Z & R & {\Sigma X} \\
	&& S & S & {\Sigma Y} \\
	&& {\Sigma Y} & {\Sigma W}
	\arrow["u", from=2-2, to=2-3]
	\arrow[equals, from=2-2, to=3-2]
	\arrow["v", from=2-3, to=2-4]
	\arrow["a", from=2-3, to=3-3]
	\arrow["\star"{description}, draw=none, from=2-3, to=3-4]
	\arrow["w", from=2-4, to=2-5]
	\arrow[dashed, from=2-4, to=3-4]
	\arrow[equals, from=2-5, to=3-5]
	\arrow["x", from=3-2, to=3-3]
	\arrow["y", from=3-3, to=3-4]
	\arrow["b", from=3-3, to=4-3]
	\arrow["z", from=3-4, to=3-5]
	\arrow[dashed, from=3-4, to=4-4]
	\arrow["{\Sigma u}", color={rgb,255:red,255;green,51;blue,78}, from=3-5, to=4-5]
	\arrow[equals, from=4-3, to=4-4]
	\arrow["c", from=4-3, to=5-3]
	\arrow["c", from=4-4, to=4-5]
	\arrow[dashed, from=4-4, to=5-4]
	\arrow["{\Sigma v}", color={rgb,255:red,255;green,51;blue,78}, from=5-3, to=5-4]
\end{tikzcd}.
	\end{equation}
	特别地, 考虑以下三角射
	\begin{equation}
		% https://q.uiver.app/#q=WzAsOCxbMCwwLCJcXGJ1bGxldCJdLFsxLDAsIlxcYnVsbGV0Il0sWzIsMCwiXFxidWxsZXQiXSxbMywwLCJcXGJ1bGxldCJdLFszLDEsIlxcYnVsbGV0Il0sWzAsMSwiXFxidWxsZXQiXSxbMSwxLCJcXGJ1bGxldCJdLFsyLDEsIlxcYnVsbGV0Il0sWzAsMV0sWzEsMl0sWzIsM10sWzMsNCwiIiwwLHsibGV2ZWwiOjIsInN0eWxlIjp7ImhlYWQiOnsibmFtZSI6Im5vbmUifX19XSxbMCw1LCIiLDAseyJsZXZlbCI6Miwic3R5bGUiOnsiaGVhZCI6eyJuYW1lIjoibm9uZSJ9fX1dLFs1LDZdLFs2LDddLFs3LDRdLFsxLDYsIlxcYWxwaGEiXSxbMiw3LCJcXGJldGEiLDAseyJzdHlsZSI6eyJib2R5Ijp7Im5hbWUiOiJkYXNoZWQifX19XV0=
\begin{tikzcd}
	\cdot & \cdot & \cdot & \cdot \\
	\cdot & \cdot & \cdot & \cdot
	\arrow[from=1-1, to=1-2]
	\arrow[equals, from=1-1, to=2-1]
	\arrow[from=1-2, to=1-3]
	\arrow["\alpha", from=1-2, to=2-2]
	\arrow[from=1-3, to=1-4]
	\arrow["\beta", dashed, from=1-3, to=2-3]
	\arrow[equals, from=1-4, to=2-4]
	\arrow[from=2-1, to=2-2]
	\arrow[from=2-2, to=2-3]
	\arrow[from=2-3, to=2-4]
\end{tikzcd},
	\end{equation}
	则存在 $\beta$ 使得中间方块为同伦的推出拉回. 由八面体公理即证.
\end{remark}

\begin{example}\label{eg: 好三角的态射与 ses 的推出拉回}
	仍将好三角的第三项态射``视作'' ses 代表的扩张元, 则有三角射
	\begin{equation}
		% https://q.uiver.app/#q=WzAsMTYsWzAsMCwiXFxidWxsZXQiXSxbMSwwLCJcXGJ1bGxldCJdLFsyLDAsIlxcYnVsbGV0Il0sWzIsMSwiXFxidWxsZXQiXSxbMCwxLCJcXGJ1bGxldCJdLFsxLDEsIlxcYnVsbGV0Il0sWzMsMCwiXFxidWxsZXQiXSxbMywxLCJcXGJ1bGxldCJdLFswLDIsIlxcYnVsbGV0Il0sWzAsMywiXFxidWxsZXQiXSxbMSwyLCJcXGJ1bGxldCJdLFsyLDIsIlxcYnVsbGV0Il0sWzMsMiwiXFxidWxsZXQiXSxbMSwzLCJcXGJ1bGxldCJdLFsyLDMsIlxcYnVsbGV0Il0sWzMsMywiXFxidWxsZXQiXSxbMCwxXSxbMSwyXSxbMiwzLCIiLDAseyJsZXZlbCI6Miwic3R5bGUiOnsiaGVhZCI6eyJuYW1lIjoibm9uZSJ9fX1dLFs0LDVdLFs1LDNdLFswLDQsImYiXSxbMSw1LCIiLDAseyJzdHlsZSI6eyJib2R5Ijp7Im5hbWUiOiJkYXNoZWQifX19XSxbMiw2LCJcXGRlbHRhIl0sWzMsNywiZl9cXGFzdCBcXGRlbHRhIl0sWzYsNywiXFxTaWdtYSBmIl0sWzgsOSwiIiwwLHsibGV2ZWwiOjIsInN0eWxlIjp7ImhlYWQiOnsibmFtZSI6Im5vbmUifX19XSxbOCwxMF0sWzEwLDExXSxbMTEsMTIsImdeXFxhc3QgXFx2YXJlcHNpbG9uICJdLFsxMCwxMywiIiwwLHsic3R5bGUiOnsiYm9keSI6eyJuYW1lIjoiZGFzaGVkIn19fV0sWzksMTNdLFsxMywxNF0sWzE0LDE1LCJcXHZhcmVwc2lsb24gIl0sWzEyLDE1LCIiLDAseyJsZXZlbCI6Miwic3R5bGUiOnsiaGVhZCI6eyJuYW1lIjoibm9uZSJ9fX1dLFsxMSwxNCwiZyJdXQ==
\begin{tikzcd}
	\cdot & \cdot & \cdot & \cdot \\
	\cdot & \cdot & \cdot & \cdot \\
	\cdot & \cdot & \cdot & \cdot \\
	\cdot & \cdot & \cdot & \cdot
	\arrow[from=1-1, to=1-2]
	\arrow["f", from=1-1, to=2-1]
	\arrow[from=1-2, to=1-3]
	\arrow[dashed, from=1-2, to=2-2]
	\arrow["\delta", from=1-3, to=1-4]
	\arrow[equals, from=1-3, to=2-3]
	\arrow["{\Sigma f}", from=1-4, to=2-4]
	\arrow[from=2-1, to=2-2]
	\arrow[from=2-2, to=2-3]
	\arrow["{f_\ast \delta}", from=2-3, to=2-4]
	\arrow[from=3-1, to=3-2]
	\arrow[equals, from=3-1, to=4-1]
	\arrow[from=3-2, to=3-3]
	\arrow[dashed, from=3-2, to=4-2]
	\arrow["{g^\ast \varepsilon }", from=3-3, to=3-4]
	\arrow["g", from=3-3, to=4-3]
	\arrow[equals, from=3-4, to=4-4]
	\arrow[from=4-1, to=4-2]
	\arrow[from=4-2, to=4-3]
	\arrow["{\varepsilon }", from=4-3, to=4-4]
\end{tikzcd}.
	\end{equation}
	以上, ``ses 的推出拉回''由态射复合简练地描述: 
	\begin{equation}
	(\Sigma f) \circ \delta = f_\ast \delta,\quad 	\varepsilon \circ g = g^\ast \varepsilon.
	\end{equation}
\end{example}
