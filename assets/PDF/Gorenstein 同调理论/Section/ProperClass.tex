\section{阅读笔记: {\small Proper classes and Gorensteinness in extriangulated categories}}

文章见 \cite{huProperClassesGorensteinness2020a}.

\subsection{带真类的外三角范畴}

本章节介绍真类 $\xi$ 的定义, 并使用图表定理简单证明了 \cref{thm:proper_class_extriangulated}.

\begin{notation}
记 $(\mathcal{C}, \mathbb E, \mathfrak s)$ 为外三角范畴.
\end{notation}

\begin{definition}[真类]\label{def:xi}
    称 $\xi$ 是 $\mathbb E$-三角组成的真类, 若满足以下条件:
    \begin{enumerate}
        \item[P1] $\xi$ 对双积与同构封闭, 且包含所有可裂的 $\mathbb E$-三角.
        \item[P2] $\xi$ 对基变换和余积变换封闭. 对下图所示的 $\mathbb E$-三角射, 若中间行属于 $\xi$, 则上下行也属于 $\xi$:
		\begin{equation}
			% https://q.uiver.app/#q=WzAsMTIsWzAsMSwiWCJdLFsxLDEsIlkiXSxbMiwxLCJaIl0sWzMsMSwiXFwsIl0sWzAsMCwiWCJdLFsxLDAsIlknIl0sWzIsMCwiWiciXSxbMCwyLCJYJyciXSxbMSwyLCJZJyciXSxbMiwyLCJaIl0sWzMsMCwiXFwsIl0sWzMsMiwiXFwsIl0sWzAsMSwiZiIsMCx7ImNvbG91ciI6WzMwLDYwLDYwXX0sWzMwLDYwLDYwLDFdXSxbMiwzLCJcXGRlbHRhIiwwLHsiY29sb3VyIjpbMzAsNjAsNjBdLCJzdHlsZSI6eyJib2R5Ijp7Im5hbWUiOiJkYXNoZWQifX19LFszMCw2MCw2MCwxXV0sWzcsOCwiZicnIiwwLHsiY29sb3VyIjpbMjQwLDYwLDYwXX0sWzI0MCw2MCw2MCwxXV0sWzgsOSwiZycnIiwwLHsiY29sb3VyIjpbMjQwLDYwLDYwXX0sWzI0MCw2MCw2MCwxXV0sWzksMTEsIlxcYWxwaGFfXFxhc3QgXFxkZWx0YSIsMCx7ImNvbG91ciI6WzI0MCw2MCw2MF0sInN0eWxlIjp7ImJvZHkiOnsibmFtZSI6ImRhc2hlZCJ9fX0sWzI0MCw2MCw2MCwxXV0sWzQsNSwiZiciLDAseyJjb2xvdXIiOlsyNDAsNjAsNjBdfSxbMjQwLDYwLDYwLDFdXSxbNSw2LCJnJyIsMCx7ImNvbG91ciI6WzI0MCw2MCw2MF19LFsyNDAsNjAsNjAsMV1dLFs2LDEwLCJcXGdhbW1hXlxcYXN0IFxcZGVsdGEiLDAseyJjb2xvdXIiOlsyNDAsNjAsNjBdLCJzdHlsZSI6eyJib2R5Ijp7Im5hbWUiOiJkYXNoZWQifX19LFsyNDAsNjAsNjAsMV1dLFsxLDIsImciLDAseyJjb2xvdXIiOlszMCw2MCw2MF19LFszMCw2MCw2MCwxXV0sWzUsMSwiXFxnYW1tYSciXSxbMCw3LCJcXGFscGhhIl0sWzEsOCwiXFxhbHBoYSciXSxbNiwyLCJcXGdhbW1hIl0sWzQsMCwiIiwwLHsibGV2ZWwiOjIsInN0eWxlIjp7ImhlYWQiOnsibmFtZSI6Im5vbmUifX19XSxbMiw5LCIiLDAseyJsZXZlbCI6Miwic3R5bGUiOnsiaGVhZCI6eyJuYW1lIjoibm9uZSJ9fX1dXQ==
\begin{tikzcd}[ampersand replacement=\&]
	X \& {Y'} \& {Z'} \& {\,} \\
	X \& Y \& Z \& {\,} \\
	{X''} \& {Y''} \& Z \& {\,}
	\arrow["{f'}", color={rgb,255:red,92;green,92;blue,214}, from=1-1, to=1-2]
	\arrow[equals, from=1-1, to=2-1]
	\arrow["{g'}", color={rgb,255:red,92;green,92;blue,214}, from=1-2, to=1-3]
	\arrow["{\gamma'}", from=1-2, to=2-2]
	\arrow["{\gamma^\ast \delta}", color={rgb,255:red,92;green,92;blue,214}, dashed, from=1-3, to=1-4]
	\arrow["\gamma", from=1-3, to=2-3]
	\arrow["f", color={rgb,255:red,214;green,153;blue,92}, from=2-1, to=2-2]
	\arrow["\alpha", from=2-1, to=3-1]
	\arrow["g", color={rgb,255:red,214;green,153;blue,92}, from=2-2, to=2-3]
	\arrow["{\alpha'}", from=2-2, to=3-2]
	\arrow["\delta", color={rgb,255:red,214;green,153;blue,92}, dashed, from=2-3, to=2-4]
	\arrow[equals, from=2-3, to=3-3]
	\arrow["{f''}", color={rgb,255:red,92;green,92;blue,214}, from=3-1, to=3-2]
	\arrow["{g''}", color={rgb,255:red,92;green,92;blue,214}, from=3-2, to=3-3]
	\arrow["{\alpha_\ast \delta}", color={rgb,255:red,92;green,92;blue,214}, dashed, from=3-3, to=3-4]
\end{tikzcd}.
		\end{equation}
        \item[P3] $\xi$ 饱和: 给定 $\delta_1$ 与 $\delta_2$ 实现的``拉回''
        \begin{equation}\label{eq:P3}
% https://q.uiver.app/#q=WzAsMTIsWzAsMSwiXFxjZG90Il0sWzEsMSwiXFxjZG90Il0sWzEsMCwiXFxjZG90Il0sWzIsMCwiXFxjZG90Il0sWzIsMSwiXFxjZG90Il0sWzEsMiwiXFxjZG90Il0sWzAsMiwiXFxjZG90Il0sWzIsMiwiXFxjZG90Il0sWzMsMSwiXFwsICJdLFszLDIsIlxcLCAiXSxbMiwzLCJcXCwgIl0sWzEsMywiXFwsICJdLFswLDYsIiIsMCx7ImxldmVsIjoyLCJzdHlsZSI6eyJoZWFkIjp7Im5hbWUiOiJub25lIn19fV0sWzIsMywiIiwwLHsibGV2ZWwiOjIsInN0eWxlIjp7ImhlYWQiOnsibmFtZSI6Im5vbmUifX19XSxbMCwxLCIiLDAseyJjb2xvdXIiOlszMCw2MCw2MF0sInN0eWxlIjp7ImJvZHkiOnsibmFtZSI6ImRhc2hlZCJ9fX1dLFsxLDQsIiIsMCx7ImNvbG91ciI6WzMwLDYwLDYwXSwic3R5bGUiOnsiYm9keSI6eyJuYW1lIjoiZGFzaGVkIn19fV0sWzQsOCwiXFxkZWx0YV8xJyIsMCx7ImNvbG91ciI6WzMwLDYwLDYwXSwic3R5bGUiOnsiYm9keSI6eyJuYW1lIjoiZGFzaGVkIn19fSxbMzAsNjAsNjAsMV1dLFsyLDEsIiIsMCx7InN0eWxlIjp7ImJvZHkiOnsibmFtZSI6ImRhc2hlZCJ9fX1dLFsxLDUsIiIsMCx7InN0eWxlIjp7ImJvZHkiOnsibmFtZSI6ImRhc2hlZCJ9fX1dLFs1LDExLCJcXGRlbHRhXzInIiwwLHsic3R5bGUiOnsiYm9keSI6eyJuYW1lIjoiZGFzaGVkIn19fV0sWzMsNCwiIiwwLHsiY29sb3VyIjpbMzAsNjAsNjBdfV0sWzQsNywiIiwwLHsiY29sb3VyIjpbMzAsNjAsNjBdfV0sWzcsMTAsIlxcZGVsdGFfMiIsMCx7ImNvbG91ciI6WzMwLDYwLDYwXSwic3R5bGUiOnsiYm9keSI6eyJuYW1lIjoiZGFzaGVkIn19fSxbMzAsNjAsNjAsMV1dLFs2LDUsIiIsMCx7ImNvbG91ciI6WzI0MCw2MCw2MF19XSxbNSw3LCIiLDAseyJjb2xvdXIiOlsyNDAsNjAsNjBdfV0sWzcsOSwiXFxkZWx0YV8xIiwwLHsiY29sb3VyIjpbMjQwLDYwLDYwXSwic3R5bGUiOnsiYm9keSI6eyJuYW1lIjoiZGFzaGVkIn19fSxbMjQwLDYwLDYwLDFdXV0=
\begin{tikzcd}[ampersand replacement=\&]
	\& \cdot \& \cdot \\
	\cdot \& \cdot \& \cdot \& {\, } \\
	\cdot \& \cdot \& \cdot \& {\, } \\
	\& {\, } \& {\, }
	\arrow[equals, from=1-2, to=1-3]
	\arrow[dashed, from=1-2, to=2-2]
	\arrow[color={rgb,255:red,214;green,153;blue,92}, from=1-3, to=2-3]
	\arrow[color={rgb,255:red,214;green,153;blue,92}, dashed, from=2-1, to=2-2]
	\arrow[equals, from=2-1, to=3-1]
	\arrow[color={rgb,255:red,214;green,153;blue,92}, dashed, from=2-2, to=2-3]
	\arrow[dashed, from=2-2, to=3-2]
	\arrow["{\delta_1'}", color={rgb,255:red,214;green,153;blue,92}, dashed, from=2-3, to=2-4]
	\arrow[color={rgb,255:red,214;green,153;blue,92}, from=2-3, to=3-3]
	\arrow[draw={rgb,255:red,92;green,92;blue,214}, from=3-1, to=3-2]
	\arrow[draw={rgb,255:red,92;green,92;blue,214}, from=3-2, to=3-3]
	\arrow["{\delta_2'}", dashed, from=3-2, to=4-2]
	\arrow["{\delta_1}", color={rgb,255:red,92;green,92;blue,214}, dashed, from=3-3, to=3-4]
	\arrow["{\delta_2}", color={rgb,255:red,214;green,153;blue,92}, dashed, from=3-3, to=4-3]
\end{tikzcd}.
        \end{equation}
        若 $\delta_2$ 与 $\delta_1'$ 的实现属于 $\xi$, 则 $\delta_1$ 的实现也属于 $\xi$.
    \end{enumerate}
\end{definition}

\begin{remark}
    \Cref{def:xi} 中的``真类 (proper class)''无关集合论中的``真类''概念.
\end{remark}

\begin{notation}
    方便起见, 将 $\xi$ 中三角标注作橘色 ($\color{rgb,255:red,214;green,153;blue,92}\blacksquare$). 引入 $\xi$-三角, $\xi$-扩张元, $\xi$-单射, $\xi$-满射等名称. 
\end{notation}

\begin{theorem}\label{thm:proper_class_extriangulated}
    假定外三角的类 $\xi$ 对同构封闭, 则 $\xi$ 是真类当且仅当 $(\mathcal{C}, \mathbb E_\xi, \mathfrak s_\xi)$ 是外三角范畴. 此处
    \begin{enumerate}
        \item $\mathbb E_\xi (Z,X)$ 是 $\mathbb E(Z,X)$ 中的 $\xi$-扩张元;
        \item $\mathfrak s_\xi$ 即 $\mathfrak s$ 在 $\mathbb E_\xi$ 上的限制.
    \end{enumerate}
    \begin{proof}
		方便起见, 将外三角范畴 $(\mathcal{C}, \mathbb E, \mathfrak s)$ 与带特定结构的加法范畴 $(\mathcal{C}, \mathbb E_\xi, \mathfrak s)$ 分别称作大范畴与小范畴. 
		\\
        ($\gets$). 假定小范畴是外三角范畴, 下证 $\xi$ 是真类. P1 与 P2 由 $\mathbb E_\xi$ 的加法双函子性保证. P3 证明如下. 假定 \cref{eq:P3} 中 $\delta_1'$, $\delta_2$ 是 $\xi$-扩张元, 由 P2 知 $\delta_2'$ 也是 $\xi$-扩张元. 由图表定理, $\delta_1$ 可在小范畴中直接构造. 

        ($\to$). ET1, ET2, ET3, ET3$^{\mathrm{op}}$ 显然成立. 只需验证 ET4 与 ET4$^{\mathrm{op}}$.
        \begin{enumerate}
            \item (ET4 的验证). 给定 $\xi$-单射 $s$ 与 $f$ 的复合, 在大范畴中作同伦的 ET4 图
            \begin{equation}
% https://q.uiver.app/#q=WzAsMTIsWzAsMCwiWCJdLFsxLDAsIlkiXSxbMiwwLCJaIl0sWzEsMSwiQSJdLFsxLDIsIkMiXSxbMCwxLCJYIl0sWzIsMSwiQiJdLFsyLDIsIkMiXSxbMywwLCJcXCwiXSxbMywxLCJcXCwiXSxbMSwzLCJcXCwiXSxbMiwzLCJcXCwiXSxbMSwyLCJnIiwwLHsiY29sb3VyIjpbMzAsNjAsNjBdfSxbMzAsNjAsNjAsMV1dLFsxLDMsInMiLDAseyJjb2xvdXIiOlszMCw2MCw2MF19LFszMCw2MCw2MCwxXV0sWzMsNCwidCIsMCx7ImNvbG91ciI6WzMwLDYwLDYwXX0sWzMwLDYwLDYwLDFdXSxbMCwxLCJmIiwwLHsiY29sb3VyIjpbMzAsNjAsNjBdfSxbMzAsNjAsNjAsMV1dLFswLDUsIiIsMCx7ImxldmVsIjoyLCJzdHlsZSI6eyJoZWFkIjp7Im5hbWUiOiJub25lIn19fV0sWzQsNywiIiwwLHsibGV2ZWwiOjIsInN0eWxlIjp7ImhlYWQiOnsibmFtZSI6Im5vbmUifX19XSxbMiw4LCJcXGRlbHRhIiwwLHsiY29sb3VyIjpbMzAsNjAsNjBdLCJzdHlsZSI6eyJib2R5Ijp7Im5hbWUiOiJkYXNoZWQifX19LFszMCw2MCw2MCwxXV0sWzUsMywic2YiLDAseyJzdHlsZSI6eyJib2R5Ijp7Im5hbWUiOiJkYXNoZWQifX19XSxbMyw2LCJwIiwwLHsic3R5bGUiOnsiYm9keSI6eyJuYW1lIjoiZGFzaGVkIn19fV0sWzYsOSwiXFx2YXJlcHNpbG9uICIsMCx7InN0eWxlIjp7ImJvZHkiOnsibmFtZSI6ImRhc2hlZCJ9fX1dLFs0LDEwLCJcXGV0YSIsMCx7ImNvbG91ciI6WzMwLDYwLDYwXSwic3R5bGUiOnsiYm9keSI6eyJuYW1lIjoiZGFzaGVkIn19fSxbMzAsNjAsNjAsMV1dLFsyLDYsIm0iLDAseyJzdHlsZSI6eyJib2R5Ijp7Im5hbWUiOiJkYXNoZWQifX19XSxbNiw3LCJuIiwwLHsic3R5bGUiOnsiYm9keSI6eyJuYW1lIjoiZGFzaGVkIn19fV0sWzcsMTEsIlxca2FwcGEiLDAseyJzdHlsZSI6eyJib2R5Ijp7Im5hbWUiOiJkYXNoZWQifX19XSxbMSw2LCJcXHNxdWFyZSIsMSx7InN0eWxlIjp7ImJvZHkiOnsibmFtZSI6Im5vbmUifSwiaGVhZCI6eyJuYW1lIjoibm9uZSJ9fX1dXQ==
\begin{tikzcd}[ampersand replacement=\&]
	X \& Y \& Z \& {\,} \\
	X \& A \& B \& {\,} \\
	\& C \& C \\
	\& {\,} \& {\,}
	\arrow["f", color={rgb,255:red,214;green,153;blue,92}, from=1-1, to=1-2]
	\arrow[equals, from=1-1, to=2-1]
	\arrow["g", color={rgb,255:red,214;green,153;blue,92}, from=1-2, to=1-3]
	\arrow["s", color={rgb,255:red,214;green,153;blue,92}, from=1-2, to=2-2]
	\arrow["\square"{description}, draw=none, from=1-2, to=2-3]
	\arrow["\delta", color={rgb,255:red,214;green,153;blue,92}, dashed, from=1-3, to=1-4]
	\arrow["m", dashed, from=1-3, to=2-3]
	\arrow["sf", dashed, from=2-1, to=2-2]
	\arrow["p", dashed, from=2-2, to=2-3]
	\arrow["t", color={rgb,255:red,214;green,153;blue,92}, from=2-2, to=3-2]
	\arrow["{\varepsilon }", dashed, from=2-3, to=2-4]
	\arrow["n", dashed, from=2-3, to=3-3]
	\arrow[equals, from=3-2, to=3-3]
	\arrow["\eta", color={rgb,255:red,214;green,153;blue,92}, dashed, from=3-2, to=4-2]
	\arrow["\kappa", dashed, from=3-3, to=4-3]
\end{tikzcd}.
            \end{equation}
            直接地, $\kappa = g_\ast \eta$ 与 $n^\ast \eta = f_\ast \varepsilon$ 都是 $\xi$-扩张元. 在大范畴中作``拉回'':
\begin{equation}
	% https://q.uiver.app/#q=WzAsMTIsWzIsMCwiWSAiXSxbMiwxLCJaIFxcb3BsdXMgQSJdLFsyLDIsIkIiXSxbMiwzLCJcXCwiXSxbMCwyLCJYIl0sWzEsMiwiQSJdLFszLDIsIlxcLCJdLFsxLDAsIlkiXSxbMCwxLCJYIl0sWzEsMSwiWSBcXG9wbHVzIEEiXSxbMywxLCJcXCwiXSxbMSwzLCJcXCwiXSxbMCwxLCJcXGJpbm9te2d9e3N9IiwwLHsiY29sb3VyIjpbMzAsNjAsNjBdfSxbMzAsNjAsNjAsMV1dLFsxLDIsIigtbSxwKSIsMCx7ImNvbG91ciI6WzMwLDYwLDYwXX0sWzMwLDYwLDYwLDFdXSxbMiwzLCJuXlxcYXN0IFxcZXRhIiwwLHsiY29sb3VyIjpbMzAsNjAsNjBdLCJzdHlsZSI6eyJib2R5Ijp7Im5hbWUiOiJkYXNoZWQifX19LFszMCw2MCw2MCwxXV0sWzQsNSwic2YiXSxbNSwyLCJwIl0sWzIsNiwiXFx2YXJlcHNpbG9uICIsMCx7InN0eWxlIjp7ImJvZHkiOnsibmFtZSI6ImRhc2hlZCJ9fX1dLFs4LDQsIiIsMCx7ImxldmVsIjoyLCJzdHlsZSI6eyJoZWFkIjp7Im5hbWUiOiJub25lIn19fV0sWzcsMCwiIiwwLHsibGV2ZWwiOjIsInN0eWxlIjp7ImhlYWQiOnsibmFtZSI6Im5vbmUifX19XSxbNyw5LCJcXGJpbm9tIDEwIiwwLHsiY29sb3VyIjpbMzAsNjAsNjBdLCJzdHlsZSI6eyJib2R5Ijp7Im5hbWUiOiJkYXNoZWQifX19LFszMCw2MCw2MCwxXV0sWzksNSwiKDAsMSkiLDAseyJjb2xvdXIiOlszMCw2MCw2MF0sInN0eWxlIjp7ImJvZHkiOnsibmFtZSI6ImRhc2hlZCJ9fX0sWzMwLDYwLDYwLDFdXSxbMSwxMCwiXFxkZWx0YSBcXG9wbHVzIDAiLDAseyJjb2xvdXIiOlszMCw2MCw2MF0sInN0eWxlIjp7ImJvZHkiOnsibmFtZSI6ImRhc2hlZCJ9fX0sWzMwLDYwLDYwLDFdXSxbNSwxMSwiMCIsMCx7ImNvbG91ciI6WzMwLDYwLDYwXSwic3R5bGUiOnsiYm9keSI6eyJuYW1lIjoiZGFzaGVkIn19fSxbMzAsNjAsNjAsMV1dLFs4LDksIlxcYmlub20ge2Z9ezB9IiwwLHsiY29sb3VyIjpbMzAsNjAsNjBdLCJzdHlsZSI6eyJib2R5Ijp7Im5hbWUiOiJkYXNoZWQifX19LFszMCw2MCw2MCwxXV0sWzksMSwiZ1xcb3BsdXMgMV9BIiwwLHsiY29sb3VyIjpbMzAsNjAsNjBdLCJzdHlsZSI6eyJib2R5Ijp7Im5hbWUiOiJkYXNoZWQifX19LFszMCw2MCw2MCwxXV1d
\begin{tikzcd}[ampersand replacement=\&]
	\& Y \& {Y } \\
	X \& {Y \oplus A} \& {Z \oplus A} \& {\,} \\
	X \& A \& B \& {\,} \\
	\& {\,} \& {\,}
	\arrow[equals, from=1-2, to=1-3]
	\arrow["{\binom 10}", color={rgb,255:red,214;green,153;blue,92}, dashed, from=1-2, to=2-2]
	\arrow["{\binom{g}{s}}", color={rgb,255:red,214;green,153;blue,92}, from=1-3, to=2-3]
	\arrow["{\binom {f}{0}}", color={rgb,255:red,214;green,153;blue,92}, dashed, from=2-1, to=2-2]
	\arrow[equals, from=2-1, to=3-1]
	\arrow["{g\oplus 1_A}", color={rgb,255:red,214;green,153;blue,92}, dashed, from=2-2, to=2-3]
	\arrow["{(0,1)}", color={rgb,255:red,214;green,153;blue,92}, dashed, from=2-2, to=3-2]
	\arrow["{\delta \oplus 0}", color={rgb,255:red,214;green,153;blue,92}, dashed, from=2-3, to=2-4]
	\arrow["{(-m,p)}", color={rgb,255:red,214;green,153;blue,92}, from=2-3, to=3-3]
	\arrow["sf", from=3-1, to=3-2]
	\arrow["p", from=3-2, to=3-3]
	\arrow["0", color={rgb,255:red,214;green,153;blue,92}, dashed, from=3-2, to=4-2]
	\arrow["{\varepsilon }", dashed, from=3-3, to=3-4]
	\arrow["{n^\ast \eta}", color={rgb,255:red,214;green,153;blue,92}, dashed, from=3-3, to=4-3]
\end{tikzcd}.
\end{equation}
            由 P3 得 $X \xrightarrow{sf} A \xrightarrow p Y \overset {\delta} \dashrightarrow$ 是 $\xi$-三角.
            \item (ET4$^{\mathrm{op}}$ 的验证). 给定 $\xi$-满射 $n$ 与 $p$ 的复合, 在大范畴中作同伦的 ET4$^{\mathrm{op}}$ 图
            \begin{equation}
% https://q.uiver.app/#q=WzAsMTIsWzAsMCwiWCJdLFsxLDAsIlkiXSxbMiwwLCJaIl0sWzEsMSwiQSJdLFsxLDIsIkMiXSxbMCwxLCJYIl0sWzIsMSwiQiJdLFsyLDIsIkMiXSxbMywwLCJcXCwiXSxbMywxLCJcXCwiXSxbMSwzLCJcXCwiXSxbMiwzLCJcXCwiXSxbMSwyLCJnIiwwLHsic3R5bGUiOnsiYm9keSI6eyJuYW1lIjoiZGFzaGVkIn19fV0sWzEsMywicyIsMCx7InN0eWxlIjp7ImJvZHkiOnsibmFtZSI6ImRhc2hlZCJ9fX1dLFszLDQsIm5wIiwwLHsic3R5bGUiOnsiYm9keSI6eyJuYW1lIjoiZGFzaGVkIn19fV0sWzAsMSwiZiIsMCx7InN0eWxlIjp7ImJvZHkiOnsibmFtZSI6ImRhc2hlZCJ9fX1dLFswLDUsIiIsMCx7ImxldmVsIjoyLCJzdHlsZSI6eyJoZWFkIjp7Im5hbWUiOiJub25lIn19fV0sWzQsNywiIiwwLHsibGV2ZWwiOjIsInN0eWxlIjp7ImhlYWQiOnsibmFtZSI6Im5vbmUifX19XSxbMiw4LCJcXGRlbHRhIiwwLHsic3R5bGUiOnsiYm9keSI6eyJuYW1lIjoiZGFzaGVkIn19fV0sWzUsMywiaSIsMCx7ImNvbG91ciI6WzMwLDYwLDYwXX0sWzMwLDYwLDYwLDFdXSxbMyw2LCJwIiwwLHsiY29sb3VyIjpbMzAsNjAsNjBdfSxbMzAsNjAsNjAsMV1dLFs2LDksIlxcdmFyZXBzaWxvbiAiLDAseyJjb2xvdXIiOlszMCw2MCw2MF0sInN0eWxlIjp7ImJvZHkiOnsibmFtZSI6ImRhc2hlZCJ9fX0sWzMwLDYwLDYwLDFdXSxbNCwxMCwiXFxldGEiLDAseyJzdHlsZSI6eyJib2R5Ijp7Im5hbWUiOiJkYXNoZWQifX19XSxbMiw2LCJtIiwwLHsiY29sb3VyIjpbMzAsNjAsNjBdfSxbMzAsNjAsNjAsMV1dLFs2LDcsIm4iLDAseyJjb2xvdXIiOlszMCw2MCw2MF19LFszMCw2MCw2MCwxXV0sWzcsMTEsIlxca2FwcGEiLDAseyJjb2xvdXIiOlszMCw2MCw2MF0sInN0eWxlIjp7ImJvZHkiOnsibmFtZSI6ImRhc2hlZCJ9fX0sWzMwLDYwLDYwLDFdXSxbMSw2LCJcXHNxdWFyZSIsMSx7InN0eWxlIjp7ImJvZHkiOnsibmFtZSI6Im5vbmUifSwiaGVhZCI6eyJuYW1lIjoibm9uZSJ9fX1dXQ==
\begin{tikzcd}[ampersand replacement=\&]
	X \& Y \& Z \& {\,} \\
	X \& A \& B \& {\,} \\
	\& C \& C \\
	\& {\,} \& {\,}
	\arrow["f", dashed, from=1-1, to=1-2]
	\arrow[equals, from=1-1, to=2-1]
	\arrow["g", dashed, from=1-2, to=1-3]
	\arrow["s", dashed, from=1-2, to=2-2]
	\arrow["\square"{description}, draw=none, from=1-2, to=2-3]
	\arrow["\delta", dashed, from=1-3, to=1-4]
	\arrow["m", color={rgb,255:red,214;green,153;blue,92}, from=1-3, to=2-3]
	\arrow["i", color={rgb,255:red,214;green,153;blue,92}, from=2-1, to=2-2]
	\arrow["p", color={rgb,255:red,214;green,153;blue,92}, from=2-2, to=2-3]
	\arrow["np", dashed, from=2-2, to=3-2]
	\arrow["{\varepsilon }", color={rgb,255:red,214;green,153;blue,92}, dashed, from=2-3, to=2-4]
	\arrow["n", color={rgb,255:red,214;green,153;blue,92}, from=2-3, to=3-3]
	\arrow[equals, from=3-2, to=3-3]
	\arrow["\eta", dashed, from=3-2, to=4-2]
	\arrow["\kappa", color={rgb,255:red,214;green,153;blue,92}, dashed, from=3-3, to=4-3]
\end{tikzcd}.
            \end{equation}
            直接地, $\delta = m^\ast \varepsilon$ 与 $n^\ast \eta = f_\ast \varepsilon$ 都是 $\xi$-扩张元. 在大范畴中作``拉回'':
\begin{equation}
	% https://q.uiver.app/#q=WzAsMTIsWzAsMiwiWSJdLFsxLDIsIkEiXSxbMiwyLCJDIl0sWzIsMSwiQiJdLFsyLDAsIloiXSxbMCwxLCJZIl0sWzEsMSwiQSBcXG9wbHVzIFoiXSxbMSwwLCJaIl0sWzMsMiwiXFwsIl0sWzIsMywiXFwsIl0sWzMsMSwiXFwsIl0sWzEsMywiXFwsIl0sWzAsMSwicyJdLFsxLDIsIm5wIl0sWzIsOCwiXFxldGEiLDAseyJzdHlsZSI6eyJib2R5Ijp7Im5hbWUiOiJkYXNoZWQifX19XSxbNCwzLCJtIiwwLHsiY29sb3VyIjpbMzAsNjAsNjBdfSxbMzAsNjAsNjAsMV1dLFszLDIsIm4iLDAseyJjb2xvdXIiOlszMCw2MCw2MF19LFszMCw2MCw2MCwxXV0sWzIsOSwiXFxrYXBwYSIsMCx7ImNvbG91ciI6WzMwLDYwLDYwXSwic3R5bGUiOnsiYm9keSI6eyJuYW1lIjoiZGFzaGVkIn19fSxbMzAsNjAsNjAsMV1dLFs1LDAsIiIsMCx7ImxldmVsIjoyLCJzdHlsZSI6eyJoZWFkIjp7Im5hbWUiOiJub25lIn19fV0sWzcsNCwiIiwwLHsibGV2ZWwiOjIsInN0eWxlIjp7ImhlYWQiOnsibmFtZSI6Im5vbmUifX19XSxbNyw2LCJcXGJpbm9tIDAxIiwwLHsiY29sb3VyIjpbMzAsNjAsNjBdLCJzdHlsZSI6eyJib2R5Ijp7Im5hbWUiOiJkYXNoZWQifX19LFszMCw2MCw2MCwxXV0sWzYsMSwiKDEsMCkiLDAseyJjb2xvdXIiOlszMCw2MCw2MF0sInN0eWxlIjp7ImJvZHkiOnsibmFtZSI6ImRhc2hlZCJ9fX0sWzMwLDYwLDYwLDFdXSxbNSw2LCJcXGJpbm9tIHtzfXstZ30iLDAseyJjb2xvdXIiOlszMCw2MCw2MF0sInN0eWxlIjp7ImJvZHkiOnsibmFtZSI6ImRhc2hlZCJ9fX0sWzMwLDYwLDYwLDFdXSxbMywxMCwiZl9cXGFzdCBcXHZhcmVwc2lsb24gIiwwLHsiY29sb3VyIjpbMzAsNjAsNjBdLCJzdHlsZSI6eyJib2R5Ijp7Im5hbWUiOiJkYXNoZWQifX19LFszMCw2MCw2MCwxXV0sWzEsMTEsIjAiLDAseyJjb2xvdXIiOlszMCw2MCw2MF0sInN0eWxlIjp7ImJvZHkiOnsibmFtZSI6ImRhc2hlZCJ9fX0sWzMwLDYwLDYwLDFdXSxbNiwzLCIocCxtKSIsMCx7ImNvbG91ciI6WzMwLDYwLDYwXSwic3R5bGUiOnsiYm9keSI6eyJuYW1lIjoiZGFzaGVkIn19fSxbMzAsNjAsNjAsMV1dXQ==
\begin{tikzcd}[ampersand replacement=\&]
	\& Z \& Z \\
	Y \& {A \oplus Z} \& B \& {\,} \\
	Y \& A \& C \& {\,} \\
	\& {\,} \& {\,}
	\arrow[equals, from=1-2, to=1-3]
	\arrow["{\binom 01}", color={rgb,255:red,214;green,153;blue,92}, dashed, from=1-2, to=2-2]
	\arrow["m", color={rgb,255:red,214;green,153;blue,92}, from=1-3, to=2-3]
	\arrow["{\binom {s}{-g}}", color={rgb,255:red,214;green,153;blue,92}, dashed, from=2-1, to=2-2]
	\arrow[equals, from=2-1, to=3-1]
	\arrow["{(p,m)}", color={rgb,255:red,214;green,153;blue,92}, dashed, from=2-2, to=2-3]
	\arrow["{(1,0)}", color={rgb,255:red,214;green,153;blue,92}, dashed, from=2-2, to=3-2]
	\arrow["{f_\ast \varepsilon }", color={rgb,255:red,214;green,153;blue,92}, dashed, from=2-3, to=2-4]
	\arrow["n", color={rgb,255:red,214;green,153;blue,92}, from=2-3, to=3-3]
	\arrow["s", from=3-1, to=3-2]
	\arrow["np", from=3-2, to=3-3]
	\arrow["0", color={rgb,255:red,214;green,153;blue,92}, dashed, from=3-2, to=4-2]
	\arrow["\eta", dashed, from=3-3, to=3-4]
	\arrow["\kappa", color={rgb,255:red,214;green,153;blue,92}, dashed, from=3-3, to=4-3]
\end{tikzcd}.
\end{equation}
            由 P3 得 $Y \xrightarrow{s} A \xrightarrow{np} C \overset \eta \dashrightarrow$ 是 $\xi$-三角.
        \end{enumerate}
    \end{proof}
\end{theorem}

\begin{proposition}
    若 $(\mathcal{C}, \mathbb E, \mathfrak s)$ 弱幂等完备, 则 $(\mathcal{C},\mathbb E_\xi, \mathfrak s_\xi)$ 亦然.
    \begin{proof}
        即证, 若 $\binom i0 : A \to B \oplus C$ 是 $\xi$-单射, 则 $i$ 是 $\xi$-单射. 由 WIC 条件, $i$ 是 $\mathbb E$-单射, 从而有交换图
        \begin{equation}
% https://q.uiver.app/#q=WzAsOCxbMCwwLCJBIl0sWzAsMSwiQSJdLFsxLDEsIkIgXFxvcGx1cyBDIl0sWzEsMCwiQiJdLFsyLDEsIk0iXSxbMiwwLCJNJyJdLFszLDAsIlxcLCJdLFszLDEsIlxcLCJdLFswLDMsImkiXSxbMyw1LCJwIl0sWzMsMiwiXFxiaW5vbSAxMCJdLFswLDEsIiIsMCx7ImxldmVsIjoyLCJzdHlsZSI6eyJoZWFkIjp7Im5hbWUiOiJub25lIn19fV0sWzEsMiwiXFxiaW5vbSBpMCIsMCx7ImNvbG91ciI6WzMwLDYwLDYwXX0sWzMwLDYwLDYwLDFdXSxbMiw0LCIiLDAseyJjb2xvdXIiOlszMCw2MCw2MF19XSxbNSw2LCJcXGRlbHRhJyIsMCx7InN0eWxlIjp7ImJvZHkiOnsibmFtZSI6ImRhc2hlZCJ9fX1dLFs0LDcsIlxcZGVsdGEiLDAseyJjb2xvdXIiOlszMCw2MCw2MF0sInN0eWxlIjp7ImJvZHkiOnsibmFtZSI6ImRhc2hlZCJ9fX0sWzMwLDYwLDYwLDFdXSxbNSw0LCJcXGdhbW1hIiwwLHsic3R5bGUiOnsiYm9keSI6eyJuYW1lIjoiZGFzaGVkIn19fV1d
\begin{tikzcd}[ampersand replacement=\&]
	A \& B \& {M'} \& {\,} \\
	A \& {B \oplus C} \& M \& {\,}
	\arrow["i", from=1-1, to=1-2]
	\arrow[equals, from=1-1, to=2-1]
	\arrow["p", from=1-2, to=1-3]
	\arrow["{\binom 10}", from=1-2, to=2-2]
	\arrow["{\delta'}", dashed, from=1-3, to=1-4]
	\arrow["\gamma", dashed, from=1-3, to=2-3]
	\arrow["{\binom i0}", color={rgb,255:red,214;green,153;blue,92}, from=2-1, to=2-2]
	\arrow[color={rgb,255:red,214;green,153;blue,92}, from=2-2, to=2-3]
	\arrow["\delta", color={rgb,255:red,214;green,153;blue,92}, dashed, from=2-3, to=2-4]
\end{tikzcd}.
        \end{equation} 
        由 ET3 构造 $\gamma$. 此时 $\delta' = \gamma^\ast \delta$ 是 $\xi$-扩张元, 因此 $i$ 是 $\xi$-单射.
    \end{proof}
\end{proposition}

\subsection{\texorpdfstring{$\xi$}{}-同调代数}

\begin{notation}
    往后假定范畴弱幂等完备. 取定真类 $\xi$.
\end{notation}

\begin{definition}[$\xi$-投射对象]
    称 $X$ 是 $\xi$-投射对象, 若对一切 $\xi$-三角 $A \xrightarrow f B \xrightarrow g C \overset \delta \dashrightarrow$, 有短正合列
    \begin{equation}
        0 \to (X,A) \xrightarrow {(X,f)} (X,B) \xrightarrow {(X,g)} (X,C) \to 0.
    \end{equation}
\end{definition}

\begin{remark}
    传统的相对同调代数仅要求 $(X,-)$ 映 $\xi$-满射为满射, 对单射不设要求.
\end{remark}

\begin{notation}
    记 $\mathcal{P}(\xi)$ 是 $\xi$-投射对象构成的类.
\end{notation}

\begin{remark}
    其对余积 (若存在) 与缩回封闭. $\mathcal{P}(\xi)$ 显然是 $(\mathcal{C}, \mathbb E_\xi, \mathfrak s_\xi)$ 的外三角子范畴, 且是半单 Abel 范畴; 但是 $\mathcal{P}(\xi)$ 未必是 $(\mathcal{C}, \mathbb E, \mathfrak s)$ 的外三角子范畴.
\end{remark}

\begin{notation}
    假定外三角范畴有足够 $\mathcal{P}(\xi)$-投射对象. 即是说, 对任意对象 $A$ 都存在 $\xi$-三角 $K \xrightarrow f P \xrightarrow g A \overset \delta \dashrightarrow$ 使得 $P$ 是 $\mathcal{P}(\xi)$-投射对象.
\end{notation}

\begin{proposition}
	$\mathcal{P}(\xi)$-投射预盖恰好是 $\mathcal{P}(\xi)$ 出发的 $\xi$-满射.
\begin{proof}
	任取 $\mathcal{P}(\xi)$-投射预盖 $q : Q \to A$. 存在 $\xi$-三角 $K \xrightarrow f P \xrightarrow g A \overset \delta \dashrightarrow$. 由投射预盖的定义, $g$ 经 $q$ 分解. 由 WIC, $q$ 是 $\xi$-满射. 反之, 设 $q : Q \to A$ 是 $\mathcal{P}(\xi)$ 出发的 $\xi$-满射. 由定义知 $(-, q)$ 映 $\mathcal{P}(\xi)$ 为满射, 从而 $q$ 是 $\mathcal{P}(\xi)$-投射预盖.
\end{proof}
\end{proposition}

下一引理说明 $\xi$-三角与 $\mathcal{P}(\xi)$-投射对象通过某种``垂直配对''决定彼此.

\begin{proposition}
    假定范畴有足够 $\mathcal{P}(\xi)$-投射对象. 则 $\mathbb E$-三角 $X \xrightarrow f Y \xrightarrow g Z \overset \delta \dashrightarrow$ 是 $\xi$-三角, 当且仅当对所有 $\mathcal{P}(\xi)$-投射对象 $P$ 都有短正合列
    \begin{equation}
        0 \to (P,X) \xrightarrow {(P,f)} (P,Y) \xrightarrow {(P,g)} (P,Z) \to 0.
    \end{equation}
    \begin{proof}
        仅证明 ($\gets$) 部分. 给定 $\mathbb E$-三角 $X \xrightarrow f Y \xrightarrow g Z \overset \delta \dashrightarrow$. 由范畴有足够 $\mathcal{P}(\xi)$-投射对象, 作交换图
        \begin{equation}
% https://q.uiver.app/#q=WzAsMTIsWzAsMiwiWCJdLFsxLDIsIlkiXSxbMiwyLCJaIl0sWzMsMiwiXFwsIl0sWzIsMSwiUCJdLFsyLDAsIksiXSxbMiwzLCJcXCwiXSxbMSwwLCJLIl0sWzAsMSwiWCJdLFsxLDEsIlggXFxvcGx1cyBQIl0sWzEsMywiXFwsIl0sWzMsMSwiXFwsIl0sWzAsMSwiZiJdLFsxLDIsImciXSxbMiwzLCJcXGRlbHRhIiwwLHsic3R5bGUiOnsiYm9keSI6eyJuYW1lIjoiZGFzaGVkIn19fV0sWzUsNCwiIiwwLHsiY29sb3VyIjpbMzAsNjAsNjBdfV0sWzQsMiwiIiwwLHsiY29sb3VyIjpbMzAsNjAsNjBdfV0sWzIsNiwiIiwwLHsiY29sb3VyIjpbMzAsNjAsNjBdLCJzdHlsZSI6eyJib2R5Ijp7Im5hbWUiOiJkYXNoZWQifX19XSxbOCwwLCIiLDAseyJsZXZlbCI6Miwic3R5bGUiOnsiaGVhZCI6eyJuYW1lIjoibm9uZSJ9fX1dLFs3LDUsIiIsMCx7ImxldmVsIjoyLCJzdHlsZSI6eyJoZWFkIjp7Im5hbWUiOiJub25lIn19fV0sWzgsOSwiIiwyLHsiY29sb3VyIjpbMzAsNjAsNjBdLCJzdHlsZSI6eyJib2R5Ijp7Im5hbWUiOiJkYXNoZWQifX19XSxbOSw0LCIiLDIseyJjb2xvdXIiOlszMCw2MCw2MF0sInN0eWxlIjp7ImJvZHkiOnsibmFtZSI6ImRhc2hlZCJ9fX1dLFs3LDksIiIsMix7InN0eWxlIjp7ImJvZHkiOnsibmFtZSI6ImRhc2hlZCJ9fX1dLFs5LDEsIiIsMix7InN0eWxlIjp7ImJvZHkiOnsibmFtZSI6ImRhc2hlZCJ9fX1dLFsxLDEwLCIiLDIseyJzdHlsZSI6eyJib2R5Ijp7Im5hbWUiOiJkYXNoZWQifX19XSxbNCwxMSwiIiwyLHsiY29sb3VyIjpbMzAsNjAsNjBdLCJzdHlsZSI6eyJib2R5Ijp7Im5hbWUiOiJkYXNoZWQifX19XV0=
\begin{tikzcd}[ampersand replacement=\&]
	\& K \& K \\
	X \& {X \oplus P} \& P \& {\,} \\
	X \& Y \& Z \& {\,} \\
	\& {\,} \& {\,}
	\arrow[equals, from=1-2, to=1-3]
	\arrow[dashed, from=1-2, to=2-2]
	\arrow[draw={rgb,255:red,214;green,153;blue,92}, from=1-3, to=2-3]
	\arrow[draw={rgb,255:red,214;green,153;blue,92}, dashed, from=2-1, to=2-2]
	\arrow[equals, from=2-1, to=3-1]
	\arrow[draw={rgb,255:red,214;green,153;blue,92}, dashed, from=2-2, to=2-3]
	\arrow[dashed, from=2-2, to=3-2]
	\arrow[draw={rgb,255:red,214;green,153;blue,92}, dashed, from=2-3, to=2-4]
	\arrow[draw={rgb,255:red,214;green,153;blue,92}, from=2-3, to=3-3]
	\arrow["f", from=3-1, to=3-2]
	\arrow["g", from=3-2, to=3-3]
	\arrow[dashed, from=3-2, to=4-2]
	\arrow["\delta", dashed, from=3-3, to=3-4]
	\arrow[draw={rgb,255:red,214;green,153;blue,92}, dashed, from=3-3, to=4-3]
\end{tikzcd}.
        \end{equation}
    	可裂短正合列是 $\xi$-三角. 由 P3, $X \xrightarrow f Y \xrightarrow g Z \overset \delta \dashrightarrow$ 是 $\xi$-三角.
    \end{proof}
\end{proposition}

\begin{notation}
    假定范畴有足够 $\mathcal{P}(\xi)$-投射对象. 此时可以定义投射分解与维数.
\end{notation}

\begin{definition}[$\mathcal{P}(\xi)$-投射维数 $pd_\xi$]
    约定 $\mathcal{P}(\xi)$ 中对象的 $\mathcal{P}(\xi)$-投射维数为 $0$. 归纳地, 称 $pd_\xi M \leq n + 1$, 若存在 $\xi$-三角 $K \xrightarrow f P \xrightarrow g M \overset \delta \dashrightarrow$ 使得 $P \in \mathcal{P}(\xi)$ 且 $pd_\xi K \leq n$. 称 $pd_\xi M = n$, 若有 $pd_\xi M \leq n$, 但没有 $pd_\xi M \leq n-1$.
\end{definition}

\begin{lemma}\label{lem:pd_smd}
	对 $P \in \mathcal{P}(\xi)$, 总有 $pd_\xi M = pd_\xi (M \oplus P)$.
	\begin{proof}
		记命题 $\mathfrak M (n)$ 为 ``$pd_\xi M = pd_\xi (M \oplus P)$ 对 $pd_\xi M = n$ 成立'', 证明 $\{\mathfrak M(n)\}_{n \in \mathbb N\cup \{\infty\}}$. 由定义知 $\mathfrak M(0)$ 成立. 假定 $\mathfrak M(\leq n)$ 成立. 今任取 $pd_\xi M = n+1$.
		\begin{enumerate}
			\item 一方面, $pd_\xi (M \oplus P) \leq n + 1$. 取 $\xi$-三角 $K \xrightarrow f Q \xrightarrow g M \overset \delta \dashrightarrow$ 使得 $Q \in \mathcal{P}(\xi)$ 且 $pd_\xi K \leq n$. 则有 $\xi$-三角 $K \xrightarrow {\binom f0} Q \oplus P \xrightarrow {(g,0)} M \oplus P \overset {\delta'} \dashrightarrow$.
			\item 另一方面, $pd_\xi (M \oplus P) \not\leq n$. 假定 $pd_\xi (M \oplus P) \leq n$. 取 $\xi$-三角 $K' \xrightarrow {f'} Q' \xrightarrow {g'} M \oplus P \overset {\delta'} \dashrightarrow$ 使得 $Q' \in \mathcal{P}(\xi)$ 且 $pd_\xi K' \leq (n-1)$. 作交换图
			\begin{equation}
				% https://q.uiver.app/#q=WzAsMTIsWzAsMSwiSyciXSxbMSwxLCJRJyJdLFsyLDEsIk0gXFxvcGx1cyBQIl0sWzMsMSwiXFwsIl0sWzIsMCwiUCJdLFsyLDIsIk0iXSxbMSwwLCJQIl0sWzAsMiwiSyciXSxbMiwzLCJcXCwiXSxbMSwzLCJcXCwiXSxbMywyLCJcXCwiXSxbMSwyLCJFIl0sWzAsMSwiZiciLDAseyJjb2xvdXIiOlszMCw2MCw2MF19LFszMCw2MCw2MCwxXV0sWzEsMiwiZyciLDAseyJjb2xvdXIiOlszMCw2MCw2MF19LFszMCw2MCw2MCwxXV0sWzIsMywiXFxkZWx0YSciLDAseyJjb2xvdXIiOlszMCw2MCw2MF0sInN0eWxlIjp7ImJvZHkiOnsibmFtZSI6ImRhc2hlZCJ9fX0sWzMwLDYwLDYwLDFdXSxbNiw0LCIiLDAseyJsZXZlbCI6Miwic3R5bGUiOnsiaGVhZCI6eyJuYW1lIjoibm9uZSJ9fX1dLFs0LDIsIiIsMCx7ImNvbG91ciI6WzMwLDYwLDYwXX1dLFswLDcsIiIsMCx7ImxldmVsIjoyLCJzdHlsZSI6eyJoZWFkIjp7Im5hbWUiOiJub25lIn19fV0sWzcsMTEsIiIsMCx7InN0eWxlIjp7ImJvZHkiOnsibmFtZSI6ImRhc2hlZCJ9fX1dLFsxMSw1LCIiLDAseyJzdHlsZSI6eyJib2R5Ijp7Im5hbWUiOiJkYXNoZWQifX19XSxbNiwxLCIiLDAseyJjb2xvdXIiOlszMCw2MCw2MF19XSxbMSwxMSwiIiwwLHsiY29sb3VyIjpbMzAsNjAsNjBdfV0sWzExLDksIjAiLDAseyJjb2xvdXIiOlszMCw2MCw2MF0sInN0eWxlIjp7ImJvZHkiOnsibmFtZSI6ImRhc2hlZCJ9fX0sWzMwLDYwLDYwLDFdXSxbMiw1LCIiLDAseyJjb2xvdXIiOlszMCw2MCw2MF19XSxbNSw4LCIwIiwwLHsiY29sb3VyIjpbMzAsNjAsNjBdLCJzdHlsZSI6eyJib2R5Ijp7Im5hbWUiOiJkYXNoZWQifX19LFszMCw2MCw2MCwxXV0sWzUsMTAsIlxcdmFyZXBzaWxvbiAiLDAseyJzdHlsZSI6eyJib2R5Ijp7Im5hbWUiOiJkYXNoZWQifX19XV0=
\begin{tikzcd}[ampersand replacement=\&]
	\& P \& P \\
	{K'} \& {Q'} \& {M \oplus P} \& {\,} \\
	{K'} \& E \& M \& {\,} \\
	\& {\,} \& {\,}
	\arrow[equals, from=1-2, to=1-3]
	\arrow[color={rgb,255:red,214;green,153;blue,92}, from=1-2, to=2-2]
	\arrow[color={rgb,255:red,214;green,153;blue,92}, from=1-3, to=2-3]
	\arrow["{f'}", color={rgb,255:red,214;green,153;blue,92}, from=2-1, to=2-2]
	\arrow[equals, from=2-1, to=3-1]
	\arrow["{g'}", color={rgb,255:red,214;green,153;blue,92}, from=2-2, to=2-3]
	\arrow[color={rgb,255:red,214;green,153;blue,92}, from=2-2, to=3-2]
	\arrow["{\delta'}", color={rgb,255:red,214;green,153;blue,92}, dashed, from=2-3, to=2-4]
	\arrow[color={rgb,255:red,214;green,153;blue,92}, from=2-3, to=3-3]
	\arrow[dashed, from=3-1, to=3-2]
	\arrow[dashed, from=3-2, to=3-3]
	\arrow["0", color={rgb,255:red,214;green,153;blue,92}, dashed, from=3-2, to=4-2]
	\arrow["{\varepsilon }", dashed, from=3-3, to=3-4]
	\arrow["0", color={rgb,255:red,214;green,153;blue,92}, dashed, from=3-3, to=4-3]
\end{tikzcd}.
			\end{equation}
			此处 $E$ 是 $Q'$ 的直和项. 由底行得 $pd_\xi M \leq n$, 矛盾.
		\end{enumerate}
		最后证明 $\mathfrak M(\infty)$. 实际上, 以上第二步说明 $pd_\xi (M \oplus P) \geq pd_\xi M$, 从而得证.
	\end{proof}
\end{lemma}

\begin{remark}\label{rmk:pd_well_defined}
    投射维数的归纳定义无关 $\mathcal{P}(\xi)$-投射模的选取. 可使用若松技巧证明, 假定有 $\xi$-三角
    \begin{equation}
        K_1 \xrightarrow{f_1} P_1 \xrightarrow{g_1} X \overset {\delta_1} \dashrightarrow, \quad K_2 \xrightarrow{f_2} P_2 \xrightarrow{g_2} X \overset {\delta_2} \dashrightarrow,
    \end{equation}
    两者在 $X$ 处的拉回是两条可裂的 $\xi$-三角, 从而 $K_1 \oplus P_2 \cong K_2 \oplus P_1$. 依照 \cref{lem:pd_smd} 得 $pd_\xi K_1 = pd_\xi K_2$. 这一备注是为了说明 \cref{prop:projective_dimension_equiv}.
\end{remark}

\begin{definition}[完全 $\mathcal{P}(\xi)$-投射分解]
    给定双边无界复形 $X_\bullet = [\cdots X_1 \xrightarrow{d_1} X_0 \xrightarrow{d_0} X_{-1} \to \cdots]$.
    \begin{itemize}
        \item 称 $X_\bullet$ 是 $\xi$-正合的, 若存在分解 $d_n = g_{n-1}f_n$ 使得每个 $K_{n+1} \xrightarrow{g_n} X_n \xrightarrow{f_n} K_n \overset {\delta_n} \dashrightarrow$ 是 $\xi$-三角. 
        \item 称 $\mathbb E$-三角 $X \xrightarrow f Y \xrightarrow g Z \overset \delta \dashrightarrow$ 是 $(-, \mathcal{W})$ 正合的, 若对任意 $W \in \mathcal{W}$, 有短正合列
        \begin{equation}
            0 \to (Z,W) \xrightarrow {(g,W)} (Y,W) \xrightarrow {(f,W)} (X,W) \to 0.
        \end{equation}
        \item 称 $X_\bullet$ 是完全 $\mathcal{P}(\xi)$-投射分解, 若 $X_n \in \mathcal{P}(\xi)$, $X_\bullet$ 是 $\xi$-正合的, 且每一 $K_{n+1} \xrightarrow{g_n} X_n \xrightarrow{f_n} K_n \overset {\delta_n} \dashrightarrow$ 是 $(-, \mathcal{P}(\xi))$-正合的.
    \end{itemize}
    可以定义对偶地定义完全 $\xi$-内射分解.
\end{definition}

\begin{remark}
    在原文中, 默认 $(-, \mathcal{W})$-正合的 $\mathbb E$-三角 (复形) 是 $\xi$-三角 (复形).
\end{remark}

\begin{remark}
	给定完全 $\mathcal{P}(\xi)$-投射分解 $X_\bullet$, 态射 $d = gf$ 的选取未必唯一, 即 $\xi$-三角的拆解未必唯一. 然而, $(P, X_\bullet)$ 的满-单分解是唯一的, 即 $\xi$-三角在 $(\mathcal{P}(\xi), -)$ 下的的像唯一. 鉴于外三角范畴缺乏函子的满-单分解, $\mathcal{GP}(\xi)$ 更宜被定义为``能被中项为 $\mathcal{P}(\xi)$ 的 $(-, \mathcal{P}(\xi))$-正合 $\mathbb E$-三角双边无界延拓''.
\end{remark}

\begin{notation}
	方便起见, 称一个 $\mathbb E$-单射 ($\mathbb E$-满射, 或扩张元) 是 $(-,\mathcal{W})$ 正合的, 若其所在的 $\mathbb E$-三角是 $(-,\mathcal{W})$ 正合的.
\end{notation}

\begin{proposition}[$\mathcal{P}(\xi)$-投射维数的等价定义]\label{prop:projective_dimension_equiv}
	$pd_\xi M \leq n$, 当且仅当对任意 $\xi$-正合复形
	\begin{equation}
		K_n \xrightarrow{d_n} P_{n-1} \xrightarrow{d_{n-1}} \cdots \xrightarrow{d_1} P_0 \xrightarrow{d_0} M\quad (P_i \in \mathcal{P}(\xi)),
	\end{equation}
	总有 $K_n \in \mathcal{P}(\xi)$.
	\begin{proof}
		由定义归纳得 ($\gets$). 由 \cref{rmk:pd_well_defined} 得 ($\to$).
	\end{proof}
\end{proposition}

\begin{definition}[$\xi$-Gorenstein 投射]
    对于完全 $\mathcal{P}(\xi)$-投射分解 $P_\bullet$, 记 $K$-项为 $\xi$-Gorenstein 投射对象. 记作 $\mathcal{GP}(\xi)$. 可以对偶地定义 $\xi$-Gorenstein 内射对象, 记作 $\mathcal{GI}(\xi)$.
\end{definition}

\begin{lemma}\label{lem:W_exact}
    给定 $\mathbb E$-三角 $A \xrightarrow f B \xrightarrow g C \overset \delta \dashrightarrow$. 其 $(-, \mathcal{W})$-正合, 当且仅当 $(g,\mathcal{W})$ 与 $\mathbb E(g,\mathcal{W})$ 总是单的.
    \begin{proof}
        考虑五项长正合列
        \begin{equation}
            (C,W) \xrightarrow{(g,W)} (B,W) \xrightarrow{(f,W)} (A,W) \xrightarrow{\delta^\sharp} \mathbb E(C,W) \xrightarrow{g^\ast } \mathbb E(B,W).
        \end{equation}
        $\mathbb E$-三角是 $(-, \mathcal{W})$-正合的, 当且仅当 $(g,W)$ 单且 $(f,W)$ 满. 注意到 $(f,W)$ 满当且仅当 $g^\ast$ 单.
    \end{proof}
\end{lemma}

\begin{notation}
	往后将交换图中 $\mathcal{GP}(\xi)$ 中对象标注为紫色 ($\color{rgb,255:red,153;green,92;blue,214}\blacksquare$), 将 $(-, \mathcal{W})$ 正合 $\mathbb E$-三角标注为红色 ($\color{rgb,255:red,214;green,92;blue,92}\blacksquare$), 将待定的 $\mathbb E$-三角标注为蓝色 ($\color{rgb,255:red,92;green,92;blue,214}\blacksquare$). \cref{thm:proper_class_extriangulated} 已证明 $(\mathcal{C}, \mathbb E_\xi, \mathfrak s_\xi)$ 是外三角范畴, 故无需强调对 $\xi$-三角的橘色标记 ($\color{rgb,255:red,214;green,153;blue,92}\blacksquare$). 
\end{notation}

\begin{remark}
    $(-, \mathcal{W})$-正合 $\mathbb E$-三角的图表定理, 相当于满态射的图表定理.
\end{remark}

\begin{proposition}\label{prop:W_exact_0}
    若 $\varepsilon$ 是 $(-, \mathcal{W})$-正合的, 则任意 $\lambda_\ast \delta$ 亦然.
    \begin{equation}
        % https://q.uiver.app/#q=WzAsOCxbMiwxLCJcXGNkb3QiXSxbMSwxLCJcXGNkb3QiXSxbMCwxLCJcXGNkb3QiXSxbMSwwLCJcXGNkb3QiXSxbMiwwLCJcXGNkb3QiXSxbMCwwLCJcXGNkb3QiXSxbMywwLCJcXCwiXSxbMywxLCJcXCwiXSxbMSwwLCJmJyIsMCx7ImNvbG91ciI6WzI0MCw2MCw2MF0sInN0eWxlIjp7ImJvZHkiOnsibmFtZSI6ImRhc2hlZCJ9fX0sWzI0MCw2MCw2MCwxXV0sWzIsMSwiIiwwLHsiY29sb3VyIjpbMjQwLDYwLDYwXSwic3R5bGUiOnsiYm9keSI6eyJuYW1lIjoiZGFzaGVkIn19fV0sWzMsMSwiIiwwLHsic3R5bGUiOnsiYm9keSI6eyJuYW1lIjoiZGFzaGVkIn19fV0sWzQsMCwiIiwwLHsibGV2ZWwiOjIsInN0eWxlIjp7ImhlYWQiOnsibmFtZSI6Im5vbmUifX19XSxbMyw0LCJmIiwwLHsiY29sb3VyIjpbMCw2MCw2MF19LFswLDYwLDYwLDFdXSxbNSwzLCIiLDAseyJjb2xvdXIiOlswLDYwLDYwXX1dLFs1LDIsIlxcbGFtYmRhIiwwLHsic3R5bGUiOnsiYm9keSI6eyJuYW1lIjoiZGFzaGVkIn19fV0sWzQsNiwiXFx2YXJlcHNpbG9uICIsMCx7ImNvbG91ciI6WzAsNjAsNjBdLCJzdHlsZSI6eyJib2R5Ijp7Im5hbWUiOiJkYXNoZWQifX19LFswLDYwLDYwLDFdXSxbMCw3LCJcXGxhbWJkYV9cXGFzdCBcXHZhcmVwc2lsb24gIiwwLHsiY29sb3VyIjpbMjQwLDYwLDYwXSwic3R5bGUiOnsiYm9keSI6eyJuYW1lIjoiZGFzaGVkIn19fSxbMjQwLDYwLDYwLDFdXV0=
\begin{tikzcd}[ampersand replacement=\&]
	\cdot \& \cdot \& \cdot \& {\,} \\
	\cdot \& \cdot \& \cdot \& {\,}
	\arrow[color={rgb,255:red,214;green,92;blue,92}, from=1-1, to=1-2]
	\arrow["\lambda", dashed, from=1-1, to=2-1]
	\arrow["f", color={rgb,255:red,214;green,92;blue,92}, from=1-2, to=1-3]
	\arrow[dashed, from=1-2, to=2-2]
	\arrow["{\varepsilon }", color={rgb,255:red,214;green,92;blue,92}, dashed, from=1-3, to=1-4]
	\arrow[equals, from=1-3, to=2-3]
	\arrow[color={rgb,255:red,92;green,92;blue,214}, dashed, from=2-1, to=2-2]
	\arrow["{f'}", color={rgb,255:red,92;green,92;blue,214}, dashed, from=2-2, to=2-3]
	\arrow["{\lambda_\ast \varepsilon }", color={rgb,255:red,92;green,92;blue,214}, dashed, from=2-3, to=2-4]
\end{tikzcd}.
    \end{equation}
    \begin{proof}
        $\lambda_\ast \delta$ 是 $\xi$-扩张元. 由 \cref{lem:W_exact}, 只需证 $(f',\mathcal{W})$ 与 $\mathbb E(f',\mathcal{W})$ 单. 由 $(f,\mathcal{W})$ 与 $\mathbb E(f,\mathcal{W})$ 单, 证毕.
    \end{proof}
\end{proposition}

\begin{proposition}\label{prop:W_exact_1}
    若 $\delta$ 与 $\varepsilon$ 是 $(-, \mathcal{W})$-正合的, 则 $\varepsilon'$ 亦然.
    \begin{equation}
% https://q.uiver.app/#q=WzAsMTIsWzIsMiwiXFxjZG90Il0sWzIsMSwiXFxjZG90Il0sWzIsMCwiXFxjZG90Il0sWzEsMSwiXFxjZG90Il0sWzEsMiwiXFxjZG90Il0sWzAsMSwiXFxjZG90Il0sWzMsMSwiXFwsIl0sWzEsMCwiXFxjZG90Il0sWzEsMywiXFwsIl0sWzAsMCwiXFxjZG90Il0sWzMsMCwiXFwsIl0sWzIsMywiXFwsIl0sWzEsMCwiZiIsMCx7ImNvbG91ciI6WzAsNjAsNjBdfSxbMCw2MCw2MCwxXV0sWzIsMSwiIiwwLHsiY29sb3VyIjpbMCw2MCw2MF19XSxbMywxLCJnIiwwLHsiY29sb3VyIjpbMCw2MCw2MF19LFswLDYwLDYwLDFdXSxbNCwwLCIiLDAseyJsZXZlbCI6Miwic3R5bGUiOnsiaGVhZCI6eyJuYW1lIjoibm9uZSJ9fX1dLFszLDQsImdmIiwwLHsiY29sb3VyIjpbMjQwLDYwLDYwXSwic3R5bGUiOnsiYm9keSI6eyJuYW1lIjoiZGFzaGVkIn19fSxbMjQwLDYwLDYwLDFdXSxbNSwzLCIiLDAseyJjb2xvdXIiOlswLDYwLDYwXX1dLFs3LDMsIiIsMCx7ImNvbG91ciI6WzI0MCw2MCw2MF0sInN0eWxlIjp7ImJvZHkiOnsibmFtZSI6ImRhc2hlZCJ9fX1dLFs3LDIsIiIsMCx7InN0eWxlIjp7ImJvZHkiOnsibmFtZSI6ImRhc2hlZCJ9fX1dLFs5LDcsIiIsMCx7InN0eWxlIjp7ImJvZHkiOnsibmFtZSI6ImRhc2hlZCJ9fX1dLFs5LDUsIiIsMCx7ImxldmVsIjoyLCJzdHlsZSI6eyJoZWFkIjp7Im5hbWUiOiJub25lIn19fV0sWzIsMTAsIlxcZGVsdGEnIiwwLHsic3R5bGUiOnsiYm9keSI6eyJuYW1lIjoiZGFzaGVkIn19fV0sWzEsNiwiXFxkZWx0YSIsMCx7ImNvbG91ciI6WzAsNjAsNjBdLCJzdHlsZSI6eyJib2R5Ijp7Im5hbWUiOiJkYXNoZWQifX19LFswLDYwLDYwLDFdXSxbNCw4LCJcXHZhcmVwc2lsb24gJyIsMCx7ImNvbG91ciI6WzI0MCw2MCw2MF0sInN0eWxlIjp7ImJvZHkiOnsibmFtZSI6ImRhc2hlZCJ9fX0sWzI0MCw2MCw2MCwxXV0sWzAsMTEsIlxcdmFyZXBzaWxvbiAiLDAseyJjb2xvdXIiOlswLDYwLDYwXSwic3R5bGUiOnsiYm9keSI6eyJuYW1lIjoiZGFzaGVkIn19fSxbMCw2MCw2MCwxXV1d
\begin{tikzcd}[ampersand replacement=\&]
	\cdot \& \cdot \& \cdot \& {\,} \\
	\cdot \& \cdot \& \cdot \& {\,} \\
	\& \cdot \& \cdot \\
	\& {\,} \& {\,}
	\arrow[dashed, from=1-1, to=1-2]
	\arrow[equals, from=1-1, to=2-1]
	\arrow[dashed, from=1-2, to=1-3]
	\arrow[color={rgb,255:red,92;green,92;blue,214}, dashed, from=1-2, to=2-2]
	\arrow["{\delta'}", dashed, from=1-3, to=1-4]
	\arrow[color={rgb,255:red,214;green,92;blue,92}, from=1-3, to=2-3]
	\arrow[color={rgb,255:red,214;green,92;blue,92}, from=2-1, to=2-2]
	\arrow["g", color={rgb,255:red,214;green,92;blue,92}, from=2-2, to=2-3]
	\arrow["gf", color={rgb,255:red,92;green,92;blue,214}, dashed, from=2-2, to=3-2]
	\arrow["\delta", color={rgb,255:red,214;green,92;blue,92}, dashed, from=2-3, to=2-4]
	\arrow["f", color={rgb,255:red,214;green,92;blue,92}, from=2-3, to=3-3]
	\arrow[equals, from=3-2, to=3-3]
	\arrow["{\varepsilon '}", color={rgb,255:red,92;green,92;blue,214}, dashed, from=3-2, to=4-2]
	\arrow["{\varepsilon }", color={rgb,255:red,214;green,92;blue,92}, dashed, from=3-3, to=4-3]
\end{tikzcd}.
    \end{equation}
    \begin{proof}
        由 \cref{thm:proper_class_extriangulated}, 上图是 $(\mathcal{C}, \mathbb E_\xi, \mathfrak s_\xi)$ 中的交换图. 由 \cref{lem:W_exact}, 只需证明 $(gf, \mathcal{W})$ 与 $\mathbb E(gf, \mathcal{W})$ 单. 由假设, $(g, \mathcal{W})$, $\mathbb E(g, \mathcal{W})$, $(f, \mathcal{W})$ 与 $\mathbb E(f, \mathcal{W})$ 单. 因此 $(gf, \mathcal{W})$ 与 $\mathbb E(gf, \mathcal{W})$ 单.
    \end{proof}
\end{proposition}

\begin{proposition}\label{prop:W_exact_2}
    若 $\delta'$ 与 $\varepsilon$ 是 $(-, \mathcal{W})$-正合的, 则 $\delta$ 亦然.
    \begin{equation}
% https://q.uiver.app/#q=WzAsMTIsWzAsMiwiXFxjZG90Il0sWzEsMiwiXFxjZG90Il0sWzIsMiwiXFxjZG90Il0sWzMsMiwiXFwsICJdLFsyLDAsIlxcY2RvdCJdLFsyLDEsIlxcY2RvdCJdLFsyLDMsIlxcLCAiXSxbMCwxLCJcXGNkb3QiXSxbMSwxLCJcXGNkb3QiXSxbMywxLCJcXCwgIl0sWzEsMCwiXFxjZG90Il0sWzEsMywiXFwsICJdLFswLDEsIiIsMCx7ImNvbG91ciI6WzI0MCw2MCw2MF19XSxbMSwyLCJmJyIsMCx7ImNvbG91ciI6WzI0MCw2MCw2MF19LFsyNDAsNjAsNjAsMV1dLFsyLDMsIlxcZGVsdGEiLDAseyJjb2xvdXIiOlsyNDAsNjAsNjBdLCJzdHlsZSI6eyJib2R5Ijp7Im5hbWUiOiJkYXNoZWQifX19LFsyNDAsNjAsNjAsMV1dLFs0LDUsIiIsMCx7ImNvbG91ciI6WzAsNjAsNjBdfV0sWzUsMiwiZyIsMCx7ImNvbG91ciI6WzAsNjAsNjBdfSxbMCw2MCw2MCwxXV0sWzIsNiwiXFx2YXJlcHNpbG9uICIsMCx7ImNvbG91ciI6WzAsNjAsNjBdLCJzdHlsZSI6eyJib2R5Ijp7Im5hbWUiOiJkYXNoZWQifX19LFswLDYwLDYwLDFdXSxbNywwLCIiLDAseyJsZXZlbCI6Miwic3R5bGUiOnsiaGVhZCI6eyJuYW1lIjoibm9uZSJ9fX1dLFs3LDgsIiIsMCx7ImNvbG91ciI6WzAsNjAsNjBdLCJzdHlsZSI6eyJib2R5Ijp7Im5hbWUiOiJkYXNoZWQifX19XSxbOCw1LCJmIiwwLHsiY29sb3VyIjpbMCw2MCw2MF0sInN0eWxlIjp7ImJvZHkiOnsibmFtZSI6ImRhc2hlZCJ9fX0sWzAsNjAsNjAsMV1dLFs1LDksIlxcZGVsdGEgJyIsMCx7ImNvbG91ciI6WzAsNjAsNjBdLCJzdHlsZSI6eyJib2R5Ijp7Im5hbWUiOiJkYXNoZWQifX19LFswLDYwLDYwLDFdXSxbMTAsNCwiIiwwLHsibGV2ZWwiOjIsInN0eWxlIjp7ImhlYWQiOnsibmFtZSI6Im5vbmUifX19XSxbMTAsOCwiIiwwLHsic3R5bGUiOnsiYm9keSI6eyJuYW1lIjoiZGFzaGVkIn19fV0sWzgsMSwiIiwwLHsic3R5bGUiOnsiYm9keSI6eyJuYW1lIjoiZGFzaGVkIn19fV0sWzEsMTEsIlxcdmFyZXBzaWxvbiAnIiwwLHsic3R5bGUiOnsiYm9keSI6eyJuYW1lIjoiZGFzaGVkIn19fV1d
\begin{tikzcd}[ampersand replacement=\&]
	\& \cdot \& \cdot \\
	\cdot \& \cdot \& \cdot \& {\, } \\
	\cdot \& \cdot \& \cdot \& {\, } \\
	\& {\, } \& {\, }
	\arrow[equals, from=1-2, to=1-3]
	\arrow[dashed, from=1-2, to=2-2]
	\arrow[draw={rgb,255:red,214;green,92;blue,92}, from=1-3, to=2-3]
	\arrow[draw={rgb,255:red,214;green,92;blue,92}, dashed, from=2-1, to=2-2]
	\arrow[equals, from=2-1, to=3-1]
	\arrow["f", color={rgb,255:red,214;green,92;blue,92}, dashed, from=2-2, to=2-3]
	\arrow[dashed, from=2-2, to=3-2]
	\arrow["{\delta '}", color={rgb,255:red,214;green,92;blue,92}, dashed, from=2-3, to=2-4]
	\arrow["g", color={rgb,255:red,214;green,92;blue,92}, from=2-3, to=3-3]
	\arrow[draw={rgb,255:red,92;green,92;blue,214}, from=3-1, to=3-2]
	\arrow["{f'}", color={rgb,255:red,92;green,92;blue,214}, from=3-2, to=3-3]
	\arrow["{\varepsilon '}", dashed, from=3-2, to=4-2]
	\arrow["\delta", color={rgb,255:red,92;green,92;blue,214}, dashed, from=3-3, to=3-4]
	\arrow["{\varepsilon }", color={rgb,255:red,214;green,92;blue,92}, dashed, from=3-3, to=4-3]
\end{tikzcd}.
    \end{equation}
\begin{proof}
    (细节从略). 所有三角都是 $\xi$-三角. 若 $gf$ ``满'', 则 $f'$ ``满''.
\end{proof}
\end{proposition}

\begin{theorem}\label{thm:GP_W_exact}
    给定 $\xi$-三角 $A \xrightarrow f B \xrightarrow g C \overset \delta \dashrightarrow$. 若 $C \in \mathcal{GP}(\xi)$, 则该三角是 $(-, \mathcal{P}(\xi))$-正合的.
    \begin{proof}
        由 $C$ 的定义作
        \begin{equation}
% https://q.uiver.app/% https://q.uiver.app/#q=WzAsMTIsWzAsMiwiQSJdLFsxLDIsIkIiXSxbMiwyLCJDIixbMjcwLDYwLDYwLDFdXSxbMywyLCJcXCwgIl0sWzIsMCwiSyIsWzI3MCw2MCw2MCwxXV0sWzIsMSwiUCIsWzI3MCw2MCw2MCwxXV0sWzIsMywiXFwsICJdLFswLDEsIkEiXSxbMSwxLCJBIFxcb3BsdXMgUCJdLFszLDEsIlxcLCAiXSxbMSwwLCJLIixbMjcwLDYwLDYwLDFdXSxbMSwzLCJcXCwgIl0sWzEsMiwiZyIsMCx7ImNvbG91ciI6WzI0MCw2MCw2MF19LFsyNDAsNjAsNjAsMV1dLFsyLDMsIlxcZGVsdGEiLDAseyJjb2xvdXIiOlsyNDAsNjAsNjBdLCJzdHlsZSI6eyJib2R5Ijp7Im5hbWUiOiJkYXNoZWQifX19LFsyNDAsNjAsNjAsMV1dLFs1LDIsInAiLDAseyJjb2xvdXIiOlswLDYwLDYwXX0sWzAsNjAsNjAsMV1dLFsyLDYsIlxcdmFyZXBzaWxvbiAiLDAseyJjb2xvdXIiOlswLDYwLDYwXSwic3R5bGUiOnsiYm9keSI6eyJuYW1lIjoiZGFzaGVkIn19fSxbMCw2MCw2MCwxXV0sWzcsMCwiIiwwLHsibGV2ZWwiOjIsInN0eWxlIjp7ImhlYWQiOnsibmFtZSI6Im5vbmUifX19XSxbNyw4LCIiLDAseyJjb2xvdXIiOlswLDYwLDYwXSwic3R5bGUiOnsiYm9keSI6eyJuYW1lIjoiZGFzaGVkIn19fV0sWzgsNSwiIiwwLHsiY29sb3VyIjpbMCw2MCw2MF0sInN0eWxlIjp7ImJvZHkiOnsibmFtZSI6ImRhc2hlZCJ9fX1dLFs1LDksIjAiLDAseyJjb2xvdXIiOlswLDYwLDYwXSwic3R5bGUiOnsiYm9keSI6eyJuYW1lIjoiZGFzaGVkIn19fSxbMCw2MCw2MCwxXV0sWzEwLDQsIiIsMCx7ImxldmVsIjoyLCJzdHlsZSI6eyJoZWFkIjp7Im5hbWUiOiJub25lIn19fV0sWzEwLDgsIiIsMCx7InN0eWxlIjp7ImJvZHkiOnsibmFtZSI6ImRhc2hlZCJ9fX1dLFs4LDEsIiIsMCx7InN0eWxlIjp7ImJvZHkiOnsibmFtZSI6ImRhc2hlZCJ9fX1dLFsxLDExLCJcXHZhcmVwc2lsb24gJyIsMCx7InN0eWxlIjp7ImJvZHkiOnsibmFtZSI6ImRhc2hlZCJ9fX1dLFswLDEsImYiLDAseyJjb2xvdXIiOlsyNDAsNjAsNjBdfSxbMjQwLDYwLDYwLDFdXSxbNCw1LCJpIiwwLHsiY29sb3VyIjpbMCw2MCw2MF19LFswLDYwLDYwLDFdXV0=
\begin{tikzcd}[ampersand replacement=\&]
	\& \textcolor{rgb,255:red,153;green,92;blue,214}{K} \& \textcolor{rgb,255:red,153;green,92;blue,214}{K} \\
	A \& {A \oplus P} \& \textcolor{rgb,255:red,153;green,92;blue,214}{P} \& {\, } \\
	A \& B \& \textcolor{rgb,255:red,153;green,92;blue,214}{C} \& {\, } \\
	\& {\, } \& {\, }
	\arrow[equals, from=1-2, to=1-3]
	\arrow[dashed, from=1-2, to=2-2]
	\arrow["i", color={rgb,255:red,214;green,92;blue,92}, from=1-3, to=2-3]
	\arrow[draw={rgb,255:red,214;green,92;blue,92}, dashed, from=2-1, to=2-2]
	\arrow[equals, from=2-1, to=3-1]
	\arrow[draw={rgb,255:red,214;green,92;blue,92}, dashed, from=2-2, to=2-3]
	\arrow[dashed, from=2-2, to=3-2]
	\arrow["0", color={rgb,255:red,214;green,92;blue,92}, dashed, from=2-3, to=2-4]
	\arrow["p", color={rgb,255:red,214;green,92;blue,92}, from=2-3, to=3-3]
	\arrow["f", color={rgb,255:red,92;green,92;blue,214}, from=3-1, to=3-2]
	\arrow["g", color={rgb,255:red,92;green,92;blue,214}, from=3-2, to=3-3]
	\arrow["{\varepsilon '}", dashed, from=3-2, to=4-2]
	\arrow["\delta", color={rgb,255:red,92;green,92;blue,214}, dashed, from=3-3, to=3-4]
	\arrow["{\varepsilon }", color={rgb,255:red,214;green,92;blue,92}, dashed, from=3-3, to=4-3]
\end{tikzcd}.
        \end{equation}
        依照 \cref{prop:W_exact_1}, $\delta$ $(-, \mathcal{P}(\xi))$-正合的.
    \end{proof}
\end{theorem}

\begin{theorem}\label{thm:W_exact_3}
	假定存在六个 $\mathbb E$-三角使得下图四个方块交换 (不要求扩张元间的态射):
	\begin{equation}
		% https://q.uiver.app/#q=WzAsMTUsWzAsMCwiXFxjZG90Il0sWzEsMCwiXFxjZG90Il0sWzIsMCwiXFxjZG90Il0sWzAsMSwiXFxjZG90Il0sWzEsMSwiXFxjZG90Il0sWzIsMSwiXFxjZG90Il0sWzAsMiwiXFxjZG90Il0sWzEsMiwiXFxjZG90Il0sWzIsMiwiXFxjZG90Il0sWzMsMCwiXFwsIl0sWzMsMSwiXFwsIl0sWzMsMiwiXFwsIl0sWzIsMywiXFwsIl0sWzEsMywiXFwsIl0sWzAsMywiXFwsIl0sWzAsMV0sWzMsNF0sWzQsNV0sWzAsM10sWzMsNl0sWzYsN10sWzcsOF0sWzEsNF0sWzQsN10sWzEsMl0sWzIsNV0sWzUsOF0sWzIsOSwiIiwwLHsic3R5bGUiOnsiYm9keSI6eyJuYW1lIjoiZGFzaGVkIn19fV0sWzUsMTAsIiIsMCx7InN0eWxlIjp7ImJvZHkiOnsibmFtZSI6ImRhc2hlZCJ9fX1dLFs4LDExLCIiLDEseyJzdHlsZSI6eyJib2R5Ijp7Im5hbWUiOiJkYXNoZWQifX19XSxbOCwxMiwiIiwxLHsic3R5bGUiOnsiYm9keSI6eyJuYW1lIjoiZGFzaGVkIn19fV0sWzcsMTMsIiIsMSx7InN0eWxlIjp7ImJvZHkiOnsibmFtZSI6ImRhc2hlZCJ9fX1dLFs2LDE0LCIiLDEseyJzdHlsZSI6eyJib2R5Ijp7Im5hbWUiOiJkYXNoZWQifX19XV0=
\begin{tikzcd}[ampersand replacement=\&]
	\cdot \& \cdot \& \cdot \& {\,} \\
	\cdot \& \cdot \& \cdot \& {\,} \\
	\cdot \& \cdot \& \cdot \& {\,} \\
	{\,} \& {\,} \& {\,}
	\arrow[from=1-1, to=1-2]
	\arrow[from=1-1, to=2-1]
	\arrow[from=1-2, to=1-3]
	\arrow[from=1-2, to=2-2]
	\arrow[dashed, from=1-3, to=1-4]
	\arrow[from=1-3, to=2-3]
	\arrow[from=2-1, to=2-2]
	\arrow[from=2-1, to=3-1]
	\arrow[from=2-2, to=2-3]
	\arrow[from=2-2, to=3-2]
	\arrow[dashed, from=2-3, to=2-4]
	\arrow[from=2-3, to=3-3]
	\arrow[from=3-1, to=3-2]
	\arrow[dashed, from=3-1, to=4-1]
	\arrow[from=3-2, to=3-3]
	\arrow[dashed, from=3-2, to=4-2]
	\arrow[dashed, from=3-3, to=3-4]
	\arrow[dashed, from=3-3, to=4-3]
\end{tikzcd}.
	\end{equation}
	若任意五个 $\mathbb E$-三角是 $(-, \mathcal{W})$ 正合的, 则第六个三角亦然.
	\begin{proof}
		对上图作用 $(-, \mathcal{W})$, 得五个短正合列与一个中间项正合的态射序列. 依照 Abel 范畴的 $3 \times 3$-引理求解即可.
	\end{proof}
\end{theorem}

\subsection{\texorpdfstring{$\mathcal{GP}(\xi)$}{} 的结构}

本节证明 $\mathcal{GP}(\xi)$ 是消解的加法子范畴, 其对直和项封闭.

\begin{lemma}\label{lem:GP_P_sum}
	$\mathcal{GP}(\xi)$ 中对象对 $\mathcal{P}(\xi)$ 中直和项的消去封闭. 即, 若 $P \in \mathcal{P}(\xi)$, $P \oplus G \in \mathcal{GP}(\xi)$, 则 $G \in \mathcal{GP}(\xi)$.
	\begin{proof}
		取 $P \oplus G$ 的完全 $\mathcal{P}(\xi)$-投射分解
		\begin{equation}
			% https://q.uiver.app/#q=WzAsNyxbMywxLCJQIFxcb3BsdXMgRyJdLFsyLDAsIlFfMCJdLFs0LDAsIlFfMSJdLFsxLDEsIktfMCJdLFs1LDEsIktfMiJdLFswLDAsIlxcY2RvdHMiXSxbNiwwLCJcXGNkb3RzIl0sWzEsMF0sWzAsMl0sWzIsNF0sWzUsM10sWzQsNl0sWzUsMV0sWzEsMl0sWzIsNl0sWzMsMV1d
\begin{tikzcd}[ampersand replacement=\&]
	\cdots \&\& {Q_0} \&\& {Q_1} \&\& \cdots \\
	\& {K_0} \&\& {P \oplus G} \&\& {K_2}
	\arrow[from=1-1, to=1-3]
	\arrow[from=1-1, to=2-2]
	\arrow[from=1-3, to=1-5]
	\arrow[from=1-3, to=2-4]
	\arrow[from=1-5, to=1-7]
	\arrow[from=1-5, to=2-6]
	\arrow[from=2-2, to=1-3]
	\arrow[from=2-4, to=1-5]
	\arrow[from=2-6, to=1-7]
\end{tikzcd}.
		\end{equation}
		下证明 $G$ 嵌入某一完全 $\mathcal{P}(\xi)$-投射分解.
		\begin{enumerate}
			\item (向左延伸). 作以下 $\xi$-三角的交换图
			\begin{equation}
% https://q.uiver.app/#q=WzAsMTIsWzIsMSwiUFxcb3BsdXMgRyIsWzI3MCw2MCw2MCwxXV0sWzEsMSwiUV8wIl0sWzAsMSwiS18wIl0sWzMsMSwiXFwsIl0sWzIsMiwiUCIsWzI3MCw2MCw2MCwxXV0sWzEsMiwiUCIsWzI3MCw2MCw2MCwxXV0sWzIsMCwiRyIsWzI3MCw2MCw2MCwxXV0sWzMsMCwiXFwsIl0sWzIsMywiXFwsIl0sWzEsMywiXFwsIl0sWzAsMCwiS18wIl0sWzEsMCwiUCciXSxbMiwxLCIiLDAseyJjb2xvdXIiOlswLDYwLDYwXX1dLFsxLDAsIiIsMCx7ImNvbG91ciI6WzAsNjAsNjBdfV0sWzAsMywiIiwwLHsiY29sb3VyIjpbMCw2MCw2MF0sInN0eWxlIjp7ImJvZHkiOnsibmFtZSI6ImRhc2hlZCJ9fX1dLFsxMCwyLCIiLDAseyJsZXZlbCI6Miwic3R5bGUiOnsiaGVhZCI6eyJuYW1lIjoibm9uZSJ9fX1dLFs1LDQsIiIsMCx7ImxldmVsIjoyLCJzdHlsZSI6eyJoZWFkIjp7Im5hbWUiOiJub25lIn19fV0sWzEwLDExLCIiLDAseyJjb2xvdXIiOlsyNDAsNjAsNjBdLCJzdHlsZSI6eyJib2R5Ijp7Im5hbWUiOiJkYXNoZWQifX19XSxbMTEsNiwiIiwwLHsiY29sb3VyIjpbMjQwLDYwLDYwXSwic3R5bGUiOnsiYm9keSI6eyJuYW1lIjoiZGFzaGVkIn19fV0sWzYsNywiIiwwLHsiY29sb3VyIjpbMjQwLDYwLDYwXSwic3R5bGUiOnsiYm9keSI6eyJuYW1lIjoiZGFzaGVkIn19fV0sWzExLDEsIiIsMCx7ImNvbG91ciI6WzAsNjAsNjBdfV0sWzEsNSwiIiwwLHsiY29sb3VyIjpbMCw2MCw2MF19XSxbNSw5LCIiLDAseyJjb2xvdXIiOlswLDYwLDYwXSwic3R5bGUiOnsiYm9keSI6eyJuYW1lIjoiZGFzaGVkIn19fV0sWzYsMCwiIiwwLHsiY29sb3VyIjpbMCw2MCw2MF19XSxbMCw0LCIiLDAseyJjb2xvdXIiOlswLDYwLDYwXX1dLFs0LDgsIjAiLDAseyJjb2xvdXIiOlswLDYwLDYwXSwic3R5bGUiOnsiYm9keSI6eyJuYW1lIjoiZGFzaGVkIn19fSxbMCw2MCw2MCwxXV1d
\begin{tikzcd}[ampersand replacement=\&]
	{K_0} \& {P'} \& \textcolor{rgb,255:red,153;green,92;blue,214}{G} \& {\,} \\
	{K_0} \& {Q_0} \& \textcolor{rgb,255:red,153;green,92;blue,214}{{P\oplus G}} \& {\,} \\
	\& \textcolor{rgb,255:red,153;green,92;blue,214}{P} \& \textcolor{rgb,255:red,153;green,92;blue,214}{P} \\
	\& {\,} \& {\,}
	\arrow[color={rgb,255:red,92;green,92;blue,214}, dashed, from=1-1, to=1-2]
	\arrow[equals, from=1-1, to=2-1]
	\arrow[color={rgb,255:red,92;green,92;blue,214}, dashed, from=1-2, to=1-3]
	\arrow[draw={rgb,255:red,214;green,92;blue,92}, from=1-2, to=2-2]
	\arrow[color={rgb,255:red,92;green,92;blue,214}, dashed, from=1-3, to=1-4]
	\arrow[draw={rgb,255:red,214;green,92;blue,92}, from=1-3, to=2-3]
	\arrow[draw={rgb,255:red,214;green,92;blue,92}, from=2-1, to=2-2]
	\arrow[draw={rgb,255:red,214;green,92;blue,92}, from=2-2, to=2-3]
	\arrow[draw={rgb,255:red,214;green,92;blue,92}, from=2-2, to=3-2]
	\arrow[draw={rgb,255:red,214;green,92;blue,92}, dashed, from=2-3, to=2-4]
	\arrow[draw={rgb,255:red,214;green,92;blue,92}, from=2-3, to=3-3]
	\arrow[equals, from=3-2, to=3-3]
	\arrow[draw={rgb,255:red,214;green,92;blue,92}, dashed, from=3-2, to=4-2]
	\arrow["0", color={rgb,255:red,214;green,92;blue,92}, dashed, from=3-3, to=4-3]
\end{tikzcd}.
			\end{equation}
			由 \cref{thm:W_exact_3}, 第一行是 $(-,\mathcal{P}(\xi))$-正合的.
			\item (向右延伸). 作以下 $\xi$-三角的交换图, 这两个 $\mathbb E$-三角也是 $(-,\mathcal{P}(\xi))$-正合的:
			\begin{equation}
				% https://q.uiver.app/#q=WzAsOCxbMCwwLCJQIFxcb3BsdXMgRyJdLFsxLDAsIlFfMSJdLFsyLDAsIktfMiIsWzI3MCw2MCw2MCwxXV0sWzMsMCwiXFwsIl0sWzAsMSwiUCJdLFsxLDEsIlAgXFxvcGx1cyBLXzIiXSxbMiwxLCJLXzIiLFsyNzAsNjAsNjAsMV1dLFszLDEsIlxcLCJdLFswLDEsIihmXzEsZl8yKSIsMCx7ImNvbG91ciI6WzAsNjAsNjBdfSxbMCw2MCw2MCwxXV0sWzEsMiwiZyIsMCx7ImNvbG91ciI6WzAsNjAsNjBdfSxbMCw2MCw2MCwxXV0sWzIsMywiIiwwLHsiY29sb3VyIjpbMCw2MCw2MF0sInN0eWxlIjp7ImJvZHkiOnsibmFtZSI6ImRhc2hlZCJ9fX1dLFswLDQsIigxLDApIiwyXSxbMiw2LCIiLDAseyJsZXZlbCI6Miwic3R5bGUiOnsiaGVhZCI6eyJuYW1lIjoibm9uZSJ9fX1dLFs0LDUsIlxcYmlub20gMTAiLDAseyJjb2xvdXIiOlswLDYwLDYwXX0sWzAsNjAsNjAsMV1dLFs1LDYsIigwLDEpIiwwLHsiY29sb3VyIjpbMCw2MCw2MF19LFswLDYwLDYwLDFdXSxbNiw3LCIwIiwwLHsiY29sb3VyIjpbMCw2MCw2MF0sInN0eWxlIjp7ImJvZHkiOnsibmFtZSI6ImRhc2hlZCJ9fX0sWzAsNjAsNjAsMV1dLFsxLDUsIlxcYmlub20gbG0iLDAseyJzdHlsZSI6eyJib2R5Ijp7Im5hbWUiOiJkYXNoZWQifX19XSxbMCw1LCJcXHNxdWFyZSIsMSx7InN0eWxlIjp7ImJvZHkiOnsibmFtZSI6Im5vbmUifSwiaGVhZCI6eyJuYW1lIjoibm9uZSJ9fX1dXQ==
\begin{tikzcd}[ampersand replacement=\&]
	{P \oplus G} \& {Q_1} \& \textcolor{rgb,255:red,153;green,92;blue,214}{{K_2}} \& {\,} \\
	P \& {P \oplus K_2} \& \textcolor{rgb,255:red,153;green,92;blue,214}{{K_2}} \& {\,}
	\arrow["{(f_1,f_2)}", color={rgb,255:red,214;green,92;blue,92}, from=1-1, to=1-2]
	\arrow["{(1,0)}"', from=1-1, to=2-1]
	\arrow["\square"{description}, draw=none, from=1-1, to=2-2]
	\arrow["g", color={rgb,255:red,214;green,92;blue,92}, from=1-2, to=1-3]
	\arrow["{\binom lm}", dashed, from=1-2, to=2-2]
	\arrow[color={rgb,255:red,214;green,92;blue,92}, dashed, from=1-3, to=1-4]
	\arrow[equals, from=1-3, to=2-3]
	\arrow["{\binom 10}", color={rgb,255:red,214;green,92;blue,92}, from=2-1, to=2-2]
	\arrow["{(0,1)}", color={rgb,255:red,214;green,92;blue,92}, from=2-2, to=2-3]
	\arrow["0", color={rgb,255:red,214;green,92;blue,92}, dashed, from=2-3, to=2-4]
\end{tikzcd}.
			\end{equation}
			此时有 $(-,\mathcal{P}(\xi))$-正合三角的交换图, 
			\begin{equation}
% https://q.uiver.app/#q=WzAsMTIsWzAsMSwiUCBcXG9wbHVzIEciXSxbMSwxLCJQIFxcb3BsdXMgUV8xIl0sWzIsMSwiUCBcXG9wbHVzIEtfMiIsWzI3MCw2MCw2MCwxXV0sWzMsMSwiXFwsIl0sWzAsMiwiUCIsWzI3MCw2MCw2MCwxXV0sWzEsMiwiUCIsWzI3MCw2MCw2MCwxXV0sWzEsMCwiUV8xIl0sWzAsMCwiRyJdLFsyLDAsIlAgXFxvcGx1cyBLXzIiLFsyNzAsNjAsNjAsMV1dLFszLDAsIlxcLCJdLFswLDMsIlxcLCJdLFsxLDMsIlxcLCJdLFswLDEsIlxcYmlub20gezEgXFwgXFwgMH17Zl8xIFxcIGZfMn0iLDAseyJjb2xvdXIiOlswLDYwLDYwXX0sWzAsNjAsNjAsMV1dLFsxLDIsIlxcYmlub20gezEgXFwgXFwgbFxcIH17MCBcXCBtfSIsMCx7ImNvbG91ciI6WzAsNjAsNjBdfSxbMCw2MCw2MCwxXV0sWzIsMywiIiwwLHsiY29sb3VyIjpbMCw2MCw2MF0sInN0eWxlIjp7ImJvZHkiOnsibmFtZSI6ImRhc2hlZCJ9fX1dLFs0LDUsIiIsMCx7ImxldmVsIjoyLCJzdHlsZSI6eyJoZWFkIjp7Im5hbWUiOiJub25lIn19fV0sWzcsMCwiXFxiaW5vbSAwMSIsMCx7ImNvbG91ciI6WzAsNjAsNjBdfSxbMCw2MCw2MCwxXV0sWzAsNCwiKDEsMCkiLDAseyJjb2xvdXIiOlswLDYwLDYwXX0sWzAsNjAsNjAsMV1dLFs2LDEsIlxcYmlub20gMDEiLDAseyJjb2xvdXIiOlswLDYwLDYwXX0sWzAsNjAsNjAsMV1dLFsxLDUsIigxLDApIiwwLHsiY29sb3VyIjpbMCw2MCw2MF19LFswLDYwLDYwLDFdXSxbNyw2LCIiLDEseyJjb2xvdXIiOlsyNDAsNjAsNjBdLCJzdHlsZSI6eyJib2R5Ijp7Im5hbWUiOiJkYXNoZWQifX19XSxbNiw4LCIiLDEseyJjb2xvdXIiOlsyNDAsNjAsNjBdLCJzdHlsZSI6eyJib2R5Ijp7Im5hbWUiOiJkYXNoZWQifX19XSxbOCw5LCIiLDEseyJjb2xvdXIiOlsyNDAsNjAsNjBdLCJzdHlsZSI6eyJib2R5Ijp7Im5hbWUiOiJkYXNoZWQifX19XSxbOCwyLCIiLDAseyJsZXZlbCI6Miwic3R5bGUiOnsiaGVhZCI6eyJuYW1lIjoibm9uZSJ9fX1dLFs1LDExLCIiLDAseyJjb2xvdXIiOlswLDYwLDYwXSwic3R5bGUiOnsiYm9keSI6eyJuYW1lIjoiZGFzaGVkIn19fV0sWzQsMTAsIiIsMCx7ImNvbG91ciI6WzAsNjAsNjBdLCJzdHlsZSI6eyJib2R5Ijp7Im5hbWUiOiJkYXNoZWQifX19XV0=
\begin{tikzcd}[ampersand replacement=\&]
	G \& {Q_1} \& \textcolor{rgb,255:red,153;green,92;blue,214}{{P \oplus K_2}} \& {\,} \\
	{P \oplus G} \& {P \oplus Q_1} \& \textcolor{rgb,255:red,153;green,92;blue,214}{{P \oplus K_2}} \& {\,} \\
	\textcolor{rgb,255:red,153;green,92;blue,214}{P} \& \textcolor{rgb,255:red,153;green,92;blue,214}{P} \\
	{\,} \& {\,}
	\arrow[draw={rgb,255:red,92;green,92;blue,214}, dashed, from=1-1, to=1-2]
	\arrow["{\binom 01}", color={rgb,255:red,214;green,92;blue,92}, from=1-1, to=2-1]
	\arrow[draw={rgb,255:red,92;green,92;blue,214}, dashed, from=1-2, to=1-3]
	\arrow["{\binom 01}", color={rgb,255:red,214;green,92;blue,92}, from=1-2, to=2-2]
	\arrow[draw={rgb,255:red,92;green,92;blue,214}, dashed, from=1-3, to=1-4]
	\arrow[equals, from=1-3, to=2-3]
	\arrow["{\binom {1 \ \ 0}{f_1 \ f_2}}", color={rgb,255:red,214;green,92;blue,92}, from=2-1, to=2-2]
	\arrow["{(1,0)}", color={rgb,255:red,214;green,92;blue,92}, from=2-1, to=3-1]
	\arrow["{\binom {1 \ \ l\ }{0 \ m}}", color={rgb,255:red,214;green,92;blue,92}, from=2-2, to=2-3]
	\arrow["{(1,0)}", color={rgb,255:red,214;green,92;blue,92}, from=2-2, to=3-2]
	\arrow[draw={rgb,255:red,214;green,92;blue,92}, dashed, from=2-3, to=2-4]
	\arrow[equals, from=3-1, to=3-2]
	\arrow[draw={rgb,255:red,214;green,92;blue,92}, dashed, from=3-1, to=4-1]
	\arrow[draw={rgb,255:red,214;green,92;blue,92}, dashed, from=3-2, to=4-2]
\end{tikzcd}.
			\end{equation}
			由 \cref{thm:W_exact_3}, 第一行是 $(-,\mathcal{P}(\xi))$-正合的.
		\end{enumerate} 
	\end{proof}
\end{lemma}

\begin{lemma}\label{lem:GP_syzygy_closed}
	对任意 $G \in \mathcal{GP}(\xi)$. 任取 $P,Q \in \mathcal{P}(\xi)$ 使得有 $\xi$-三角
	\begin{equation}
		G \xrightarrow{i} P \xrightarrow{p} G' \overset \delta \dashrightarrow,\quad G'' \xrightarrow{j} Q \xrightarrow{q} G \overset {\varepsilon} \dashrightarrow.
	\end{equation}
	则 $G', G'' \in \mathcal{GP}(\xi)$. 
	\begin{proof}
		考虑 \cref{rmk:pd_well_defined}. 以及 $\mathcal{GP}(\xi)$ 中对象对 $\mathcal{P}(\xi)$-中直和项的消去封闭 (\cref{lem:GP_P_sum}).
	\end{proof}
\end{lemma}

\begin{theorem}\label{thm:GP_resolving}
    给定 $\xi$-三角 $A \xrightarrow f B \xrightarrow g C \overset \delta \dashrightarrow$, 其中 $C \in \mathcal{GP}(\xi)$. 此时 $A \in \mathcal{GP}(\xi)$ 当且仅当 $B \in \mathcal{GP}(\xi)$.
    \begin{proof}
        假定 $A \in \mathcal{GP}(\xi)$. 下图 $(i_A; i_C)$ 是扩张元的态射. 由 WIC-free 的 $3\times 3$-引理得 $\xi$-三角的交换图
        \begin{equation}
% https://q.uiver.app/#q=WzAsMTUsWzAsMCwiQSIsWzI3MCw2MCw2MCwxXV0sWzEsMCwiQiJdLFsyLDAsIkMiLFsyNzAsNjAsNjAsMV1dLFszLDAsIlxcLCJdLFswLDEsIlBfQSIsWzI3MCw2MCw2MCwxXV0sWzEsMSwiUF9BIFxcb3BsdXMgUF9DIixbMjcwLDYwLDYwLDFdXSxbMiwxLCJQX0MiLFsyNzAsNjAsNjAsMV1dLFszLDEsIlxcLCJdLFswLDIsIkEnIixbMjcwLDYwLDYwLDFdXSxbMiwyLCJDJyIsWzI3MCw2MCw2MCwxXV0sWzEsMiwiQiciXSxbMCwzLCJcXCwiXSxbMSwzLCJcXCwiXSxbMiwzLCJcXCwiXSxbMywyLCJcXCwiXSxbMCwxLCJmIiwwLHsiY29sb3VyIjpbMCw2MCw2MF19LFswLDYwLDYwLDFdXSxbMSwyLCJnIiwwLHsiY29sb3VyIjpbMCw2MCw2MF19LFswLDYwLDYwLDFdXSxbMiwzLCJcXGRlbHRhIiwwLHsiY29sb3VyIjpbMCw2MCw2MF0sInN0eWxlIjp7ImJvZHkiOnsibmFtZSI6ImRhc2hlZCJ9fX0sWzAsNjAsNjAsMV1dLFswLDQsImlfQSIsMCx7ImNvbG91ciI6WzAsNjAsNjBdfSxbMCw2MCw2MCwxXV0sWzQsNSwiXFxiaW5vbSAxMCIsMCx7ImNvbG91ciI6WzAsNjAsNjBdfSxbMCw2MCw2MCwxXV0sWzUsNiwiKDAsMSkiLDAseyJjb2xvdXIiOlswLDYwLDYwXX0sWzAsNjAsNjAsMV1dLFs2LDcsIjAiLDAseyJjb2xvdXIiOlswLDYwLDYwXSwic3R5bGUiOnsiYm9keSI6eyJuYW1lIjoiZGFzaGVkIn19fSxbMCw2MCw2MCwxXV0sWzIsNiwiaV9DIiwwLHsiY29sb3VyIjpbMCw2MCw2MF19LFswLDYwLDYwLDFdXSxbNCw4LCJwX0EiLDAseyJjb2xvdXIiOlswLDYwLDYwXX0sWzAsNjAsNjAsMV1dLFs2LDksInBfQyIsMCx7ImNvbG91ciI6WzAsNjAsNjBdfSxbMCw2MCw2MCwxXV0sWzEsNSwiXFxiaW5vbSB7c317aV9DIGd9IiwwLHsiY29sb3VyIjpbMjQwLDYwLDYwXSwic3R5bGUiOnsiYm9keSI6eyJuYW1lIjoiZGFzaGVkIn19fSxbMjQwLDYwLDYwLDFdXSxbOCwxMCwiZiciLDAseyJjb2xvdXIiOlswLDYwLDYwXSwic3R5bGUiOnsiYm9keSI6eyJuYW1lIjoiZGFzaGVkIn19fSxbMCw2MCw2MCwxXV0sWzEwLDksImcnIiwwLHsiY29sb3VyIjpbMCw2MCw2MF0sInN0eWxlIjp7ImJvZHkiOnsibmFtZSI6ImRhc2hlZCJ9fX0sWzAsNjAsNjAsMV1dLFs1LDEwLCIoZidwX0EsIHQpIiwwLHsiY29sb3VyIjpbMjQwLDYwLDYwXSwic3R5bGUiOnsiYm9keSI6eyJuYW1lIjoiZGFzaGVkIn19fSxbMjQwLDYwLDYwLDFdXSxbOCwxMSwiXFx2YXJlcHNpbG9uIF9BIiwwLHsiY29sb3VyIjpbMCw2MCw2MF0sInN0eWxlIjp7ImJvZHkiOnsibmFtZSI6ImRhc2hlZCJ9fX0sWzAsNjAsNjAsMV1dLFsxMCwxMiwiXFx2YXJlcHNpbG9uIF9CIiwwLHsiY29sb3VyIjpbMjQwLDYwLDYwXSwic3R5bGUiOnsiYm9keSI6eyJuYW1lIjoiZGFzaGVkIn19fSxbMjQwLDYwLDYwLDFdXSxbOSwxMywiXFx2YXJlcHNpbG9uIF9DIiwwLHsiY29sb3VyIjpbMCw2MCw2MF0sInN0eWxlIjp7ImJvZHkiOnsibmFtZSI6ImRhc2hlZCJ9fX0sWzAsNjAsNjAsMV1dLFs5LDE0LCJcXGRlbHRhJyIsMCx7ImNvbG91ciI6WzAsNjAsNjBdLCJzdHlsZSI6eyJib2R5Ijp7Im5hbWUiOiJkYXNoZWQifX19LFswLDYwLDYwLDFdXV0=
\begin{tikzcd}[ampersand replacement=\&]
	\textcolor{rgb,255:red,153;green,92;blue,214}{A} \& B \& \textcolor{rgb,255:red,153;green,92;blue,214}{C} \& {\,} \\
	\textcolor{rgb,255:red,153;green,92;blue,214}{{P_A}} \& \textcolor{rgb,255:red,153;green,92;blue,214}{{P_A \oplus P_C}} \& \textcolor{rgb,255:red,153;green,92;blue,214}{{P_C}} \& {\,} \\
	\textcolor{rgb,255:red,153;green,92;blue,214}{{A'}} \& {B'} \& \textcolor{rgb,255:red,153;green,92;blue,214}{{C'}} \& {\,} \\
	{\,} \& {\,} \& {\,}
	\arrow["f", color={rgb,255:red,214;green,92;blue,92}, from=1-1, to=1-2]
	\arrow["{i_A}", color={rgb,255:red,214;green,92;blue,92}, from=1-1, to=2-1]
	\arrow["g", color={rgb,255:red,214;green,92;blue,92}, from=1-2, to=1-3]
	\arrow["{\binom {s}{i_C g}}", color={rgb,255:red,92;green,92;blue,214}, dashed, from=1-2, to=2-2]
	\arrow["\delta", color={rgb,255:red,214;green,92;blue,92}, dashed, from=1-3, to=1-4]
	\arrow["{i_C}", color={rgb,255:red,214;green,92;blue,92}, from=1-3, to=2-3]
	\arrow["{\binom 10}", color={rgb,255:red,214;green,92;blue,92}, from=2-1, to=2-2]
	\arrow["{p_A}", color={rgb,255:red,214;green,92;blue,92}, from=2-1, to=3-1]
	\arrow["{(0,1)}", color={rgb,255:red,214;green,92;blue,92}, from=2-2, to=2-3]
	\arrow["{(f'p_A, t)}", color={rgb,255:red,92;green,92;blue,214}, dashed, from=2-2, to=3-2]
	\arrow["0", color={rgb,255:red,214;green,92;blue,92}, dashed, from=2-3, to=2-4]
	\arrow["{p_C}", color={rgb,255:red,214;green,92;blue,92}, from=2-3, to=3-3]
	\arrow["{f'}", color={rgb,255:red,214;green,92;blue,92}, dashed, from=3-1, to=3-2]
	\arrow["{\varepsilon _A}", color={rgb,255:red,214;green,92;blue,92}, dashed, from=3-1, to=4-1]
	\arrow["{g'}", color={rgb,255:red,214;green,92;blue,92}, dashed, from=3-2, to=3-3]
	\arrow["{\varepsilon _B}", color={rgb,255:red,92;green,92;blue,214}, dashed, from=3-2, to=4-2]
	\arrow["{\delta'}", color={rgb,255:red,214;green,92;blue,92}, dashed, from=3-3, to=3-4]
	\arrow["{\varepsilon _C}", color={rgb,255:red,214;green,92;blue,92}, dashed, from=3-3, to=4-3]
\end{tikzcd}.
        \end{equation}
       	由 \cref{thm:GP_W_exact}, 上图除 $\varepsilon_B$ 外的五个扩张元都是 $(-, \mathcal{P}(\xi))$-正合的. \cref{thm:W_exact_3} 说明中间列 $(-,\mathcal{P}(\xi))$正合. 这说明 $B$ 可以通过中项为 $\mathcal{P}(\xi)$ 的 $\mathcal{P}(\xi)$-正合 $\mathbb E$-三角向右扩张. 对偶地, $B$ 可以同样地向左扩张. 这构造了 $\mathcal{P}(\xi)$-正合的完全 $\mathcal{P}(\xi)$-投射分解.

        假定 $B \in \mathcal{GP}(\xi)$, 作 $(-,\mathcal{P}(\xi))$-正合三角 $B \to Q_B \to B''\dashrightarrow$, 其中 $Q_B \in \mathcal{P}(\xi)$. ET4 给出下图:
        \begin{equation}
% https://q.uiver.app/#q=WzAsMTIsWzAsMCwiQSJdLFsxLDAsIkIiLFsyNzAsNjAsNjAsMV1dLFsyLDAsIkMiLFsyNzAsNjAsNjAsMV1dLFszLDAsIlxcLCJdLFszLDEsIlxcLCJdLFsyLDMsIlxcLCJdLFsxLDMsIlxcLCJdLFsxLDEsIlFfQiIsWzI3MCw2MCw2MCwxXV0sWzIsMSwiRSJdLFsxLDIsIkInJyIsWzI3MCw2MCw2MCwxXV0sWzAsMSwiQSJdLFsyLDIsIkInJyIsWzI3MCw2MCw2MCwxXV0sWzAsMSwiZiIsMCx7ImNvbG91ciI6WzAsNjAsNjBdfSxbMCw2MCw2MCwxXV0sWzEsMiwiZyIsMCx7ImNvbG91ciI6WzAsNjAsNjBdfSxbMCw2MCw2MCwxXV0sWzIsMywiXFxkZWx0YSIsMCx7ImNvbG91ciI6WzAsNjAsNjBdLCJzdHlsZSI6eyJib2R5Ijp7Im5hbWUiOiJkYXNoZWQifX19LFswLDYwLDYwLDFdXSxbMTAsNywiIiwwLHsiY29sb3VyIjpbMjQwLDYwLDYwXSwic3R5bGUiOnsiYm9keSI6eyJuYW1lIjoiZGFzaGVkIn19fV0sWzcsOCwiIiwwLHsiY29sb3VyIjpbMjQwLDYwLDYwXSwic3R5bGUiOnsiYm9keSI6eyJuYW1lIjoiZGFzaGVkIn19fV0sWzEsNywiIiwwLHsiY29sb3VyIjpbMCw2MCw2MF19XSxbOSw2LCJcXHZhcmVwc2lsb24gIiwwLHsiY29sb3VyIjpbMCw2MCw2MF0sInN0eWxlIjp7ImJvZHkiOnsibmFtZSI6ImRhc2hlZCJ9fX0sWzAsNjAsNjAsMV1dLFs3LDksIiIsMCx7ImNvbG91ciI6WzAsNjAsNjBdfV0sWzksMTEsIiIsMCx7ImxldmVsIjoyLCJzdHlsZSI6eyJoZWFkIjp7Im5hbWUiOiJub25lIn19fV0sWzgsMTEsIiIsMCx7ImNvbG91ciI6WzAsNjAsNjBdLCJzdHlsZSI6eyJib2R5Ijp7Im5hbWUiOiJkYXNoZWQifX19XSxbMTEsNSwiXFx2YXJlcHNpbG9uICciLDAseyJjb2xvdXIiOlswLDYwLDYwXSwic3R5bGUiOnsiYm9keSI6eyJuYW1lIjoiZGFzaGVkIn19fSxbMCw2MCw2MCwxXV0sWzAsMTAsIiIsMCx7ImxldmVsIjoyLCJzdHlsZSI6eyJoZWFkIjp7Im5hbWUiOiJub25lIn19fV0sWzIsOCwiIiwwLHsiY29sb3VyIjpbMCw2MCw2MF0sInN0eWxlIjp7ImJvZHkiOnsibmFtZSI6ImRhc2hlZCJ9fX1dLFs4LDQsIlxcZGVsdGEgJyciLDAseyJjb2xvdXIiOlsyNDAsNjAsNjBdLCJzdHlsZSI6eyJib2R5Ijp7Im5hbWUiOiJkYXNoZWQifX19LFsyNDAsNjAsNjAsMV1dXQ==
\begin{tikzcd}[ampersand replacement=\&]
	A \& \textcolor{rgb,255:red,153;green,92;blue,214}{B} \& \textcolor{rgb,255:red,153;green,92;blue,214}{C} \& {\,} \\
	A \& \textcolor{rgb,255:red,153;green,92;blue,214}{{Q_B}} \& E \& {\,} \\
	\& \textcolor{rgb,255:red,153;green,92;blue,214}{{B''}} \& \textcolor{rgb,255:red,153;green,92;blue,214}{{B''}} \\
	\& {\,} \& {\,}
	\arrow["f", color={rgb,255:red,214;green,92;blue,92}, from=1-1, to=1-2]
	\arrow[equals, from=1-1, to=2-1]
	\arrow["g", color={rgb,255:red,214;green,92;blue,92}, from=1-2, to=1-3]
	\arrow[draw={rgb,255:red,214;green,92;blue,92}, from=1-2, to=2-2]
	\arrow["\delta", color={rgb,255:red,214;green,92;blue,92}, dashed, from=1-3, to=1-4]
	\arrow[draw={rgb,255:red,214;green,92;blue,92}, dashed, from=1-3, to=2-3]
	\arrow[color={rgb,255:red,92;green,92;blue,214}, dashed, from=2-1, to=2-2]
	\arrow[color={rgb,255:red,92;green,92;blue,214}, dashed, from=2-2, to=2-3]
	\arrow[draw={rgb,255:red,214;green,92;blue,92}, from=2-2, to=3-2]
	\arrow["{\delta ''}", color={rgb,255:red,92;green,92;blue,214}, dashed, from=2-3, to=2-4]
	\arrow[draw={rgb,255:red,214;green,92;blue,92}, dashed, from=2-3, to=3-3]
	\arrow[equals, from=3-2, to=3-3]
	\arrow["{\varepsilon }", color={rgb,255:red,214;green,92;blue,92}, dashed, from=3-2, to=4-2]
	\arrow["{\varepsilon '}", color={rgb,255:red,214;green,92;blue,92}, dashed, from=3-3, to=4-3]
\end{tikzcd}.
        \end{equation} 
        已证 $\mathcal{GP}(\xi)$ 扩张闭, 故 $E \in \mathcal{GP}(\xi)$. 因此 $\delta ''$ 是 $(-, \mathcal{P}(\xi))$-正合的. \cref{lem:GP_syzygy_closed} 说明 $A \in \mathcal{GP}(\xi)$.
    \end{proof}
\end{theorem}

\begin{proposition}
	$\mathcal{GP}(\xi)$ 对直和项封闭.
	\begin{proof}
		假定 $H \oplus H' \in \mathcal{GP}(\xi)$, 作下图
		\begin{equation}
			% https://q.uiver.app/#q=WzAsMTIsWzAsMCwiSCJdLFsxLDAsIkgiXSxbMCwxLCJIIFxcb3BsdXMgSCciXSxbMCwyLCJIJyJdLFsxLDEsIlAiXSxbMiwxLCJHIixbMjcwLDYwLDYwLDFdXSxbMSwyLCJNIl0sWzIsMiwiRyIsWzI3MCw2MCw2MCwxXV0sWzAsMywiXFwsICJdLFsxLDMsIlxcLCAiXSxbMywxLCJcXCwgIl0sWzMsMiwiXFwsICJdLFswLDEsIiIsMCx7ImxldmVsIjoyLCJzdHlsZSI6eyJoZWFkIjp7Im5hbWUiOiJub25lIn19fV0sWzcsNSwiIiwwLHsibGV2ZWwiOjIsInN0eWxlIjp7ImhlYWQiOnsibmFtZSI6Im5vbmUifX19XSxbMCwyLCIiLDIseyJjb2xvdXIiOlswLDYwLDYwXX1dLFsyLDMsIiIsMix7ImNvbG91ciI6WzAsNjAsNjBdfV0sWzMsOCwiIiwyLHsiY29sb3VyIjpbMCw2MCw2MF0sInN0eWxlIjp7ImJvZHkiOnsibmFtZSI6ImRhc2hlZCJ9fX1dLFsxLDRdLFs0LDZdLFs2LDksIiIsMCx7InN0eWxlIjp7ImJvZHkiOnsibmFtZSI6ImRhc2hlZCJ9fX1dLFsyLDQsIiIsMSx7ImNvbG91ciI6WzAsNjAsNjBdfV0sWzQsNSwiIiwxLHsiY29sb3VyIjpbMCw2MCw2MF19XSxbNSwxMCwiIiwxLHsiY29sb3VyIjpbMCw2MCw2MF0sInN0eWxlIjp7ImJvZHkiOnsibmFtZSI6ImRhc2hlZCJ9fX1dLFszLDYsIiIsMSx7ImNvbG91ciI6WzAsNjAsNjBdfV0sWzYsNywiIiwxLHsiY29sb3VyIjpbMCw2MCw2MF19XSxbNywxMSwiIiwxLHsiY29sb3VyIjpbMCw2MCw2MF0sInN0eWxlIjp7ImJvZHkiOnsibmFtZSI6ImRhc2hlZCJ9fX1dXQ==
\begin{tikzcd}[ampersand replacement=\&]
	H \& H \\
	{H \oplus H'} \& P \& \textcolor{rgb,255:red,153;green,92;blue,214}{G} \& {\, } \\
	{H'} \& M \& \textcolor{rgb,255:red,153;green,92;blue,214}{G} \& {\, } \\
	{\, } \& {\, }
	\arrow[equals, from=1-1, to=1-2]
	\arrow[color={rgb,255:red,214;green,92;blue,92}, from=1-1, to=2-1]
	\arrow[from=1-2, to=2-2]
	\arrow[color={rgb,255:red,214;green,92;blue,92}, from=2-1, to=2-2]
	\arrow[color={rgb,255:red,214;green,92;blue,92}, from=2-1, to=3-1]
	\arrow[color={rgb,255:red,214;green,92;blue,92}, from=2-2, to=2-3]
	\arrow[from=2-2, to=3-2]
	\arrow[color={rgb,255:red,214;green,92;blue,92}, dashed, from=2-3, to=2-4]
	\arrow[color={rgb,255:red,214;green,92;blue,92}, from=3-1, to=3-2]
	\arrow[color={rgb,255:red,214;green,92;blue,92}, dashed, from=3-1, to=4-1]
	\arrow[color={rgb,255:red,214;green,92;blue,92}, from=3-2, to=3-3]
	\arrow[dashed, from=3-2, to=4-2]
	\arrow[equals, from=3-3, to=2-3]
	\arrow[color={rgb,255:red,214;green,92;blue,92}, dashed, from=3-3, to=3-4]
\end{tikzcd}.
		\end{equation} 
		由 $3\times 3$ 引理知所有三角 $(-,\mathcal{P}(\xi))$-正合. 由同样构造得 $(-,\mathcal{P}(\xi))$-正合 $\xi$-三角 $H' \to P \to M' \dashrightarrow$. 由 $H' \oplus H \to P \oplus P \to M \oplus M' \dashrightarrow$ 是 $(-,\mathcal{P}(\xi))$-正合的, 从而 $M$ 是 $\mathcal{GP}(\xi)$ 的直和项. 由 \cref{prop:W_exact_2}, 指向 $\mathcal{GP}(\xi)$ 直和项的 $\xi$-三角也是 $(-,\mathcal{P}(\xi))$-正合的. 存在 $(-, \mathcal{P}(\xi))$-正合的 $\xi$-三角 $L \to Q \to H \dashrightarrow$ 使得 $L$ 是 $\mathcal{GP}(\xi)$ 的直和项.

		综上, $\mathcal{GP}(\xi)$ 的直和项可向左且向右扩张为 $(-, \mathcal{P}(\xi))$-正合的 $\xi$-三角, 中间项是 $\mathcal{P}(\xi)$. 这说明 $\mathcal{GP}(\xi)$ 对直和项封闭.
	\end{proof}
\end{proposition}

\begin{corollary}
	给定 $\xi$-三角 $A \xrightarrow f B \xrightarrow g C \overset \delta \dashrightarrow$. 若 $A, B \in \mathcal{GP}(\xi)$, 则 $C \in \mathcal{GP}(\xi)$ 当且仅当 $\mathbb E$-三角是 $(-, \mathcal{P}(\xi))$-正合的.
	\begin{proof}
		由 \cref{thm:GP_W_exact} 可知必要性 ($\to$ 方向). 对充分性 ($\gets$ 方向), 若 $\mathbb E$-三角是 $\mathcal{P}(\xi)$-正合的, 则其关于 $A \to P_A$ 的推出是总可裂短正合列. 作 $\xi$-三角的交换图:
		\begin{equation}
% https://q.uiver.app/#q=WzAsMTIsWzAsMCwiQSIsWzI3MCw2MCw2MCwxXV0sWzEsMCwiQiIsWzI3MCw2MCw2MCwxXV0sWzIsMCwiQyJdLFszLDAsIlxcLCJdLFswLDEsIlBfQSIsWzI3MCw2MCw2MCwxXV0sWzAsMiwiQSciLFsyNzAsNjAsNjAsMV1dLFswLDMsIlxcLCJdLFsxLDEsIlBfQSBcXG9wbHVzIEMiXSxbMiwxLCJDIl0sWzEsMiwiQSciLFsyNzAsNjAsNjAsMV1dLFszLDEsIlxcLCJdLFsxLDMsIlxcLCJdLFswLDEsIiIsMCx7ImNvbG91ciI6WzAsNjAsNjBdfV0sWzQsNywiIiwwLHsiY29sb3VyIjpbMCw2MCw2MF0sInN0eWxlIjp7ImJvZHkiOnsibmFtZSI6ImRhc2hlZCJ9fX1dLFs3LDgsIiIsMCx7ImNvbG91ciI6WzAsNjAsNjBdLCJzdHlsZSI6eyJib2R5Ijp7Im5hbWUiOiJkYXNoZWQifX19XSxbNSw5LCIiLDAseyJsZXZlbCI6Miwic3R5bGUiOnsiaGVhZCI6eyJuYW1lIjoibm9uZSJ9fX1dLFsxLDIsIiIsMCx7ImNvbG91ciI6WzAsNjAsNjBdfV0sWzIsMywiIiwwLHsiY29sb3VyIjpbMCw2MCw2MF0sInN0eWxlIjp7ImJvZHkiOnsibmFtZSI6ImRhc2hlZCJ9fX1dLFs4LDEwLCIiLDAseyJjb2xvdXIiOlswLDYwLDYwXSwic3R5bGUiOnsiYm9keSI6eyJuYW1lIjoiZGFzaGVkIn19fV0sWzUsNiwiIiwyLHsic3R5bGUiOnsiYm9keSI6eyJuYW1lIjoiZGFzaGVkIn19fV0sWzksMTEsIiIsMCx7InN0eWxlIjp7ImJvZHkiOnsibmFtZSI6ImRhc2hlZCJ9fX1dLFswLDRdLFs0LDVdLFsxLDcsIiIsMSx7InN0eWxlIjp7ImJvZHkiOnsibmFtZSI6ImRhc2hlZCJ9fX1dLFs3LDksIiIsMSx7InN0eWxlIjp7ImJvZHkiOnsibmFtZSI6ImRhc2hlZCJ9fX1dLFsyLDgsIiIsMSx7ImxldmVsIjoyLCJzdHlsZSI6eyJoZWFkIjp7Im5hbWUiOiJub25lIn19fV1d
\begin{tikzcd}[ampersand replacement=\&]
	\textcolor{rgb,255:red,153;green,92;blue,214}{A} \& \textcolor{rgb,255:red,153;green,92;blue,214}{B} \& C \& {\,} \\
	\textcolor{rgb,255:red,153;green,92;blue,214}{{P_A}} \& {P_A \oplus C} \& C \& {\,} \\
	\textcolor{rgb,255:red,153;green,92;blue,214}{{A'}} \& \textcolor{rgb,255:red,153;green,92;blue,214}{{A'}} \\
	{\,} \& {\,}
	\arrow[color={rgb,255:red,214;green,92;blue,92}, from=1-1, to=1-2]
	\arrow[from=1-1, to=2-1]
	\arrow[color={rgb,255:red,214;green,92;blue,92}, from=1-2, to=1-3]
	\arrow[dashed, from=1-2, to=2-2]
	\arrow[color={rgb,255:red,214;green,92;blue,92}, dashed, from=1-3, to=1-4]
	\arrow[equals, from=1-3, to=2-3]
	\arrow[color={rgb,255:red,214;green,92;blue,92}, dashed, from=2-1, to=2-2]
	\arrow[from=2-1, to=3-1]
	\arrow[color={rgb,255:red,214;green,92;blue,92}, dashed, from=2-2, to=2-3]
	\arrow[dashed, from=2-2, to=3-2]
	\arrow[color={rgb,255:red,214;green,92;blue,92}, dashed, from=2-3, to=2-4]
	\arrow[equals, from=3-1, to=3-2]
	\arrow[dashed, from=3-1, to=4-1]
	\arrow[dashed, from=3-2, to=4-2]
\end{tikzcd}.
		\end{equation}
		由 \cref{thm:GP_resolving} 知 $P_A \oplus C \in \mathcal{GP}(\xi)$, 再由 \cref{lem:GP_P_sum} 得 $C \in \mathcal{GP}(\xi)$.
	\end{proof}
\end{corollary}

\subsection{\texorpdfstring{$\mathcal{GP}(\xi)$}{}-投射维数}

本小节在外三角范畴中类推经典 Gorenstein 同调维数的相关定理.

\begin{definition}[$\mathcal{GP}(\xi)$-投射维数]
	约定 $\mathcal{GP}(\xi)$ 中对象的 $\mathcal{GP}(\xi)$-投射维数为 $0$. 归纳地, 称 $\mathcal{G} pd_\xi M \leq n + 1$, 若存在 $\xi$-三角 $K \xrightarrow f G \xrightarrow g M \overset \delta \dashrightarrow$ 使得 $G \in \mathcal{GP}(\xi)$ 且 $\mathcal{G} pd_\xi K \leq n$. 称 $\mathcal{G} pd_\xi M = n$ 若 $\mathcal{G} pd_\xi M \leq n$ 且不满足 $\mathcal{G} pd_\xi M \leq n-1$. 
\end{definition}

\begin{lemma}[维数移动]
	使用 $G$ 表示 $\mathcal{GP}(\xi)$ 中的某个元素. 以下三个命题对 $n \geq 0$ 成立:
	\begin{enumerate}
		\item[$\mathfrak A(n)$] 对 $\xi$-三角 $A \to G \to C \dashrightarrow$, 若 $\mathcal{G}pd_\xi C = n+1$, 则 $\mathcal{G}pd_\xi A = n$.
		\item[$\mathfrak B(n)$] 对 $\xi$-三角 $A \to B \to G \dashrightarrow$, 若 $\mathcal{G}pd_\xi A = n$, 则 $\mathcal{G}pd_\xi B = n$.
		\item[$\mathfrak C(n)$] 对 $\xi$-三角 $A \to B \to G \dashrightarrow$, 若 $\mathcal{G}pd_\xi B = n$, 则 $\mathcal{G}pd_\xi A = n$.
	\end{enumerate}
	\begin{proof}
		使用数学归纳法证明. 由 \cref{thm:GP_resolving}, $\mathfrak B(0)$ 与 $\mathfrak C(0)$ 成立. 下证明 $\mathfrak A(0)$. 存在 $G', G'' \in \mathcal{GP}(\xi)$ 使得
		\begin{equation}
% https://q.uiver.app/#q=WzAsMTIsWzIsMCwiRyciLFsyNzAsNjAsNjAsMV1dLFsyLDEsIkcnJyIsWzI3MCw2MCw2MCwxXV0sWzIsMiwiQyJdLFsxLDIsIkciLFsyNzAsNjAsNjAsMV1dLFswLDIsIkEiXSxbMCwxLCJBIl0sWzEsMCwiRyciLFsyNzAsNjAsNjAsMV1dLFsxLDEsIkUiXSxbMywxLCJcXCwiXSxbMywyLCJcXCwiXSxbMiwzLCJcXCwiXSxbMSwzLCJcXCwiXSxbNSw0LCIiLDEseyJsZXZlbCI6Miwic3R5bGUiOnsiaGVhZCI6eyJuYW1lIjoibm9uZSJ9fX1dLFs2LDAsIiIsMSx7ImxldmVsIjoyLCJzdHlsZSI6eyJoZWFkIjp7Im5hbWUiOiJub25lIn19fV0sWzUsNywiIiwxLHsiY29sb3VyIjpbMCw2MCw2MF0sInN0eWxlIjp7ImJvZHkiOnsibmFtZSI6ImRhc2hlZCJ9fX1dLFs3LDEsIiIsMSx7ImNvbG91ciI6WzAsNjAsNjBdLCJzdHlsZSI6eyJib2R5Ijp7Im5hbWUiOiJkYXNoZWQifX19XSxbNiw3LCIiLDEseyJjb2xvdXIiOlswLDYwLDYwXSwic3R5bGUiOnsiYm9keSI6eyJuYW1lIjoiZGFzaGVkIn19fV0sWzcsMywiIiwxLHsiY29sb3VyIjpbMCw2MCw2MF0sInN0eWxlIjp7ImJvZHkiOnsibmFtZSI6ImRhc2hlZCJ9fX1dLFs0LDNdLFszLDJdLFswLDFdLFsxLDJdLFsyLDEwLCIiLDEseyJzdHlsZSI6eyJib2R5Ijp7Im5hbWUiOiJkYXNoZWQifX19XSxbMywxMSwiIiwxLHsiY29sb3VyIjpbMCw2MCw2MF0sInN0eWxlIjp7ImJvZHkiOnsibmFtZSI6ImRhc2hlZCJ9fX1dLFsxLDgsIiIsMSx7ImNvbG91ciI6WzAsNjAsNjBdLCJzdHlsZSI6eyJib2R5Ijp7Im5hbWUiOiJkYXNoZWQifX19XSxbMiw5LCIiLDEseyJzdHlsZSI6eyJib2R5Ijp7Im5hbWUiOiJkYXNoZWQifX19XV0=
\begin{tikzcd}[ampersand replacement=\&]
	\& \textcolor{rgb,255:red,153;green,92;blue,214}{{G'}} \& \textcolor{rgb,255:red,153;green,92;blue,214}{{G'}} \\
	A \& E \& \textcolor{rgb,255:red,153;green,92;blue,214}{{G''}} \& {\,} \\
	A \& \textcolor{rgb,255:red,153;green,92;blue,214}{G} \& C \& {\,} \\
	\& {\,} \& {\,}
	\arrow[equals, from=1-2, to=1-3]
	\arrow[draw={rgb,255:red,214;green,92;blue,92}, dashed, from=1-2, to=2-2]
	\arrow[from=1-3, to=2-3]
	\arrow[draw={rgb,255:red,214;green,92;blue,92}, dashed, from=2-1, to=2-2]
	\arrow[equals, from=2-1, to=3-1]
	\arrow[draw={rgb,255:red,214;green,92;blue,92}, dashed, from=2-2, to=2-3]
	\arrow[draw={rgb,255:red,214;green,92;blue,92}, dashed, from=2-2, to=3-2]
	\arrow[draw={rgb,255:red,214;green,92;blue,92}, dashed, from=2-3, to=2-4]
	\arrow[from=2-3, to=3-3]
	\arrow[from=3-1, to=3-2]
	\arrow[from=3-2, to=3-3]
	\arrow[draw={rgb,255:red,214;green,92;blue,92}, dashed, from=3-2, to=4-2]
	\arrow[dashed, from=3-3, to=3-4]
	\arrow[dashed, from=3-3, to=4-3]
\end{tikzcd}.
		\end{equation}
		由 \cref{thm:GP_resolving}, $E\in \mathcal{GP}(\xi)$, 进而 $A \in \mathcal{GP}(\xi)$.
		\begin{itemize}
			\item (归纳: $\mathfrak B(n+1) \land \mathfrak C(n+1) \to \mathfrak A(n+1)$). 考虑
			\begin{equation}
% https://q.uiver.app/#q=WzAsMTQsWzIsMCwiQyciXSxbMiwxLCJHJyciLFsyNzAsNjAsNjAsMV1dLFsyLDIsIkMiXSxbMSwyLCJHIixbMjcwLDYwLDYwLDFdXSxbMCwyLCJBIl0sWzAsMSwiQSJdLFsxLDAsIkMnIl0sWzEsMSwiRSJdLFszLDEsIlxcLCJdLFszLDIsIlxcLCJdLFsyLDMsIlxcLCJdLFsxLDMsIlxcLCJdLFs0LDIsIlxcbWF0aGNhbCBHcGRfXFx4aSBDID0gbisyIl0sWzQsMCwiXFxtYXRoY2FsIEdwZF9cXHhpIEMnID0gbisxIl0sWzUsNCwiIiwxLHsibGV2ZWwiOjIsInN0eWxlIjp7ImhlYWQiOnsibmFtZSI6Im5vbmUifX19XSxbNiwwLCIiLDEseyJsZXZlbCI6Miwic3R5bGUiOnsiaGVhZCI6eyJuYW1lIjoibm9uZSJ9fX1dLFs1LDcsIiIsMSx7ImNvbG91ciI6WzAsNjAsNjBdLCJzdHlsZSI6eyJib2R5Ijp7Im5hbWUiOiJkYXNoZWQifX19XSxbNywxLCIiLDEseyJjb2xvdXIiOlswLDYwLDYwXSwic3R5bGUiOnsiYm9keSI6eyJuYW1lIjoiZGFzaGVkIn19fV0sWzYsNywiIiwxLHsiY29sb3VyIjpbMCw2MCw2MF0sInN0eWxlIjp7ImJvZHkiOnsibmFtZSI6ImRhc2hlZCJ9fX1dLFs3LDMsIiIsMSx7ImNvbG91ciI6WzAsNjAsNjBdLCJzdHlsZSI6eyJib2R5Ijp7Im5hbWUiOiJkYXNoZWQifX19XSxbNCwzXSxbMywyXSxbMCwxXSxbMSwyXSxbMiwxMCwiIiwxLHsic3R5bGUiOnsiYm9keSI6eyJuYW1lIjoiZGFzaGVkIn19fV0sWzMsMTEsIiIsMSx7ImNvbG91ciI6WzAsNjAsNjBdLCJzdHlsZSI6eyJib2R5Ijp7Im5hbWUiOiJkYXNoZWQifX19XSxbMSw4LCIiLDEseyJjb2xvdXIiOlswLDYwLDYwXSwic3R5bGUiOnsiYm9keSI6eyJuYW1lIjoiZGFzaGVkIn19fV0sWzIsOSwiIiwxLHsic3R5bGUiOnsiYm9keSI6eyJuYW1lIjoiZGFzaGVkIn19fV1d
\begin{tikzcd}[ampersand replacement=\&]
	\& {C'} \& {C'} \&\& {\mathcal Gpd_\xi C' = n+1} \\
	A \& E \& \textcolor{rgb,255:red,153;green,92;blue,214}{{G''}} \& {\,} \\
	A \& \textcolor{rgb,255:red,153;green,92;blue,214}{G} \& C \& {\,} \& {\mathcal Gpd_\xi C = n+2} \\
	\& {\,} \& {\,}
	\arrow[equals, from=1-2, to=1-3]
	\arrow[draw={rgb,255:red,214;green,92;blue,92}, dashed, from=1-2, to=2-2]
	\arrow[from=1-3, to=2-3]
	\arrow[draw={rgb,255:red,214;green,92;blue,92}, dashed, from=2-1, to=2-2]
	\arrow[equals, from=2-1, to=3-1]
	\arrow[draw={rgb,255:red,214;green,92;blue,92}, dashed, from=2-2, to=2-3]
	\arrow[draw={rgb,255:red,214;green,92;blue,92}, dashed, from=2-2, to=3-2]
	\arrow[draw={rgb,255:red,214;green,92;blue,92}, dashed, from=2-3, to=2-4]
	\arrow[from=2-3, to=3-3]
	\arrow[from=3-1, to=3-2]
	\arrow[from=3-2, to=3-3]
	\arrow[draw={rgb,255:red,214;green,92;blue,92}, dashed, from=3-2, to=4-2]
	\arrow[dashed, from=3-3, to=3-4]
	\arrow[dashed, from=3-3, to=4-3]
\end{tikzcd}.
			\end{equation}
			由 $\mathfrak B(n+1)$, 得 $\mathcal{G}pd_\xi E = n+1$. 由 $\mathfrak A(n+1)$, 得 $\mathcal{G}pd_\xi A = n + 1$.
			\item (归纳: $\mathfrak A(n) \land \mathfrak B(n) \to \mathfrak B(n+1)$). 由 WIC-free 的 $3\times 3$-引理作得下图
			\begin{equation}
% https://q.uiver.app/#q=WzAsMTcsWzAsMiwiQSJdLFsxLDIsIkIiXSxbMiwyLCJHIixbMjcwLDYwLDYwLDFdXSxbMywyLCJcXCwiXSxbNCwyLCJcXG1hdGhjYWwgR3BkX1xceGkgQSA9IG4rMSJdLFswLDEsIkcnJyJdLFswLDAsIkEnIl0sWzIsMCwiRyciLFsyNzAsNjAsNjAsMV1dLFsyLDEsIlAiLFsyNzAsNjAsNjAsMV1dLFszLDAsIlxcLCJdLFswLDMsIlxcLCJdLFsyLDMsIlxcLCJdLFszLDEsIlxcLCAiXSxbMSwwLCJCJyJdLFsxLDEsIkcnJyBcXG9wbHVzIFAiXSxbMSwzLCJcXCwgIl0sWzQsMCwiXFxtYXRoY2FsIEdwZF9cXHhpIEEnID0gbiJdLFswLDEsIiIsMCx7ImNvbG91ciI6WzAsNjAsNjBdfV0sWzEsMiwiIiwwLHsiY29sb3VyIjpbMCw2MCw2MF19XSxbMiwzLCIiLDAseyJjb2xvdXIiOlswLDYwLDYwXSwic3R5bGUiOnsiYm9keSI6eyJuYW1lIjoiZGFzaGVkIn19fV0sWzYsNV0sWzUsMF0sWzAsMTAsIiIsMCx7InN0eWxlIjp7ImJvZHkiOnsibmFtZSI6ImRhc2hlZCJ9fX1dLFs3LDgsIiIsMCx7ImNvbG91ciI6WzAsNjAsNjBdfV0sWzgsMiwiIiwwLHsiY29sb3VyIjpbMCw2MCw2MF19XSxbMiwxMSwiIiwwLHsiY29sb3VyIjpbMCw2MCw2MF0sInN0eWxlIjp7ImJvZHkiOnsibmFtZSI6ImRhc2hlZCJ9fX1dLFs2LDEzLCIiLDIseyJjb2xvdXIiOlswLDYwLDYwXSwic3R5bGUiOnsiYm9keSI6eyJuYW1lIjoiZGFzaGVkIn19fV0sWzEzLDcsIiIsMSx7ImNvbG91ciI6WzAsNjAsNjBdLCJzdHlsZSI6eyJib2R5Ijp7Im5hbWUiOiJkYXNoZWQifX19XSxbNyw5LCIiLDEseyJjb2xvdXIiOlswLDYwLDYwXSwic3R5bGUiOnsiYm9keSI6eyJuYW1lIjoiZGFzaGVkIn19fV0sWzUsMTQsIiIsMSx7ImNvbG91ciI6WzAsNjAsNjBdfV0sWzE0LDgsIiIsMSx7ImNvbG91ciI6WzAsNjAsNjBdfV0sWzgsMTIsIjAiLDAseyJjb2xvdXIiOlswLDYwLDYwXSwic3R5bGUiOnsiYm9keSI6eyJuYW1lIjoiZGFzaGVkIn19fSxbMCw2MCw2MCwxXV0sWzEzLDE0LCIiLDEseyJzdHlsZSI6eyJib2R5Ijp7Im5hbWUiOiJkYXNoZWQifX19XSxbMTQsMSwiIiwxLHsic3R5bGUiOnsiYm9keSI6eyJuYW1lIjoiZGFzaGVkIn19fV0sWzEsMTUsIiIsMSx7InN0eWxlIjp7ImJvZHkiOnsibmFtZSI6ImRhc2hlZCJ9fX1dXQ==
\begin{tikzcd}[ampersand replacement=\&]
	{A'} \& {B'} \& \textcolor{rgb,255:red,153;green,92;blue,214}{{G'}} \& {\,} \& {\mathcal Gpd_\xi A' = n} \\
	{G''} \& {G'' \oplus P} \& \textcolor{rgb,255:red,153;green,92;blue,214}{P} \& {\, } \\
	A \& B \& \textcolor{rgb,255:red,153;green,92;blue,214}{G} \& {\,} \& {\mathcal Gpd_\xi A = n+1} \\
	{\,} \& {\, } \& {\,}
	\arrow[draw={rgb,255:red,214;green,92;blue,92}, dashed, from=1-1, to=1-2]
	\arrow[from=1-1, to=2-1]
	\arrow[draw={rgb,255:red,214;green,92;blue,92}, dashed, from=1-2, to=1-3]
	\arrow[dashed, from=1-2, to=2-2]
	\arrow[draw={rgb,255:red,214;green,92;blue,92}, dashed, from=1-3, to=1-4]
	\arrow[draw={rgb,255:red,214;green,92;blue,92}, from=1-3, to=2-3]
	\arrow[draw={rgb,255:red,214;green,92;blue,92}, from=2-1, to=2-2]
	\arrow[from=2-1, to=3-1]
	\arrow[draw={rgb,255:red,214;green,92;blue,92}, from=2-2, to=2-3]
	\arrow[dashed, from=2-2, to=3-2]
	\arrow["0", color={rgb,255:red,214;green,92;blue,92}, dashed, from=2-3, to=2-4]
	\arrow[draw={rgb,255:red,214;green,92;blue,92}, from=2-3, to=3-3]
	\arrow[draw={rgb,255:red,214;green,92;blue,92}, from=3-1, to=3-2]
	\arrow[dashed, from=3-1, to=4-1]
	\arrow[draw={rgb,255:red,214;green,92;blue,92}, from=3-2, to=3-3]
	\arrow[dashed, from=3-2, to=4-2]
	\arrow[draw={rgb,255:red,214;green,92;blue,92}, dashed, from=3-3, to=3-4]
	\arrow[draw={rgb,255:red,214;green,92;blue,92}, dashed, from=3-3, to=4-3]
\end{tikzcd}.
			\end{equation}
			由 $\mathfrak B(n)$ 知 $\mathcal Gpd_\xi B' = n$. 由 $\mathfrak A(n)$ 知 $\mathcal{G}pd_\xi B = n+1$.
			\item (归纳: $\mathfrak A(n) \to \mathfrak C(n+1)$). 考虑
			\begin{equation}
% https://q.uiver.app/#q=WzAsMTYsWzAsMiwiQSJdLFsxLDIsIkIiXSxbMiwyLCJHIixbMjcwLDYwLDYwLDFdXSxbMywyLCJcXCwiXSxbNCwyLCJcXG1hdGhjYWwgR3BkX1xceGkgQiA9IG4rMSJdLFswLDEsIkgiXSxbMiwxLCJHIixbMjcwLDYwLDYwLDFdXSxbMywwLCJcXCwiXSxbMCwzLCJcXCwiXSxbMiwzLCJcXCwiXSxbMywxLCJcXCwgIl0sWzEsMCwiQiciXSxbMSwzLCJcXCwgIl0sWzQsMCwiXFxtYXRoY2FsIEdwZF9cXHhpIEInID0gbiJdLFsxLDEsIkcnIixbMjcwLDYwLDYwLDFdXSxbMCwwLCJCJyJdLFswLDEsIiIsMCx7ImNvbG91ciI6WzAsNjAsNjBdfV0sWzEsMiwiIiwwLHsiY29sb3VyIjpbMCw2MCw2MF19XSxbMiwzLCIiLDAseyJzdHlsZSI6eyJib2R5Ijp7Im5hbWUiOiJkYXNoZWQifX19XSxbNSwwLCIiLDAseyJzdHlsZSI6eyJib2R5Ijp7Im5hbWUiOiJkYXNoZWQifX19XSxbMCw4LCIiLDAseyJzdHlsZSI6eyJib2R5Ijp7Im5hbWUiOiJkYXNoZWQifX19XSxbMiw5LCIiLDAseyJzdHlsZSI6eyJib2R5Ijp7Im5hbWUiOiJkYXNoZWQifX19XSxbNiwxMCwiIiwxLHsic3R5bGUiOnsiYm9keSI6eyJuYW1lIjoiZGFzaGVkIn19fV0sWzEsMTIsIiIsMSx7InN0eWxlIjp7ImJvZHkiOnsibmFtZSI6ImRhc2hlZCJ9fX1dLFsxMSwxNF0sWzUsMTQsIiIsMSx7ImNvbG91ciI6WzAsNjAsNjBdfV0sWzE0LDYsIiIsMSx7ImNvbG91ciI6WzAsNjAsNjBdfV0sWzE0LDFdLFsxNSwxMSwiIiwyLHsibGV2ZWwiOjIsInN0eWxlIjp7ImhlYWQiOnsibmFtZSI6Im5vbmUifX19XSxbMTUsNSwiIiwyLHsic3R5bGUiOnsiYm9keSI6eyJuYW1lIjoiZGFzaGVkIn19fV0sWzYsMiwiIiwwLHsibGV2ZWwiOjIsInN0eWxlIjp7ImhlYWQiOnsibmFtZSI6Im5vbmUifX19XV0=
\begin{tikzcd}[ampersand replacement=\&]
	{B'} \& {B'} \&\& {\,} \& {\mathcal Gpd_\xi B' = n} \\
	H \& \textcolor{rgb,255:red,153;green,92;blue,214}{{G'}} \& \textcolor{rgb,255:red,153;green,92;blue,214}{G} \& {\, } \\
	A \& B \& \textcolor{rgb,255:red,153;green,92;blue,214}{G} \& {\,} \& {\mathcal Gpd_\xi B = n+1} \\
	{\,} \& {\, } \& {\,}
	\arrow[equals, from=1-1, to=1-2]
	\arrow[dashed, from=1-1, to=2-1]
	\arrow[from=1-2, to=2-2]
	\arrow[draw={rgb,255:red,214;green,92;blue,92}, from=2-1, to=2-2]
	\arrow[dashed, from=2-1, to=3-1]
	\arrow[draw={rgb,255:red,214;green,92;blue,92}, from=2-2, to=2-3]
	\arrow[from=2-2, to=3-2]
	\arrow[dashed, from=2-3, to=2-4]
	\arrow[equals, from=2-3, to=3-3]
	\arrow[draw={rgb,255:red,214;green,92;blue,92}, from=3-1, to=3-2]
	\arrow[dashed, from=3-1, to=4-1]
	\arrow[draw={rgb,255:red,214;green,92;blue,92}, from=3-2, to=3-3]
	\arrow[dashed, from=3-2, to=4-2]
	\arrow[dashed, from=3-3, to=3-4]
	\arrow[dashed, from=3-3, to=4-3]
\end{tikzcd}.
			\end{equation}
			由 \cref{thm:GP_resolving} 得 $H \in \mathcal{GP}$. 由 $\mathfrak A(n)$ 知 $\mathcal Gpd_\xi A = n+1$.
		\end{itemize}
		归纳图示:
		\begin{equation}
% https://q.uiver.app/#q=WzAsNixbMCwwLCJcXG1hdGhmcmFrIEEobikiXSxbMCwxLCJcXG1hdGhmcmFrIEIobikiXSxbMiwyLCJcXG1hdGhmcmFrIEMobisxKSJdLFsyLDEsIlxcbWF0aGZyYWsgQihuKzEpIl0sWzIsMCwiXFxtYXRoZnJhayBBKG4rMSkiXSxbMCwyLCJcXG1hdGhmcmFrIEMobikiXSxbMCwyLCIiLDAseyJjb2xvdXIiOlswLDYwLDYwXX1dLFswLDMsIiIsMix7ImNvbG91ciI6WzI0MCw2MCw2MF19XSxbMSwzLCIiLDAseyJjb2xvdXIiOlsyNDAsNjAsNjBdfV0sWzMsNCwiIiwwLHsiY29sb3VyIjpbMTIwLDYwLDYwXX1dLFsyLDQsIiIsMSx7ImN1cnZlIjo1LCJjb2xvdXIiOlsxMjAsNjAsNjBdfV1d
\begin{tikzcd}[ampersand replacement=\&]
	{\mathfrak A(n)} \&\& {\mathfrak A(n+1)} \\
	{\mathfrak B(n)} \&\& {\mathfrak B(n+1)} \\
	{\mathfrak C(n)} \&\& {\mathfrak C(n+1)}
	\arrow[color={rgb,255:red,92;green,92;blue,214}, from=1-1, to=2-3]
	\arrow[color={rgb,255:red,214;green,92;blue,92}, from=1-1, to=3-3]
	\arrow[color={rgb,255:red,92;green,92;blue,214}, from=2-1, to=2-3]
	\arrow[color={rgb,255:red,92;green,214;blue,92}, from=2-3, to=1-3]
	\arrow[color={rgb,255:red,92;green,214;blue,92}, curve={height=30pt}, from=3-3, to=1-3]
\end{tikzcd}.
		\end{equation}
	\end{proof}
\end{lemma}

\begin{corollary}
	对 $\xi$-三角 $A \to G \to C \dashrightarrow$, 若 $\mathcal{G}pd_\xi A = n+1$, 则 $\mathcal{G}pd_\xi A = n+2$.
	\begin{proof}
		假定 $\mathcal{G}pd_\xi A = n + 1$. 由定义, $\mathcal{G}pd_\xi C\leq n+2$. 由 $\mathfrak A(n)$ 知只能有 $\mathcal{G}pd_\xi C = n+2$.  
	\end{proof}
\end{corollary}

\begin{corollary}[$\mathcal{G}pd_\xi$ 的等价定义]
	$\mathcal{G}pd_\xi M \leq n$ 当且仅当对任意 $\xi$-正合复形
	\begin{equation}
		K_n \xrightarrow{d_n} P_{n-1} \xrightarrow{d_{n-1}} \cdots \xrightarrow{d_1} P_0 \xrightarrow{d_0} M\quad (P_i \in \mathcal{P}(\xi)),
	\end{equation}
	总有 $K_n \in \mathcal{GP}(\xi)$.
	\begin{proof}
		($\to$). 依照 $\mathcal{P}(\xi) \subseteq \mathcal{GP}(\xi)$ 与维数移位. ($\gets$). 取如上所述的 $M$. 依照定义, $\mathcal{G}pd_\xi M \leq n$. 
	\end{proof}
\end{corollary}

\begin{corollary}
	若 $pd_\xi M < \infty$, 则 $pd_\xi M = \mathcal{G}pd_\xi M$.
\end{corollary}

\begin{notation}
	给定 $I \subseteq (\mathbb N \cup \{\infty\})$.
	\begin{itemize}
		\item 记 $\mathcal{P}(\xi)$ 是 $\mathcal{P}(\xi)$-投射对象, $\widetilde{\mathcal{P}(\xi)}$ 是 $pd_\xi$ 有限的对象, ${\mathcal{P}(\xi)}^I$ 是 $\mathcal{P}(\xi)$-投射维数在 $I$ 中的对象.
		\item 记 $\mathcal{GP}(\xi)$ 是 $\mathcal{GP}(\xi)$-投射对象, $\widetilde{\mathcal{GP}(\xi)}$ 是 $\mathcal{GP}(\xi)$-投射维数有限的对象, ${\mathcal{GP}(\xi)}^I$ 是 $\mathcal{GP}(\xi)$-投射维数在 $I$ 内的对象.
	\end{itemize}
\end{notation}

\begin{proposition}\label{prop:GP_ext_vanish}
	$\mathbb E_\xi(\mathcal{GP}(\xi), \widetilde{\mathcal{P}(\xi)}) = 0$.
	\begin{proof}
		只需证明 $\mathbb E_\xi(\mathcal{GP}(\xi), {\mathcal{P}(\xi)}^{\leq n}) = 0$ 对一切 $n \geq 0$ 成立. 由 \cref{thm:GP_resolving} 知 $n = 0$ 时显然成立. 假定 $\mathbb E_\xi(\mathcal{GP}(\xi), {\mathcal{P}(\xi)}^{\leq n}) = 0$ 成立, 下证明 $\mathbb E_\xi(\mathcal{GP}(\xi), {\mathcal{P}(\xi)}^{\leq n+1}) = 0$. 取定
		\begin{equation}
			N \xrightarrow{f} P \xrightarrow{g} M \dashrightarrow ,\quad pd_\xi N = n, \quad P\in \mathcal{P}(\xi), \quad pd_\xi M = n+1,
		\end{equation}
		以及 $(-, \mathcal{P})$-正合 $\mathbb E$-三角
		\begin{equation}
			G' \xrightarrow{i} Q \xrightarrow{p} G \dashrightarrow ,\quad G',G \in \mathcal{GP}(\xi), \quad Q \in \mathcal{P}(\xi).
		\end{equation}
		此时有正合列的交换图 ($\color{rgb,255:red,92;green,214;blue,214}\blacksquare$ 色对象是零 Abel 群):
		\begin{equation}
% https://q.uiver.app/#q=WzAsOSxbMCwwLCIoUSxQKSJdLFsxLDAsIihRLE0pIl0sWzIsMCwiXFxtYXRoYmIgRV9cXHhpKFEsTikiLFsxODAsNjAsNjAsMV1dLFswLDEsIihHJyxQKSJdLFsxLDEsIihHJyxNKSJdLFswLDIsIlxcbWF0aGJiIEVfXFx4aShHLFApIixbMTgwLDYwLDYwLDFdXSxbMSwyLCJcXG1hdGhiYiBFX1xceGkoRyxNKSJdLFsyLDEsIlxcbWF0aGJiIEVfXFx4aShHJyxOKSIsWzE4MCw2MCw2MCwxXV0sWzEsMywiXFxtYXRoYmIgRV9cXHhpKFEsTSkiLFsxODAsNjAsNjAsMV1dLFswLDFdLFsxLDJdLFszLDVdLFsxLDRdLFs0LDYsIigoXFxkZWx0YV9cXHhpKV5cXHNoYXJwKV9NIl0sWzIsN10sWzQsN10sWzUsNl0sWzAsM10sWzMsNCwiKEcnLCBnKSJdLFs2LDhdXQ==
\begin{tikzcd}[ampersand replacement=\&]
	{(Q,P)} \& {(Q,M)} \& \textcolor{rgb,255:red,92;green,214;blue,214}{{\mathbb E_\xi(Q,N)}} \\
	{(G',P)} \& {(G',M)} \& \textcolor{rgb,255:red,92;green,214;blue,214}{{\mathbb E_\xi(G',N)}} \\
	\textcolor{rgb,255:red,92;green,214;blue,214}{{\mathbb E_\xi(G,P)}} \& {\mathbb E_\xi(G,M)} \\
	\& \textcolor{rgb,255:red,92;green,214;blue,214}{{\mathbb E_\xi(Q,M)}}
	\arrow[from=1-1, to=1-2]
	\arrow[from=1-1, to=2-1]
	\arrow[from=1-2, to=1-3]
	\arrow[from=1-2, to=2-2]
	\arrow[from=1-3, to=2-3]
	\arrow["{(G', g)}", from=2-1, to=2-2]
	\arrow[from=2-1, to=3-1]
	\arrow[from=2-2, to=2-3]
	\arrow["{((\delta_\xi)^\sharp)_M}", from=2-2, to=3-2]
	\arrow[from=3-1, to=3-2]
	\arrow[from=3-2, to=4-2]
\end{tikzcd}.
		\end{equation}
		由于 $((\delta_\xi)^\sharp)_M \circ (G', g)$ 是零态射, 且 $(G', g)$ 是满态射, 故 $((\delta_\xi)^\sharp)_M = 0$. 由长正合列, 得 $\mathbb E_\xi(G,M) = 0$.
	\end{proof}
\end{proposition}

\begin{proposition}
	完全投射分解是 $(-,\widetilde {\mathcal{P}(\xi)})$-正合的.
	\begin{proof}
		由定义, 完全投射分解是 $(-, \mathcal{P}(\xi))$-正合的. 假定是 $(-, \mathcal{P}(\xi)^{\leq n})$-正合的, 下证明其 $(-,\mathcal{P}(\xi)^{\leq n+1})$ 正合. 沿用 \cref{prop:GP_ext_vanish} 中记号, 得正合列的交换图
		\begin{equation}
% https://q.uiver.app/#q=WzAsMTUsWzEsMSwiKFEsUCkiXSxbMSwyLCIoUSxNKSJdLFsxLDMsIlxcbWF0aGJiIEVfXFx4aShRLE4pIixbMTgwLDYwLDYwLDFdXSxbMiwxLCIoRycsUCkiXSxbMiwyLCIoRycsTSkiXSxbMywxLCJcXG1hdGhiYiBFX1xceGkoRyxQKSIsWzE4MCw2MCw2MCwxXV0sWzMsMiwiXFxtYXRoYmIgRV9cXHhpKEcsTSkiLFsxODAsNjAsNjAsMV1dLFsyLDMsIlxcbWF0aGJiIEVfXFx4aShHJyxOKSIsWzE4MCw2MCw2MCwxXV0sWzMsMCwiXFxtYXRoYmIgRV9cXHhpKEcsTikiLFsxODAsNjAsNjAsMV1dLFsyLDAsIihHJyxOKSJdLFsxLDAsIihRLE4pIl0sWzAsMCwiKEcsTikiXSxbMCwxLCIoRyxQKSJdLFswLDIsIihHLE0pIl0sWzAsMywiXFxtYXRoYmIgRV9cXHhpKEcsTikiLFsxODAsNjAsNjAsMV1dLFswLDFdLFszLDVdLFsxLDRdLFsyLDddLFs0LDddLFs1LDZdLFswLDNdLFszLDRdLFs4LDVdLFs5LDhdLFs5LDMsIihHJyxmKSJdLFsxMCw5XSxbMTAsMCwiKFEsZikiLDAseyJjb2xvdXIiOlsyNDAsNjAsNjBdLCJzdHlsZSI6eyJ0YWlsIjp7Im5hbWUiOiJtb25vIn19fSxbMjQwLDYwLDYwLDFdXSxbMTEsMTAsIihwLE4pIiwwLHsiY29sb3VyIjpbMjQwLDYwLDYwXSwic3R5bGUiOnsidGFpbCI6eyJuYW1lIjoibW9ubyJ9fX0sWzI0MCw2MCw2MCwxXV0sWzExLDEyLCIoRyxmKSJdLFsxMiwwLCIocCxQKSIsMCx7ImNvbG91ciI6WzI0MCw2MCw2MF0sInN0eWxlIjp7InRhaWwiOnsibmFtZSI6Im1vbm8ifX19LFsyNDAsNjAsNjAsMV1dLFsxMiwxM10sWzEzLDFdLFsxMywxNF0sWzQsNl0sWzE0LDJdLFsxLDJdXQ==
\begin{tikzcd}[ampersand replacement=\&]
	{(G,N)} \& {(Q,N)} \& {(G',N)} \& \textcolor{rgb,255:red,92;green,214;blue,214}{{\mathbb E_\xi(G,N)}} \\
	{(G,P)} \& {(Q,P)} \& {(G',P)} \& \textcolor{rgb,255:red,92;green,214;blue,214}{{\mathbb E_\xi(G,P)}} \\
	{(G,M)} \& {(Q,M)} \& {(G',M)} \& \textcolor{rgb,255:red,92;green,214;blue,214}{{\mathbb E_\xi(G,M)}} \\
	\textcolor{rgb,255:red,92;green,214;blue,214}{{\mathbb E_\xi(G,N)}} \& \textcolor{rgb,255:red,92;green,214;blue,214}{{\mathbb E_\xi(Q,N)}} \& \textcolor{rgb,255:red,92;green,214;blue,214}{{\mathbb E_\xi(G',N)}}
	\arrow["{(p,N)}", color={rgb,255:red,92;green,92;blue,214}, tail, from=1-1, to=1-2]
	\arrow["{(G,f)}", from=1-1, to=2-1]
	\arrow[from=1-2, to=1-3]
	\arrow["{(Q,f)}", color={rgb,255:red,92;green,92;blue,214}, tail, from=1-2, to=2-2]
	\arrow[from=1-3, to=1-4]
	\arrow["{(G',f)}", from=1-3, to=2-3]
	\arrow[from=1-4, to=2-4]
	\arrow["{(p,P)}", color={rgb,255:red,92;green,92;blue,214}, tail, from=2-1, to=2-2]
	\arrow[from=2-1, to=3-1]
	\arrow[from=2-2, to=2-3]
	\arrow[from=2-2, to=3-2]
	\arrow[from=2-3, to=2-4]
	\arrow[from=2-3, to=3-3]
	\arrow[from=2-4, to=3-4]
	\arrow[from=3-1, to=3-2]
	\arrow[from=3-1, to=4-1]
	\arrow[from=3-2, to=3-3]
	\arrow[from=3-2, to=4-2]
	\arrow[from=3-3, to=3-4]
	\arrow[from=3-3, to=4-3]
	\arrow[from=4-1, to=4-2]
	\arrow[from=4-2, to=4-3]
\end{tikzcd}.
		\end{equation}
		由态射合成, $(G,f)$-单. 对 $G' \to P' \to G'' \dashrightarrow$ 作同样的交换图, 得 $(G',f)$-单. 上图包含五条短正合列, 由 $3\times 3$ 引理得第六条短正合列, 这完成了归纳假设.
	\end{proof}
\end{proposition}

\begin{corollary}
	由维数移动知 $\widetilde {\mathcal{P}(\xi)}$ 是厚子范畴. 所有 $\widetilde {\mathcal{P}(\xi)}$ 中的 $\mathbb E$-三角都是 $(\mathcal{GP}(\xi),-)$-正合的.
\end{corollary}

\begin{proposition}
	给定 $\xi$-三角 $A \to B \to C \dashrightarrow$, 其中 $C \in \mathcal{GP}(\xi)$. \Cref{thm:GP_W_exact} 证明了该 $\mathbb E$-三角是 $(-, \mathcal{P}(\xi))$-正合的. 实际上, 该 $\mathbb E$-三角也是 $(-, \widetilde{\mathcal{P}(\xi)})$-正合的.
	\begin{proof}
		依照与 \cref{thm:GP_W_exact} 一致的证明思路, 只需说明完全投射分解 $(-, \widetilde{\mathcal{P}(\xi)})$-正合, 即上一命题.
	\end{proof}
\end{proposition}

\subsection{Hovey 四元组}

\begin{lemma}
	约定 $\mathcal{P}(\xi)^{-1} = \{0\}$. 对 $n\geq 0$, 有如下等式:
	\begin{equation}
			\mathcal{GP} (\xi)^{\leq n} \subseteq \mathrm{Cone}(\mathcal{P}(\xi)^{\leq n-1}, \mathcal{GP}(\xi)), \quad \mathcal{GP} (\xi)^{\leq n} \subseteq \mathrm{coCone}( \mathcal{P}(\xi)^{\leq n}, \mathcal{GP}(\xi)).
		\end{equation}
		\begin{proof}
			下归纳证明. 上述命题对 $\mathcal{GP}(\xi)^{\leq 0} = \mathcal{GP}(\xi)$ 显然成立. 假定上述命题对 $\mathcal{GP}(\xi)^{n}$ 成立, 下证明其对 $\mathcal{GP}(\xi)^{n+1}$ 成立.
			\begin{enumerate}
				\item (前式). 取定 $M \in \mathcal{GP}(\xi)^{\leq n+1}$, 则存在 $\xi$-三角的交换图
				\begin{equation}
					% https://q.uiver.app/#q=WzAsMTUsWzIsMCwiTSJdLFsxLDAsIlAiXSxbMCwwLCJMIl0sWzQsMCwiXFxtYXRoY2FsIEdwZF9cXHhpIE0gPSBuKzEiXSxbNCwxLCJcXG1hdGhjYWwgR3BkX1xceGkgTCA9IG4iXSxbMywwLCJcXCwiXSxbMCwxLCJBIl0sWzAsMiwiRyJdLFs0LDIsInBkX1xceGkgQSBcXGxlcSBuIl0sWzEsMiwiRyJdLFsyLDEsIk0iXSxbMSwxLCJIIl0sWzAsMywiXFwsIl0sWzEsMywiXFwsIl0sWzMsMSwiXFwsIl0sWzIsMV0sWzEsMF0sWzAsNSwiIiwwLHsic3R5bGUiOnsiYm9keSI6eyJuYW1lIjoiZGFzaGVkIn19fV0sWzcsOSwiIiwwLHsibGV2ZWwiOjIsInN0eWxlIjp7ImhlYWQiOnsibmFtZSI6Im5vbmUifX19XSxbMCwxMCwiIiwwLHsibGV2ZWwiOjIsInN0eWxlIjp7ImhlYWQiOnsibmFtZSI6Im5vbmUifX19XSxbMiw2XSxbNiw3XSxbNywxMiwiIiwxLHsic3R5bGUiOnsiYm9keSI6eyJuYW1lIjoiZGFzaGVkIn19fV0sWzEwLDE0LCIiLDAseyJzdHlsZSI6eyJib2R5Ijp7Im5hbWUiOiJkYXNoZWQifX19XSxbOSwxMywiIiwwLHsic3R5bGUiOnsiYm9keSI6eyJuYW1lIjoiZGFzaGVkIn19fV0sWzYsMTEsIiIsMCx7InN0eWxlIjp7ImJvZHkiOnsibmFtZSI6ImRhc2hlZCJ9fX1dLFsxMSwxMCwiIiwwLHsic3R5bGUiOnsiYm9keSI6eyJuYW1lIjoiZGFzaGVkIn19fV0sWzEsMTEsIiIsMSx7InN0eWxlIjp7ImJvZHkiOnsibmFtZSI6ImRhc2hlZCJ9fX1dLFsxMSw5LCIiLDEseyJzdHlsZSI6eyJib2R5Ijp7Im5hbWUiOiJkYXNoZWQifX19XV0=
\begin{tikzcd}[ampersand replacement=\&]
	L \& P \& M \& {\,} \& {\mathcal Gpd_\xi M = n+1} \\
	A \& H \& M \& {\,} \& {\mathcal Gpd_\xi L = n} \\
	G \& G \&\&\& {pd_\xi A \leq n} \\
	{\,} \& {\,}
	\arrow[from=1-1, to=1-2]
	\arrow[from=1-1, to=2-1]
	\arrow[from=1-2, to=1-3]
	\arrow[dashed, from=1-2, to=2-2]
	\arrow[dashed, from=1-3, to=1-4]
	\arrow[equals, from=1-3, to=2-3]
	\arrow[dashed, from=2-1, to=2-2]
	\arrow[from=2-1, to=3-1]
	\arrow[dashed, from=2-2, to=2-3]
	\arrow[dashed, from=2-2, to=3-2]
	\arrow[dashed, from=2-3, to=2-4]
	\arrow[equals, from=3-1, to=3-2]
	\arrow[dashed, from=3-1, to=4-1]
	\arrow[dashed, from=3-2, to=4-2]
\end{tikzcd}.
				\end{equation}
				因此 $\mathcal{GP}(\xi)^{\leq n+1} \subseteq \mathrm{Cone}(\mathcal{P}(\xi)^{\leq n}, \mathcal{GP}(\xi))$.
				\item (后式). 对上述 $M$, 作下图:
				\begin{equation}
					% https://q.uiver.app/#q=WzAsMTgsWzIsMCwiTSJdLFsxLDAsIlAiXSxbMCwwLCJMIl0sWzQsMCwiXFxtYXRoY2FsIEdwZF9cXHhpIE0gPSBuKzEiXSxbNCwxLCJcXG1hdGhjYWwgR3BkX1xceGkgTCA9IG4iXSxbMywwLCJcXCwiXSxbMCwxLCJBIl0sWzAsMiwiRyJdLFs0LDIsInBkX1xceGkgQSBcXGxlcSBuIl0sWzEsMiwiUSJdLFsyLDEsIk4iXSxbMSwxLCJQIFxcb3BsdXMgUSJdLFswLDMsIlxcLCJdLFsxLDMsIlxcLCJdLFszLDEsIlxcLCJdLFsyLDIsIkcnIl0sWzIsMywiXFwsIl0sWzMsMiwiXFwsIl0sWzIsMV0sWzEsMF0sWzAsNSwiIiwwLHsic3R5bGUiOnsiYm9keSI6eyJuYW1lIjoiZGFzaGVkIn19fV0sWzIsNl0sWzYsN10sWzcsMTIsIiIsMSx7InN0eWxlIjp7ImJvZHkiOnsibmFtZSI6ImRhc2hlZCJ9fX1dLFsxMCwxNCwiIiwwLHsic3R5bGUiOnsiYm9keSI6eyJuYW1lIjoiZGFzaGVkIn19fV0sWzksMTMsIjAiLDAseyJzdHlsZSI6eyJib2R5Ijp7Im5hbWUiOiJkYXNoZWQifX19XSxbNiwxMSwiIiwwLHsic3R5bGUiOnsiYm9keSI6eyJuYW1lIjoiZGFzaGVkIn19fV0sWzExLDEwLCIiLDAseyJzdHlsZSI6eyJib2R5Ijp7Im5hbWUiOiJkYXNoZWQifX19XSxbMSwxMSwiIiwxLHsic3R5bGUiOnsiYm9keSI6eyJuYW1lIjoiZGFzaGVkIn19fV0sWzExLDksIiIsMSx7InN0eWxlIjp7ImJvZHkiOnsibmFtZSI6ImRhc2hlZCJ9fX1dLFs3LDldLFs5LDE1XSxbMCwxMCwiIiwxLHsic3R5bGUiOnsiYm9keSI6eyJuYW1lIjoiZGFzaGVkIn19fV0sWzEwLDE1LCIiLDEseyJzdHlsZSI6eyJib2R5Ijp7Im5hbWUiOiJkYXNoZWQifX19XSxbMTUsMTYsIiIsMSx7InN0eWxlIjp7ImJvZHkiOnsibmFtZSI6ImRhc2hlZCJ9fX1dLFsxNSwxNywiIiwxLHsic3R5bGUiOnsiYm9keSI6eyJuYW1lIjoiZGFzaGVkIn19fV1d
\begin{tikzcd}[ampersand replacement=\&]
	L \& P \& M \& {\,} \& {\mathcal Gpd_\xi M = n+1} \\
	A \& {P \oplus Q} \& N \& {\,} \& {\mathcal Gpd_\xi L = n} \\
	G \& Q \& {G'} \& {\,} \& {pd_\xi A \leq n} \\
	{\,} \& {\,} \& {\,}
	\arrow[from=1-1, to=1-2]
	\arrow[from=1-1, to=2-1]
	\arrow[from=1-2, to=1-3]
	\arrow[dashed, from=1-2, to=2-2]
	\arrow[dashed, from=1-3, to=1-4]
	\arrow[dashed, from=1-3, to=2-3]
	\arrow[dashed, from=2-1, to=2-2]
	\arrow[from=2-1, to=3-1]
	\arrow[dashed, from=2-2, to=2-3]
	\arrow[dashed, from=2-2, to=3-2]
	\arrow[dashed, from=2-3, to=2-4]
	\arrow[dashed, from=2-3, to=3-3]
	\arrow[from=3-1, to=3-2]
	\arrow[dashed, from=3-1, to=4-1]
	\arrow[from=3-2, to=3-3]
	\arrow["0", dashed, from=3-2, to=4-2]
	\arrow[dashed, from=3-3, to=3-4]
	\arrow[dashed, from=3-3, to=4-3]
\end{tikzcd}.
				\end{equation}
				由维数移动, $pd_\xi N \leq n+1$. 因此 $\mathcal{GP}(\xi)^{\leq n+1} \subseteq \mathrm{coCone}( \mathcal{P}(\xi)^{\leq n+1}, \mathcal{GP}(\xi))$.
			\end{enumerate}
		\end{proof}
\end{lemma}

\begin{corollary}
	若 $(\mathcal{C}, \mathbb E_\xi, \mathfrak s_\xi)$ 中所有对象有有限 Gorenstein 投射维数, 则 $(\mathcal{GP}(\xi), \widetilde{\mathcal{P}(\xi)})$ 是完备余挠对.
\end{corollary}

\begin{theorem}
	$((\mathcal{P}(\xi),\mathcal{C}), (\mathcal{GP}(\xi), \widetilde{\mathcal{P}(\xi)}))$ 是 Hovey 四元组.
	\begin{proof}
		已证这是两个余挠对. 容易验证
		\begin{equation}
			\mathrm{Cone}(\widetilde{\mathcal{P}(\xi)}, \mathcal{P}(\xi)) = \mathrm{coCone}(\widetilde{\mathcal{P}(\xi)}, \mathcal{P}(\xi)) = \widetilde{\mathcal{P}(\xi)}.
		\end{equation}
		从而这是 Hovey 四元组.
	\end{proof}
\end{theorem}

\begin{corollary}
	若 $(\mathcal{C}, \mathbb E_\xi, \mathfrak s_\xi)$ 的 Gorenstein 投射维数 $\leq n$, 则 $\widetilde {P(\xi)} = \mathcal{P}(\xi)^{\leq n}$. 此时 $((\mathcal{P}(\xi),\mathcal{C}),(\mathcal{GP}(\xi), \widetilde{\mathcal{P}(\xi)}))$ 是 Hovey 四元组.
\end{corollary}

