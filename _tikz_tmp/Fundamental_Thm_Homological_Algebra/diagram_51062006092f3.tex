\documentclass[preview]{standalone}
\usepackage{amsthm, amsfonts, amsmath, amssymb, mathrsfs, tikz-cd, tikz, ctex, mathtools}
\usepackage{quiver}
\begin{document}

   % https://q.uiver.app/#q=WzAsOCxbMSwwLCJcXGtlcl5wIl0sWzEsMiwiXFxtYXRocm17Y29rfV5wIl0sWzIsMCwiXFxrZXJee3ArMX0iXSxbMiwyLCJcXG1hdGhybXtjb2t9XntwKzF9Il0sWzAsMiwiXFxtYXRocm17Y29rfV57cC0xfSJdLFszLDAsIlxca2VyXntwKzJ9Il0sWzMsMiwiXFxtYXRocm17Y29rfV57cCsyfSJdLFswLDAsIlxca2VyXntwLTF9Il0sWzQsMl0sWzEsNV0sWzAsNCwiSF57cC0xfShFKSIsMSx7InN0eWxlIjp7ImJvZHkiOnsibmFtZSI6ImRvdHRlZCJ9fX1dLFsyLDEsIkhee3B9KEUpIiwxLHsic3R5bGUiOnsiYm9keSI6eyJuYW1lIjoiZG90dGVkIn19fV0sWzUsMywiSF57cCsxfShFKSIsMSx7InN0eWxlIjp7ImJvZHkiOnsibmFtZSI6ImRvdHRlZCJ9fX1dXQ==
\begin{tikzcd}
	{\ker^{p-1}} & {\ker^p} & {\ker^{p+1}} & {\ker^{p+2}} \\
	\\
	{\mathrm{cok}^{p-1}} & {\mathrm{cok}^p} & {\mathrm{cok}^{p+1}} & {\mathrm{cok}^{p+2}}
	\arrow["{H^{p-1}(E)}"{description}, dotted, from=1-2, to=3-1]
	\arrow["{H^{p}(E)}"{description}, dotted, from=1-3, to=3-2]
	\arrow["{H^{p+1}(E)}"{description}, dotted, from=1-4, to=3-3]
	\arrow[from=3-1, to=1-3]
	\arrow[from=3-2, to=1-4]
\end{tikzcd}
   \end{document}
